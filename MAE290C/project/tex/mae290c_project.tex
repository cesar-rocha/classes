\documentclass[11pt]{article}


%% WRY has commented out some unused packages %%
%% If needed, activate these by uncommenting
\usepackage{geometry}                % See geometry.pdf to learn the layout options. There are lots.
%\geometry{letterpaper}                   % ... or a4paper or a5paper or ... 
\geometry{a4paper,left=2.5cm,right=2.5cm,top=2.5cm,bottom=2.5cm}
%\geometry{landscape}                % Activate for rotated page geometry
%\usepackage[parfill]{parskip}    % Activate to begin paragraphs with an empty line rather than an indent

%for figures
%\usepackage{graphicx}

\usepackage{color}
\definecolor{mygreen}{RGB}{28,172,0} % color values Red, Green, Blue
\definecolor{mylilas}{RGB}{170,55,241}
%% for graphics this one is also OK:
\usepackage{epsfig}

%% AMS mathsymbols are enabled with
\usepackage{amssymb,amsmath}

%% more options in enumerate
\usepackage{enumerate}
\usepackage{enumitem}

%% insert code
\usepackage{listings}

\usepackage[utf8]{inputenc}

\usepackage{hyperref}

%% To make really wide whats that cover everything:
\usepackage{scalerel}
\usepackage{stackengine}
\stackMath
\def\hatgap{2pt}
\def\subdown{-2pt}
\newcommand\what[2][]{%
\renewcommand\stackalignment{l}%
\stackon[\hatgap]{#2}{%
\stretchto{%
    \scalerel*[\widthof{$#2$}]{\kern-.6pt\bigwedge\kern-.6pt}%
    {\rule[-\textheight/2]{1ex}{\textheight}}%WIDTH-LIMITED BIG WEDGE
}{0.5ex}% THIS SQUEEZES THE WEDGE TO 0.5ex HEIGHT
_{\smash{\belowbaseline[\subdown]{\scriptstyle#1}}}%
}}

% Default fixed font does not support bold face
\DeclareFixedFont{\ttb}{T1}{txtt}{bx}{n}{12} % for bold
\DeclareFixedFont{\ttm}{T1}{txtt}{m}{n}{12}  % for normal

% Custom colors
\usepackage{color}
\definecolor{deepblue}{rgb}{0,0,0.5}
\definecolor{deepred}{rgb}{0.6,0,0}
\definecolor{deepgreen}{rgb}{0,0.5,0}


% Python style for highlighting
\newcommand\pythonstyle{\lstset{
language=Python,
basicstyle=\ttm,
otherkeywords={self},             % Add keywords here
keywordstyle=\ttb\color{deepblue},
emph={MyClass,__init__},          % Custom highlighting
emphstyle=\ttb\color{deepred},    % Custom highlighting style
stringstyle=\color{deepgreen},
frame=tb,                         % Any extra options here
showstringspaces=false            % 
}}

% Python environment
\lstnewenvironment{python}[1][]
{
\pythonstyle
\lstset{#1}
}
{}

% Python for external files
\newcommand\pythonexternal[2][]{{
\pythonstyle
\lstinputlisting[#1]{#2}}}

% Python for inline
\newcommand\pythoninline[1]{{\pythonstyle\lstinline!#1!}}

%% colors
\usepackage{graphicx,xcolor,lipsum}


\usepackage{mathtools}

\usepackage{graphicx}
\newcommand*{\matminus}{%
  \leavevmode
  \hphantom{0}%
  \llap{%
    \settowidth{\dimen0 }{$0$}%
    \resizebox{1.1\dimen0 }{\height}{$-$}%
  }%
}


\title{MAE290C, Final Project}
\author{Cesar B Rocha}
\date{\today}

\begin{document}

\include{mysymbols}
\maketitle

\section*{Preamble}
The equation to solve is

\beq
\label{eq:vort_eqn}
\p_t \zeta + \sJ\left( \psi, \zeta \right ) = \frac{1}{\Re} \nabla^2 \zeta\com
\eeq
where
\beq
\label{eq:defns}
\zeta = \nabla^2\psi \com\qquad u \defn -\p_y \psi\com\qqand v \defn -\p_x \psi\per
\eeq
Fourier transforming \eqref{eq:vort_eqn} we obtain
\beq
\label{eq:vort_eqn_hat}
\p_t \what{\zeta} + \what{\sJ\left( \psi, \zeta \right )} = -\frac{\kappa^2}{\Re} \what{\zeta}\com
\eeq
where, in a pseudo-spectral spirit, we indicated the Fourier transform of the whole Jacobian instead
 of  writing the convolution sums, i.e., we evaluate products in physical space, and then transform
 the Jacobian to Fourier space. Also, in \eqref{eq:vort_eqn_hat}, $\kappa\defn \sqrt{k^2 + l^2}$ is 
 the isotropic wavenumber. The elliptic problem that relates vorticity and streamfunction is diagonal
  in Fourier space:
\beq
\label{eq:inv_hat}
\what{\zeta} = -\kappa^2 \what{\psi}\per
\eeq
Of course, the inversion is not defined for $\kappa = 0$.  We step \eqref{eq:vort_eqn_hat}-\eqref{eq:inv_hat} forward  in a $2\pi\times 2\pi$ box using a RK3W-$\theta$ scheme. The nonlinear term is fully dealiased using the $\tfrac{2}{3}$ rule \cite{orszag1971}. With $\nmax = 1024$, the effective number of resolved modes is $\nmax_e = 682$.  
The initial condition is a random field, with a von Karman-like target isotropic spectrum \cite{mcwilliams1984}

\beq
\label{eq:init_spec}
|\what{q}_i|^2 = A\,\kappa^2\left[1 + \left(\frac{\kappa}{6}\right)^2\right]^{-2} \per
\eeq
The initial spectrum \eqref{eq:init_spec} peaks near $6$, and the constant $A$ is chosen so that
 the initial kinetic energy is unitary. This entails that the velocity of the
 energy containing eddies is $O(1)$, thereby guaranteeing that the nondimensionalization in equation
 \eqref{eq:vort_eqn} holds. The kinetic energy is 
 \beq
 \label{eq:ke}
 E = \tfrac{1}{2} \sum_{k}\sum_{l} \kappa^2 |\psi^2|\per
 \eeq
Thus
\beq
\label{const}
A = \left[\sum_{k}\sum_{l}\left[1 + \left(\frac{\kappa}{6}\right)^2\right]^{-2} \right]^{-1}\per
\eeq

\section*{Diagnostics}
Besides the kinetic energy in \eqref{eq:ke}, we also diagnose the enstrophy
\beq
    Z =\tfrac{1}{2} \sum_{k}\sum_{l} \kappa^4 |\psi^2|\com
\eeq
and the CFL condition
\beq
\text{CFL} = \frac{u_{max} \dt}{\dx}\com
\eeq
where 
\beq
\dx \defn \frac{2\pi}{\nmax}\per
\eeq
We chose the time-step so the initial CFL is $0.25$.


%The non-dimensional Navier-Stokes equations, with constant density, are
%    \begin{figure}[ht]
%    \begin{center}
%    \includegraphics[width=10pc,angle=0]{figs/errrand_0.png} 
%        \includegraphics[width=10pc,angle=0]{figs/errrand_1.png}  
%        \includegraphics[width=10pc,angle=0]{figs/errrand_2.png}\\  
%        \includegraphics[width=10pc,angle=0]{figs/errrand_3.png} 
%        \includegraphics[width=10pc,angle=0]{figs/errrand_4.png}  
%        \includegraphics[width=10pc,angle=0]{figs/errrand_5.png}  
%    \end{center}
%    \caption{The residual for the first 600 iterations with random $\dt$ between 0 and 2000.}
%    \label{fig:exps2}
%    \end{figure}


\bibliographystyle{ametsoc2014}
\bibliography{mae290c.bib}

\end{document}
