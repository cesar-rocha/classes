\documentclass[11pt]{article}


%% WRY has commented out some unused packages %%
%% If needed, activate these by uncommenting
\usepackage{geometry}                % See geometry.pdf to learn the layout options. There are lots.
%\geometry{letterpaper}                   % ... or a4paper or a5paper or ... 
\geometry{a4paper,left=2.5cm,right=2.5cm,top=2.5cm,bottom=2.5cm}
%\geometry{landscape}                % Activate for rotated page geometry
%\usepackage[parfill]{parskip}    % Activate to begin paragraphs with an empty line rather than an indent

%for figures
%\usepackage{graphicx}

\usepackage{color}
\definecolor{mygreen}{RGB}{28,172,0} % color values Red, Green, Blue
\definecolor{mylilas}{RGB}{170,55,241}
%% for graphics this one is also OK:
\usepackage{epsfig}

%% AMS mathsymbols are enabled with
\usepackage{amssymb,amsmath}

%% more options in enumerate
\usepackage{enumerate}
\usepackage{enumitem}

%% insert code
\usepackage{listings}

\usepackage[utf8]{inputenc}

\usepackage{hyperref}

%% To make really wide whats that cover everything:
\usepackage{scalerel}
\usepackage{stackengine}
\stackMath
\def\hatgap{2pt}
\def\subdown{-2pt}
\newcommand\what[2][]{%
\renewcommand\stackalignment{l}%
\stackon[\hatgap]{#2}{%
\stretchto{%
    \scalerel*[\widthof{$#2$}]{\kern-.6pt\bigwedge\kern-.6pt}%
    {\rule[-\textheight/2]{1ex}{\textheight}}%WIDTH-LIMITED BIG WEDGE
}{0.5ex}% THIS SQUEEZES THE WEDGE TO 0.5ex HEIGHT
_{\smash{\belowbaseline[\subdown]{\scriptstyle#1}}}%
}}

% Default fixed font does not support bold face
\DeclareFixedFont{\ttb}{T1}{txtt}{bx}{n}{12} % for bold
\DeclareFixedFont{\ttm}{T1}{txtt}{m}{n}{12}  % for normal

% Custom colors
\usepackage{color}
\definecolor{deepblue}{rgb}{0,0,0.5}
\definecolor{deepred}{rgb}{0.6,0,0}
\definecolor{deepgreen}{rgb}{0,0.5,0}


% Python style for highlighting
\newcommand\pythonstyle{\lstset{
language=Python,
basicstyle=\ttm,
otherkeywords={self},             % Add keywords here
keywordstyle=\ttb\color{deepblue},
emph={MyClass,__init__},          % Custom highlighting
emphstyle=\ttb\color{deepred},    % Custom highlighting style
stringstyle=\color{deepgreen},
frame=tb,                         % Any extra options here
showstringspaces=false            % 
}}

% Python environment
\lstnewenvironment{python}[1][]
{
\pythonstyle
\lstset{#1}
}
{}

% Python for external files
\newcommand\pythonexternal[2][]{{
\pythonstyle
\lstinputlisting[#1]{#2}}}

% Python for inline
\newcommand\pythoninline[1]{{\pythonstyle\lstinline!#1!}}

%% colors
\usepackage{graphicx,xcolor,lipsum}


\usepackage{mathtools}

\usepackage{graphicx}
\newcommand*{\matminus}{%
  \leavevmode
  \hphantom{0}%
  \llap{%
    \settowidth{\dimen0 }{$0$}%
    \resizebox{1.1\dimen0 }{\height}{$-$}%
  }%
}


\title{MAE290C, Homework Assignment 4}
\author{Cesar B Rocha}
\date{\today}

\begin{document}

\newcommand{\com}{\, ,}
\newcommand{\per}{\, .}

%% Averages
% Use \bar to over line solo symbols

\newcommand{\av}[1]{\bar{#1}}
\newcommand{\avbg}[1]{\overline{#1}}
\newcommand{\avbgg}[1]{\overline{#1}}

% A nice definition
\newcommand{\defn}{\ensuremath{\stackrel{\mathrm{def}}{=}}}

% equations
\def\beq{\begin{equation}}
\def\eeq{\end{equation}}

% calculus

\newcommand{\p}{\partial}
\newcommand{\ii}{{\rm i}}
\newcommand{\dd}{{\rm d}}
\newcommand{\id}{{\, \rm d}}
\newcommand{\ee}{{\rm e}}
\newcommand{\DD}{{\rm D}}
\newcommand{\wavy}{\text{wavy}}
\newcommand{\qg}{\text{qg}}


\newcommand{\be}{\beta}

\newcommand{\al}{\alpha}
\newcommand{\bx}{\boldsymbol{x}}
\newcommand{\by}{\boldsymbol{y}}
\newcommand{\bu}{\boldsymbol{u}}
\newcommand{\bv}{\boldsymbol{v}}


\newcommand{\half}{\tfrac{1}{2}}
\newcommand{\halfrho}{\tfrac{1}{2}}
\newcommand{\rz}{{}}
\newcommand{\bn}{\boldsymbol{\hat n}}
\newcommand{\br}{\boldsymbol{r}}
\newcommand{\bR}{\boldsymbol{R}}
\newcommand{\bA}{\ensuremath {\boldsymbol {A}}}
\newcommand{\bB}{\ensuremath {\boldsymbol {B}}}
\newcommand{\bU}{\ensuremath {\boldsymbol {U}}}
\newcommand{\bE}{\ensuremath {\boldsymbol {E}}}
\newcommand{\bJ}{\ensuremath {\boldsymbol {J}}}
\newcommand{\bXX}{\ensuremath {\boldsymbol {\mathcal{X}}}}
\newcommand{\bFF}{\ensuremath {\boldsymbol {F}}}
\newcommand{\bF}{\ensuremath {\boldsymbol {F}^{\sharp}}}
\newcommand{\bG}{\ensuremath {\boldsymbol G}}
\newcommand{\bSigma}{\ensuremath {\boldsymbol {\Sigma}}}
\newcommand{\bvarphi}{\ensuremath {\boldsymbol {\varphi}}}
\newcommand{\bxi}{\ensuremath {\boldsymbol {\xi}}}
\newcommand{\avbxi}{\overline{\ensuremath {\boldsymbol {\xi}}}}

% math cal

\newcommand{\J}{\mathcal{J}}
\newcommand{\K}{\mathcal{K}}
\newcommand{\cG}{\mathcal{G}}
\newcommand{\cF}{\mathcal{F}}
\newcommand{\cN}{\mathcal{N}}
\newcommand{\cL}{\mathcal{L}}


% san serif for matrices and differential operators
%\newcommand{\helmn}{\mathsf{H}_n}
\newcommand{\helmm}{\triangle_m}
\newcommand{\helmn}{\triangle_n}
\newcommand{\helms}{\triangle_s}
\newcommand{\helm}{\triangle}
\newcommand{\sA}{\mathsf{A}}
\newcommand{\sB}{\mathsf{B}}
\newcommand{\sG}{\mathsf{G}}
\newcommand{\sI}{\mathsf{I}}
\newcommand{\sJ}{\mathsf{J}}
\newcommand{\sU}{\mathsf{U}}
\newcommand{\sP}{\mathsf{P}}
\newcommand{\sQ}{\mathsf{Q}}
\newcommand{\sR}{\mathsf{R}}
\newcommand{\sL}{\mathsf{L}}
\renewcommand{\sJ}{\mathsf{J}}
\renewcommand{\sI}{\mathsf{I}}
\renewcommand{\L}{\mathsf{L}}
\newcommand{\sM}{\mathsf{M}}
\newcommand{\sT}{\mathsf{T}}
\newcommand{\sGamma}{\mathsf{\Gamma}}
\newcommand{\sOmega}{\mathsf{\Omega}}
\newcommand{\sSigma}{\mathsf{\Omega}}
\newcommand{\sbeta}{\mathsf{\beta}}
\newcommand{\sPi}{\mathsf{\Pi}}
\newcommand{\sC}{\mathsf{C}}
\newcommand{\sQy}{\mathsf{Q}}


% angle brackets

\def\la{\langle}
\def\ra{\rangle}
\def\laa{\left \langle}
\def\raa{\right \rangle}


%grads and div's
\newcommand{\bcdot}{\hspace{-0.1em} \boldsymbol{\cdot} \hspace{-0.12em}}
\newcommand{\bnabla}{\boldsymbol{\nabla}}
\newcommand{\bnablaH}{\bnabla_{\! \mathrm{h}}}
\newcommand{\grad}{\bnabla}
\newcommand{\gradH}{\bnablaH}
\newcommand{\curl}{\bnabla \!\times\!}
\newcommand{\diver}{\bnabla \bcdot }
\newcommand{\cross}{\times}
\newcommand{\lap}{\triangle}


%varthetas and thetas
\newcommand{\vth}{\vartheta}
\newcommand{\psii}{\psi^{\mathrm{i}}}
\newcommand{\thb}{\theta^{\mathrm{-}}}
\newcommand{\vthb}{\vartheta^{\mathrm{-}}}
\newcommand{\vthbhat}{{\hat{\vartheta}}^{\mathrm{-}}}
\newcommand{\vThb}{\varTheta^{\mathrm{-}}}
\newcommand{\psib}{\psi^{\mathrm{-}}}
\newcommand{\tht}{\theta^{\mathrm{+}}}
\newcommand{\vtht}{\vartheta^{\mathrm{+}}}
\newcommand{\vththat}{{\hat{\vartheta}}^{\mathrm{+}}}
\newcommand{\vthtbhat}{{\hat{\vartheta}}^{\pm}}
\newcommand{\vTht}{\varTheta^{\mathrm{+}}}
\newcommand{\vthtb}{\vartheta^{\pm}}
\newcommand{\vThtb}{\varTheta^{\pm}}

% nondimensional numbers
\renewcommand{\Re}{\mathrm{Re}}
\newcommand{\Ro}{\mathrm{Ro}}
\newcommand{\Ri}{\mathrm{Ri}}

%psi's
%Galerking coefficient for psi:
\newcommand{\gpsi}{\breve \psi}
\newcommand{\gphi}{\breve \phi}
\newcommand{\gq}{\breve q}
\newcommand{\gU}{\breve U}
\newcommand{\gQ}{\breve Q}
\newcommand{\gsigma}{\breve \sigma}


\newcommand{\psit}{\psi^{\mathrm{+}}}
\newcommand{\psithat}{{\hat{\psi}}^{\mathrm{+}}}
\newcommand{\psibhat}{{\hat{\psi}}^{\mathrm{-}}}
\newcommand{\psitb}{\psi^{\pm}}
\newcommand{\psitbhat}{{\hat{\psi}}^\pm}
\newcommand{\St}{S^{\mathrm{+}}}
\newcommand{\Sb}{S^{\mathrm{-}}}
\newcommand{\phb}{\phi^{\mathrm{-}}}
\newcommand{\pht}{\phi^{\mathrm{+}}}
\newcommand{\tautb}{\tau^{\pm}}
\newcommand{\sigmatb}{\sigma^{\pm}}


\newcommand{\bur}{\left(\tfrac{f_0}{N}\right)^2}
\newcommand{\ibur}{\left(\tfrac{N}{f_0}\right)^2}
\newcommand{\Nm}{N_{\mathrm{mix}}}
\newcommand{\xim}{\xi_{\mathrm{mix}}}
\newcommand{\hs}{h_*}
\renewcommand{\sp}{\mathsf{p}}
\newcommand{\se}{\mathsf{e}}
\newcommand{\sptb}{\mathsf{p}^\pm}


%nmax is a problem:
%\newcommand{\nmax}{n_{\mathrm{max}}}
\newcommand{\nmax}{\mathrm{N}}

\newcommand{\WKB}{\mathrm{WKB}}
\newcommand{\Lam}{\Lambda}
\newcommand{\tha}{\theta}
\newcommand{\kap}{\kappa}
\newcommand{\bphi}{\boldsymbol{\phi}}
\newcommand{\third}{\tfrac{1}{3}}
\newcommand{\cs}{c^\star}
\newcommand{\nt}{n^{\mathrm{trnc}}}
\newcommand{\sDp}{\mathsf{D}^1_{\nmax}}
\newcommand{\sDpp}{\mathsf{D}^2_{\nmax}}
\newcommand{\sD}{\mathsf{D_2}}
\newcommand{\sK}{\mathsf{K_2}}
\newcommand{\stheta}{\mathsf{\theta}}
\newcommand{\sphi}{\mathsf{\phi}}
\newcommand{\sq}{\mathsf{q}}
\newcommand{\cosech}{\text{csch}\,}
\newcommand{\sinc}{\text{sinc}\,}

%%%%%%%%% %%%%
\newcommand{\zp}{z^+}
\newcommand{\zm}{z^-}
\newcommand{\qA}{q^A_{\nmax}}
\newcommand{\psiB}{\psi^B_{\nmax}}
\newcommand{\phiB}{\phi^B_{\nmax}}
\newcommand{\eye}{\boldsymbol{\hat{i}}}
\newcommand{\jay}{\boldsymbol{\hat{j}}}
\newcommand{\kay}{\boldsymbol{\hat{k}}}
\newcommand{\psiG}{\psi^{\mathrm{G}}}
\newcommand{\qG}{q^{\mathrm{G}}}
\newcommand{\uG}{u^{\mathrm{G}}}
\newcommand{\UG}{U^{\mathrm{G}}}
\newcommand{\UGN}{U^{\mathrm{G}}_{\nmax}}
\newcommand{\QGN}{Q^{\mathrm{G}}_{\nmax}}
\newcommand{\sumoddn}{\sum_{n = 1, n~ \text{odd}}^{\nmax}}

% bretherton 
\newcommand{\qBr}{q_{\mathrm{Br}}}
\newcommand{\psiBr}{\psi_{\mathrm{Br}}}


\maketitle

Discretizing the linear wave equation with a second-oder scheme in space we obtain
\beq
\p_t u_i = -c \frac{u_{i+1}-u_{i-1}}{2 \dx}\per
\eeq
We integrate this equation on the interval $0\leq x < 10$ with $\nmax = 200$ grid points 
and $\dt = 0.01$ with physical boundary condition of $u(x=0,t) = \sin A t$. We use a RK3 scheme
 for time stepping.

To study the reflection of different boundary conditions, we introduce the model solution for 
propagating waves near boundary
\beq
\label{model_eqn}
u_{\nmax-j} = \ee^{\ii \omega t}\left(\ee^{(\ii k \dx)j} + r \ee^{\ii (\pi - k \dx)j}\right)\com
\eeq
where $r$ is the reflection factor. That is, the solution above consists of physical waves, and a 
reflected, spurious waves. We now evaluate $r$ for different boundary conditions. Of course, the
optimal boundary condition minimizes $r$.

\begin{enumerate}

    \item Homogeneous Dirichlet: $u_\nmax = 0$.
        From \eqref{model_eqn}, with $j=0$, we have $r = -1$. That is, this boundary condition
        implies total reflection, regardless the value of $k$. This is expected, since imposing
        $u_\nmax=0$ physical means that there is a physical, closed boundary on the right boundary
        of the computational domain. 
   
    \item One-sided, backward, linear extrapolation: $u_\nmax = 2 u_{\nmax-1} - u_{\nmax-2}$. From \eqref{model_eqn}
         with $j=0$, $j=1$, and $j=2$, we obtain
        \beq
            r = \frac{2\ee^{\ii k \dx}-\ee^{\ii 2 k \dx}-1}{1 + 2\ee^{\ii k \dx}+\ee^{-\ii 2 k 
            dx}}\com
        \eeq
        which varies from $r=0$ at $k\dx=0$ to $r=-1$ at $k\dx = \tfrac{\pi}{2}$. Hence, it represented an 
        improvement form (1).

    \item One-sided, backward, quadratic extrapolation: $u_\nmax = 3 u_{\nmax-1}-3 u_{\nmax-2}+ u_{\nmax-3}$. From \eqref{model_eqn} with $j=0$, $j=1$, $j=2$, and $j=3$,  we obtain
        \beq
        r = \frac{3\ee^{\ii k \dx}-3\ee^{\ii 2 k \dx}+\ee^{\ii 3 k \dx} -1}{1 + 3\ee^{-\ii k \dx}+3\ee^{-\ii 2 k 
        \dx} + \ee^{-\ii 3 k \dx}}\com 
        \eeq
        which varies from $r=0$ at $k\dx=0$ to $r=-1$ at $k\dx = \tfrac{\pi}{2}$. Hence, it represented an 
        improvement form (1). It also represents an improvement over (2), since the magnitude of $r$ decays
        faster towards $k \dx = 0$ (see Figure 1).

    \item Homogeneous Neumann, using backward first-order scheme: $u_\nmax = u_{\nmax-1}$. From \eqref{model_eqn}
         with $j=0$ and $j=1$, we obtain
         \beq
            r = \ee^{\ii\left(\tfrac{k\dx}{2}-\tfrac{\pi}{2}\right)}\tan \tfrac{k\dx}{2}\com
         \eeq
         whose value goes from $r=0$ at $k\dx = 0$ to $r=-1$ at $k\dx=\tfrac{\pi}{2}$. That is, very small wavenumber
         waves are only slightly reflected, whereas high wavenumber waves are strongly reflected. 

    \item First-order upwind convective boundary condition:
        \beq
            \dd_t u_\nmax = -c\frac{u_\nmax - u_{\nmax-1}}{\dx}\per
        \eeq
        First, we note that using the model \eqref{model_eqn}, $\dd u_\nmax = \ii \omega u_\nmax$. Notice that for this linear equation, wave-type solution satisfy the dispersion relation $\omega = c k$. Thus, there is no explicit dependence on $\omega$, only on $k \dx$. 
         Thus, from \eqref{model_eqn} with $j=0$, $j=1$, we obtain
         \beq
            r = \frac{\gamma \ee^{\ii k \dx}-\gamma -1}{1 + \gamma + \gamma \ee^{-\ii k \dx}}\com
         \eeq
         where
         \beq
            \gamma \defn \frac{c}{\ii \omega \dx}\per
         \eeq

    Figure \ref{fig:bcs} shows the dependence of the magnitude of the reflection coefficient for the 6 boundary 
    conditions listed above. Clearly, all schemes have total reflections for $k\dx = \tfrac{\pi}{2}$, but their
    decay towards $k = 0$ varies significantly. Based on this analysis, we rank the boundary conditions in the 
    following order, from worse to best: (1), (4), (5), (2), (3), (6). This ranking is somehow expected; the
    second-order schemes (extrapolation and convective boundary condition) perform much better than the order
    schemes. The best scheme is the second-order convective scheme, with very small reflection for 
    $k \dx < \tfrac{\pi}{6}$. 

    Figures 2 through 7 show the results of the simulations with $A = 0.1$ with various computational boundary 
    conditions, and figures 8 through 13 shows the results for $A=3$. As predicted by the analysis above, the
    worse boundary condition is the homogeneous Dirichlet. The Neumann boundary condition significantly improves
    the results, but there is still high wavenumber reflected waves. It is hard to tell visually the differences
    between the other boundary conditions.

         \begin{figure}[ht]
        \begin{center}
        \includegraphics[width=27pc,angle=0]{hw4.png}\\
        \end{center}
        \caption{The magnitude of the reflection coefficient for various artificial computational boundary conditions for the one-dimensional wave equation.}
        \label{fig:bcs}
        \end{figure}

        \begin{figure}[ht]
        \begin{center}
        \includegraphics[width=12pc,angle=0]{figs/01/1_25.png}
        \includegraphics[width=12pc,angle=0]{figs/01/1_50.png}
        \includegraphics[width=12pc,angle=0]{figs/01/1_125.png}\\
        \includegraphics[width=12pc,angle=0]{figs/01/1_150.png}
        \includegraphics[width=12pc,angle=0]{figs/01/1_175.png}
        \includegraphics[width=12pc,angle=0]{figs/01/1_200.png}\\
        \end{center}
        \caption{Time evolution of one-dimensional wave equation simulation with physical boundary condition
                of $\sin 0.1 t$ with homogeneous Dirichlet computational boundary condition.}
        \label{fig2_1}
        \end{figure}

        \begin{figure}[ht]
        \begin{center}
        \includegraphics[width=12pc,angle=0]{figs/01/2_25.png}
        \includegraphics[width=12pc,angle=0]{figs/01/2_50.png}
        \includegraphics[width=12pc,angle=0]{figs/01/2_100.png}\\
        \includegraphics[width=12pc,angle=0]{figs/01/2_150.png}
        \includegraphics[width=12pc,angle=0]{figs/01/2_175.png}
        \includegraphics[width=12pc,angle=0]{figs/01/2_200.png}\\
        \end{center}
        \caption{Time evolution of one-dimensional wave equation simulation with physical boundary condition
                of $\sin 0.1 t$ with linear extrapolation computational boundary condition.}
        \label{fig2_2}
        \end{figure}

        \begin{figure}[ht]
        \begin{center}
        \includegraphics[width=12pc,angle=0]{figs/01/3_25.png}
        \includegraphics[width=12pc,angle=0]{figs/01/3_50.png}
        \includegraphics[width=12pc,angle=0]{figs/01/3_100.png}\\
        \includegraphics[width=12pc,angle=0]{figs/01/3_150.png}
        \includegraphics[width=12pc,angle=0]{figs/01/3_175.png}
        \includegraphics[width=12pc,angle=0]{figs/01/3_200.png}\\
        \end{center}
        \caption{Time evolution of one-dimensional wave equation simulation with physical boundary condition
                of $\sin 0.1 t$ with quadratic extrapolation computational boundary condition.}
        \label{fig2_3}
        \end{figure}

        \begin{figure}[ht]
        \begin{center}
        \includegraphics[width=12pc,angle=0]{figs/01/4_25.png}
        \includegraphics[width=12pc,angle=0]{figs/01/4_50.png}
        \includegraphics[width=12pc,angle=0]{figs/01/4_100.png}\\
        \includegraphics[width=12pc,angle=0]{figs/01/4_150.png}
        \includegraphics[width=12pc,angle=0]{figs/01/4_175.png}
        \includegraphics[width=12pc,angle=0]{figs/01/4_200.png}\\
        \end{center}
        \caption{Time evolution of one-dimensional wave equation simulation with physical boundary condition
                of $\sin 0.1 t$ with homogeneous Neumann computational boundary condition.}
        \label{fig2_4}
        \end{figure}


        \begin{figure}[ht]
        \begin{center}
        \includegraphics[width=12pc,angle=0]{figs/01/5_25.png}
        \includegraphics[width=12pc,angle=0]{figs/01/5_50.png}
        \includegraphics[width=12pc,angle=0]{figs/01/5_100.png}\\
        \includegraphics[width=12pc,angle=0]{figs/01/5_150.png}
        \includegraphics[width=12pc,angle=0]{figs/01/5_175.png}
        \includegraphics[width=12pc,angle=0]{figs/01/5_200.png}\\
        \end{center}
        \caption{Time evolution of one-dimensional wave equation simulation with physical boundary condition
                of $\sin 0.1 t$ with first-order convective computational boundary condition.}
        \label{fig2_5}
        \end{figure}

      \begin{figure}[ht]
        \begin{center}
        \includegraphics[width=12pc,angle=0]{figs/01/6_25.png}
        \includegraphics[width=12pc,angle=0]{figs/01/6_50.png}
        \includegraphics[width=12pc,angle=0]{figs/01/6_100.png}\\
        \includegraphics[width=12pc,angle=0]{figs/01/6_150.png}
        \includegraphics[width=12pc,angle=0]{figs/01/6_175.png}
        \includegraphics[width=12pc,angle=0]{figs/01/6_200.png}\\
        \end{center}
        \caption{Time evolution of one-dimensional wave equation simulation with physical boundary condition
                of $\sin 0.1 t$ with second-order convective computational boundary condition.}
        \label{fig2_5}
        \end{figure}


        \begin{figure}[ht]
        \begin{center}
        \includegraphics[width=12pc,angle=0]{figs/3/1_5.png}
        \includegraphics[width=12pc,angle=0]{figs/3/1_10.png}
        \includegraphics[width=12pc,angle=0]{figs/3/1_15.png}\\
        \includegraphics[width=12pc,angle=0]{figs/3/1_20.png}
        \includegraphics[width=12pc,angle=0]{figs/3/1_25.png}
        \includegraphics[width=12pc,angle=0]{figs/3/1_30.png}\\
        \end{center}
        \caption{Time evolution of one-dimensional wave equation simulation with physical boundary condition
                of $\sin 3 t$ with homogeneous Dirichlet computational boundary condition.}
        \label{fig2_1}
        \end{figure}

        \begin{figure}[ht]
        \begin{center}
        \includegraphics[width=12pc,angle=0]{figs/3/2_5.png}
        \includegraphics[width=12pc,angle=0]{figs/3/2_10.png}
        \includegraphics[width=12pc,angle=0]{figs/3/2_15.png}\\
        \includegraphics[width=12pc,angle=0]{figs/3/2_20.png}
        \includegraphics[width=12pc,angle=0]{figs/3/2_25.png}
        \includegraphics[width=12pc,angle=0]{figs/3/2_30.png}\\
        \end{center}
        \caption{Time evolution of one-dimensional wave equation simulation with physical boundary condition
                of $\sin 3 t$ with linear extrapolation computational boundary condition.}
        \label{fig2_2}
        \end{figure}

        \begin{figure}[ht]
        \begin{center}
        \includegraphics[width=12pc,angle=0]{figs/3/3_5.png}
        \includegraphics[width=12pc,angle=0]{figs/3/3_10.png}
        \includegraphics[width=12pc,angle=0]{figs/3/3_15.png}\\
        \includegraphics[width=12pc,angle=0]{figs/3/3_20.png}
        \includegraphics[width=12pc,angle=0]{figs/3/3_25.png}
        \includegraphics[width=12pc,angle=0]{figs/3/3_30.png}\\
        \end{center}
        \caption{Time evolution of one-dimensional wave equation simulation with physical boundary condition
                of $\sin 3 t$ with quadratic extrapolation computational boundary condition.}
        \label{fig2_3}
        \end{figure}

        \begin{figure}[ht]
        \begin{center}
        \includegraphics[width=12pc,angle=0]{figs/3/4_5.png}
        \includegraphics[width=12pc,angle=0]{figs/3/4_10.png}
        \includegraphics[width=12pc,angle=0]{figs/3/4_15.png}\\
        \includegraphics[width=12pc,angle=0]{figs/3/4_20.png}
        \includegraphics[width=12pc,angle=0]{figs/3/4_25.png}
        \includegraphics[width=12pc,angle=0]{figs/3/4_30.png}\\
        \end{center}
        \caption{Time evolution of one-dimensional wave equation simulation with physical boundary condition
                of $\sin 3 t$ with homogeneous Neumann computational boundary condition.}
        \label{fig2_4}
        \end{figure}


        \begin{figure}[ht]
        \begin{center}
        \includegraphics[width=12pc,angle=0]{figs/3/5_5.png}
        \includegraphics[width=12pc,angle=0]{figs/3/5_10.png}
        \includegraphics[width=12pc,angle=0]{figs/3/5_15.png}\\
        \includegraphics[width=12pc,angle=0]{figs/3/5_20.png}
        \includegraphics[width=12pc,angle=0]{figs/3/5_25.png}
        \includegraphics[width=12pc,angle=0]{figs/3/5_30.png}\\
        \end{center}
        \caption{Time evolution of one-dimensional wave equation simulation with physical boundary condition
                of $\sin 3 t$ with first-order convective computational boundary condition.}
        \label{fig2_5}
        \end{figure}

      \begin{figure}[ht]
        \begin{center}
        \includegraphics[width=12pc,angle=0]{figs/3/6_5.png}
        \includegraphics[width=12pc,angle=0]{figs/3/6_10.png}
        \includegraphics[width=12pc,angle=0]{figs/3/6_15.png}\\
        \includegraphics[width=12pc,angle=0]{figs/3/6_20.png}
        \includegraphics[width=12pc,angle=0]{figs/3/6_25.png}
        \includegraphics[width=12pc,angle=0]{figs/3/6_30.png}\\
        \end{center}
        \caption{Time evolution of one-dimensional wave equation simulation with physical boundary condition
                of $\sin 3 t$ with second-order convective computational boundary condition.}
        \label{fig2_5}
        \end{figure}



\end{enumerate}

\end{document}
