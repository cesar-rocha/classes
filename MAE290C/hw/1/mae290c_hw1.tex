\documentclass[11pt]{article}


%% WRY has commented out some unused packages %%
%% If needed, activate these by uncommenting
\usepackage{geometry}                % See geometry.pdf to learn the layout options. There are lots.
%\geometry{letterpaper}                   % ... or a4paper or a5paper or ... 
\geometry{a4paper,left=2.5cm,right=2.5cm,top=2.5cm,bottom=2.5cm}
%\geometry{landscape}                % Activate for rotated page geometry
%\usepackage[parfill]{parskip}    % Activate to begin paragraphs with an empty line rather than an indent

%for figures
%\usepackage{graphicx}

\usepackage{color}
\definecolor{mygreen}{RGB}{28,172,0} % color values Red, Green, Blue
\definecolor{mylilas}{RGB}{170,55,241}
%% for graphics this one is also OK:
\usepackage{epsfig}

%% AMS mathsymbols are enabled with
\usepackage{amssymb,amsmath}

%% more options in enumerate
\usepackage{enumerate}
\usepackage{enumitem}

%% insert code
\usepackage{listings}

\usepackage[utf8]{inputenc}

\usepackage{hyperref}


% Default fixed font does not support bold face
\DeclareFixedFont{\ttb}{T1}{txtt}{bx}{n}{12} % for bold
\DeclareFixedFont{\ttm}{T1}{txtt}{m}{n}{12}  % for normal

% Custom colors
\usepackage{color}
\definecolor{deepblue}{rgb}{0,0,0.5}
\definecolor{deepred}{rgb}{0.6,0,0}
\definecolor{deepgreen}{rgb}{0,0.5,0}


% Python style for highlighting
\newcommand\pythonstyle{\lstset{
language=Python,
basicstyle=\ttm,
otherkeywords={self},             % Add keywords here
keywordstyle=\ttb\color{deepblue},
emph={MyClass,__init__},          % Custom highlighting
emphstyle=\ttb\color{deepred},    % Custom highlighting style
stringstyle=\color{deepgreen},
frame=tb,                         % Any extra options here
showstringspaces=false            % 
}}

% Python environment
\lstnewenvironment{python}[1][]
{
\pythonstyle
\lstset{#1}
}
{}

% Python for external files
\newcommand\pythonexternal[2][]{{
\pythonstyle
\lstinputlisting[#1]{#2}}}

% Python for inline
\newcommand\pythoninline[1]{{\pythonstyle\lstinline!#1!}}

%\usepackage{epstopdf}
%\DeclareGraphicsRule{.tif}{png}{.png}{`convert #1 `dirname #1`/`basename #1 .tif`.png}

%% To save typing, create some shortcuts
\newcommand{\ord}{\mbox{ord}}
\newcommand{\Ai}{\mbox{Ai}}
\newcommand{\Bi}{\mbox{Bi}}
\newcommand{\half}{\tfrac{1}{2}}
\newcommand{\defn}{\stackrel{\text{def}}{=}}
%% Use Roman font for special numbers and differentials:
\newcommand{\ii}{\mathrm{i}}
\newcommand{\dd}{\mathrm{d}}
\newcommand{\ee}{\mathrm{e}}
\newcommand{\su}{\mathsf{u}}
\newcommand{\sv}{\mathsf{v}}

\newcommand{\dstar}{{\star\star}}
\newcommand{\dt}{\Delta t}
\newcommand{\dx}{\Delta x}
\newcommand{\sL}{\mathsf{L}}


\newcommand{\com}{\, ,}
\newcommand{\per}{\, .}

\newcommand{\noi}{\noindent}

% space in equations
\newcommand{\qqand}{\qquad \text{and} \qquad}
\newcommand{\qand}{\quad \text{and} \quad}


\def\beq{\begin{equation}}
\def\eeq{\end{equation}}


%% stop typing all of epsilon and delta
\newcommand{\ep}{\ensuremath {\epsilon}}
\newcommand{\de}{\ensuremath {\delta}}

%% colors
\usepackage{graphicx,xcolor,lipsum}


\usepackage{mathtools}

\usepackage{graphicx}
\newcommand*{\matminus}{%
  \leavevmode
  \hphantom{0}%
  \llap{%
    \settowidth{\dimen0 }{$0$}%
    \resizebox{1.1\dimen0 }{\height}{$-$}%
  }%
}


\title{MAE290C, Homework Assignment 1}
\author{Cesar B Rocha}
\date{\today}

\begin{document}
\maketitle

\section*{Problem 1}
The linear equation to solve is
\beq
\label{eq:pb1_eqn1}
\frac{\dd u}{\dd t} = -\underbrace{\left(\ii \frac{c \pi}{\Delta x} + \frac{\nu \pi^2}{\Delta x^2}\right)}_{\defn \lambda}\,u\per
\eeq

\subsection*{Set up}
The RKW3 scheme for this linear problem is

\begin{align}
u^{n+1} = u^\dstar - \lambda \dt(\alpha_3 u^\star + \beta_3 u^{n+1})\com \nonumber \\
u^\dstar = u^\star - \lambda \dt(\alpha_2 u^\star + \beta_2 u^\dstar)\com \nonumber \\
u^\star = u^n - \lambda \dt (\alpha_1 u^n + \beta_1 u^\star)\com
\end{align}
where
\begin{align}
\beta_1 = \frac{37}{160}\com\qquad \beta_2 = \frac{5}{24}\com\qqand \beta_3 = \frac{1}{6}\com \nonumber \\
\alpha_1 = \frac{29}{96}\com\qquad \alpha_2 = -\frac{3}{40}\com\qqand \alpha_3 = \frac{1}{6}\per
\end{align}
Hence we solve the problem in the following order\begin{align}
u^\star = \frac{1 -  \alpha_1 \lambda\dt}{1+ \beta_1 \lambda\dt} u^n \com\qquad u^\dstar = \frac{1 -  \alpha_2 \lambda\dt}{1+\beta_2\lambda  \dt} u^\star \com\qqand u^{n+1} = \frac{1 - \alpha_3 \lambda \dt}{1+\beta_3 \lambda \dt} u^\dstar \com
\end{align}
where  we have to store only two variables per substep, or we can compute every substep in place.

\subsection*{Exact solution}
For this simple problem we have the exact solution
\beq
u_e (t) = \ee^{-\lambda t}\com
\eeq
where for the given fixed parameters
\beq
\pi c \dt/\dx = 0.5\com\qqand \text{Re}_{\dx} \defn c \dx/\nu = 1\com
\eeq
we have
\beq
\label{eq:lamdt}
\lambda\dt = \frac{1}{2} \left( \pi + \ii \right)\com\qquad \text{or}\qquad \lambda = \frac{1}{2\dt}\left(\pi + \ii\right)\per
\eeq
Notice that $u$ is complex-value. Its real part gives an exponential decay, whereas its complex part is associated with phase changes.

\subsection*{Numerical solution}
We implement this scheme to solve \eqref{eq:pb1_eqn1} with \eqref{eq:lamdt} in Fortran 90. We choose $\dt = 0.5$, and also compute the exact solution for reference. Figure \ref{fig:pb1} shows the magnitude and phase of solutions  of the solutions. For the given parameters, the solution decays very fast for $t<2$. At $t=0$ the solution magnitude is almost zero within machine double precision. The numerical scheme solves the magnitude very accurately: the relative root-mean-square difference is about $2\%$. However, the is a significant phase error. This error in phase increases with time. 

\begin{figure}[ht]
\begin{center}
\includegraphics[width=15pc,angle=0]{pb1_mag.png}
\includegraphics[width=15pc,angle=0]{pb1_phase.png}
\end{center}
\caption{A comparison between the numerical solution to \eqref{eq:pb1_eqn1} using the RKW3 scheme against the exact solution. The inset shows a zoom-in for $t<2$. Notice that the numerical solution captures the magnitude very accurately, but not the phase.}
\label{fig:pb1}
\end{figure}


\begin{lstlisting}
program pb1

    ! MAE 290C, Homework 1
    ! Cesar B Rocha, crocha@ucsd.edu

    implicit none
    real (kind=8) :: u0,dt,pi,tmax,t
    complex (kind=16) :: lamdt,u,ue,ii
    integer :: nmax,n
    logical :: debug

    ! flags
    debug = .true.

    ! set up output file
    open(unit=12, file="aoutput.txt", action="write", status="replace")

    ! pi
    pi = acos(-1.d0)

    ! time-related parameters
    dt = .5
    tmax = 10.d0
    nmax = ceiling(tmax/dt)

    ! complex constants
    ii = cmplx(0.d0,1.d0)
    lamdt = cmplx(pi/2.d0,0.5d0)

    if (debug) then
        print 11, dt, nmax
        11 format("dt=",f25.15,", nmax=", i10)
    endif

    ! initialize variables
    t = 0.d0
    u0 = 1.d0
    u = cmplx(u0,0.d0)
    ue =  exp(-n*lamdt)

    write(12,*)  "#time          real(ue)      imag(ue)      real(u)        imag(u)"
    write(12,*) t,real(ue),aimag(ue),real(u),aimag(u)

    do n=1,nmax
        t = t+dt
        call stepforward(u,lamdt)
        ue = exp(-n*lamdt)
        write(12,*) t,real(ue),aimag(ue),real(u),aimag(u)
    enddo

end program pb1

!------------------------------------------------------------
! subroutine to step forward using RKW3
!------------------------------------------------------------

subroutine stepforward(u,lamdt)

    implicit none

    ! subroutine arguments
    complex (kind=16), intent(in) :: lamdt
    complex (kind=16), intent(inout) :: u

    ! local variables
    real (kind=8), parameter :: a1 = 29.d0/96.d0, a2 = -3.d0/40.d0, & 
                                    a3 = 1.d0/6.d0
    real (kind=8), parameter :: b1 = 37.d0/160.d0, b2 = 5.d0/24.d0, & 
                                    b3 = 1.d0/6.d0

    ! step forward, computing updates in place
    u = ( (1.d0-lamdt*a1)/(1.d0 + lamdt*b1) )*u 
    u = ( (1.d0-lamdt*a2)/(1.d0 + lamdt*b2) )*u 
    u = ( (1.d0-lamdt*a3)/(1.d0 + lamdt*b3) )*u 

end subroutine stepforward
\end{lstlisting}


\section*{Problem 2}
The initial value problem to solve is
\beq
\label{eq:pb2_eqn}
\frac{\dd u}{\dd t} = - u^n\com \qquad \text{with} \qquad u(0) =1\com
\eeq
where $n\ge1$ is a natural number.

\subsection*{Exact solution}
The exact solution consistent with the initial condition is
\beq
u_{e}(t) = \ee^{-t}\com\qqand n=1\com
\eeq
and
\beq
u_e(t) = \frac{1}{[1 +(n-1)t]^{1/(n-1)}}\com\qqand n>1 \per
\eeq

\subsection*{Numerical solution}
A RK2 scheme, corresponding to choosing $c=1$ in our class notes, is
\beq
\label{eq:rk2}
u_{i+1} = u_i + \frac{\dt}{2} \left(b_1 k_1 + b_2 k_2\right)\com
\eeq
with
\begin{align}
    k_1 =  u_i^n \com\qqand k_2 =  (u_i + \dt\, a_{2,1} k_1)^n\com
\end{align}
where the subscript $i$ denotes the $i$'th time step, whereas the superscript $n$ denotes power of $n$ as in the exact problem \eqref{eq:pb2_eqn}. Also the constants are
\beq
\label{eq:param_rk2}
a_{2,1} = c\com \qquad b_2 = \frac{1}{2c}\com \qqand b_1 = 1 - \frac{1}{2c}\com
\eeq
where $c$ is a free parameter. A measure of the absolute error in the numerical solutions can be defined as the square-root of the $L_2$-norm of the difference to the exact solution:
\beq
\label{eq:error_defn}
\text{Err}(c) = \sqrt{\int_0^{t_{max}}\!\!\! [u_e(t)-u_{rk}(c,t)]^2\dd t}\per
\eeq
We evaluate the integral in \eqref{eq:error_defn} using the trapezoidal rule.

Figure \ref{fig:pb2_1} shows an example of solutions to \eqref{eq:pb2_eqn}. The numerical solution using RK2 scheme performs very well; there is no visual distinction between exact and numerical solutions. Solutions with other values of $c$ are essentially equal to the solution in Figure \ref{fig:pb2_1} ($c=1$). Indeed, the error in the numerical solution is nearly independent of the value of the free parameter $c$; from $n=.1$ through $c=5$ the error increases by less than $1\%$.

\begin{figure}[ht]
\begin{center}
\includegraphics[width=15pc,angle=0]{pb2/pb2_100.png}
\end{center}
\caption{Solution to \eqref{eq:pb2_eqn}  for values of n from $1$ through $5$. The solid thick lines are the exact solutions, and the dashed lines the numerical solutions using a RK2 scheme with $c=2$.}
\label{fig:pb2_1}
\end{figure}

\begin{figure}[ht]
\begin{center}
\includegraphics[width=25pc,angle=0]{pb2/pb2_error.png}
\end{center}
\caption{Error in numerical solution to \eqref{eq:pb2_eqn} as a function of the free parameter $c$.}
\label{fig:pb2_error}
\end{figure}

\begin{lstlisting}
program pb2

    ! MAE 290C, Homework 1, Problem 2
    ! Cesar B Rocha, crocha@ucsd.edu

    implicit none

    real (kind=8) :: ue, urk, t, err, tmax, c(6)
    real (kind=8), parameter :: dt = 0.001, u0=1.d0
    integer :: n, k, kmax, nmax,i
    logical :: debug
    character (1) :: ns
    character (2) :: cs2
    character (3) :: cs3
    character (30) :: fno

    ! flags
    debug = .true.

    ! time
    tmax = 5.d0
    kmax = ceiling(tmax/dt)

    ! free parameter
    c = (/.10d0,0.25d0,0.5d0,0.75d0,1.d0,5.d0/)

    ! max power
    nmax = 5

    do i=1,size(c)

        do n=1,nmax

            ! set up output file
            write (ns,"(i1)") n
            if (c(i)<1.d0) then
                write (cs2,"(i2)") ceiling(c(i)*100)
                fno = 'output/pb2_'//ns//'_0'//cs2//'.out'
            else
                write (cs3,"(i3)") ceiling(c(i)*100)
                fno = 'output/pb2_'//ns//'_'//cs3//'.out'

            endif
            
            !print *,fno
            open(unit=12, file=fno, action="write", status="replace")

            ! initialize variable
            t = 0.d0
            ue = u0
            urk = u0

            do k=1,kmax

                write(12,*) t,ue,urk
                !print *, t,ue,urk
            
                ! update ue and uk
                call exact(ue,t,n)
                call stepforward_rk2(urk,dt,c(i),n)

                ! update time
                t = t + dt

            enddo

        enddo

    enddo

end program pb2

!-------------------------------------------------
! subroutines 
!-------------------------------------------------

! calculate the exact soln
subroutine exact(ue,t,n)

    implicit none
    
    ! subroutine arguments
    real (kind=8), intent(in) :: t
    real (kind=8), intent(inout) :: ue
    integer, intent(in) :: n
        
    if (n==1) then
        ue = exp(-t)
    else
        ue = 1.d0 / ( (1 + (n-1)*t)**(1.d0/(n-1)) )
    endif

end subroutine exact

! step forward using RKW2
subroutine stepforward_rk2(urk,dt,c,n)

    implicit none

    ! subroutine arguments
    real (kind=8), intent(in) :: c, dt
    real (kind=8), intent(inout) :: urk
    integer, intent(in) :: n

    ! local variables
    real (kind=8) :: a21, b1, b2, k1, k2
    
    ! compute parameters given c
    a21 = c

    b2 = 1.d0/2.d0/c
    b1 = 1.d0 - 1.d0/2.d0/c

    k1 = -(urk**n)
    k2 = -( (urk + dt*a21*k1)**n )

    ! update urk in place
    urk = urk + dt*(b1*k1 + b2*k2)

end subroutine stepforward_rk2
\end{lstlisting}


\section*{Problem 3}
The problem to solve is
\beq
\label{eq:pb3_eqn}
\frac{\dd u}{\dd t} = \underbrace{\frac{\nu \pi^2}{\dx^2}}_{\defn \lambda} u\com
\eeq
with $u(0)=1$.
\subsection*{Exact solution}
The exact solution is
\beq
u_e = \ee^{\lambda t}\per
\eeq
\subsection*{Numerical solution}
The solution using a $\theta$-scheme is
\beq
\label{eq:th_soln}
u^{n+1} = u^n +  \lambda \dt [(1-\theta) u^n + \theta u^{n+1}]\per
\eeq
Hence
\beq
\label{eq:th_soln2}
u^{n+1} = \frac{1 + \lambda \dt (1-\theta)}{1-\lambda \dt \, \theta}\,u^n\per
\eeq
Implicit Euler is recovered with $\theta=1$, and Crank-Nicolson uses $\theta=1/2$. We implement this scheme in Fortran 90 to solve \eqref{eq:pb3_eqn} with $\lambda \dt = -0.1,-0.5,-1.0,-2.0,-4.0,-8.0$.

Before we even start to analyze the results, we anticipate problems with the numerical schemes to particular values of $\lambda \dt$. With $\lambda \dt = -2$ the numerical solutions \eqref{eq:th_soln2}  is identically zero for the Crank-Nilcolson scheme ($\theta = 1/2$); for this combination of parameter the nominator on the right of \eqref{eq:th_soln2} is identically zero, which is obviously inaccurate. Similarly, with $\lambda \dt = -4$ the numerical solutions identically zero for the $\theta = 3/4$ scheme. These are pathologies of the numeric solution for these specific parameters.

Crank-Nicolson perform better for $|\lambda \dt| < 1$ (see figure \ref{fig:pb3}). This is because Crank-Nicolson is second-order accuracy, as opposed to the first-order accurate implicit Euler scheme; the error of the $\theta=3/4$ has is half of the error of the Implicit Euler scheme. For   $|\lambda \dt| > 2$, Crank-Nicolson become inaccurate. This is expected because, for Crank-Nicolson, we have
\beq
\frac{u^{n+1}}{u^n} = \frac{1+\lambda \dt}{1-\lambda \dt}\per
\eeq
For large negative $\lambda \dt$ the growth factor above approached -1. Hence, the solution oscillated between positive and negative values, a behavior absent in the exact solution. Thus, Crank-Nicolson becomes qualitatively misleading for large $|\lambda \dt|$. For these parameter, the $\theta=3/4$ scheme outperforms Crank-Nilcolson (see the last two panel in figure \ref{fig:pb3}). 

\begin{figure}[ht]
\begin{center}
\includegraphics[width=12pc,angle=0]{pb3/pb3_01.png}\includegraphics[width=12pc,angle=0]{pb3/pb3_05.png}\includegraphics[width=12pc,angle=0]{pb3/pb3_10.png}
\includegraphics[width=12pc,angle=0]{pb3/pb3_20.png}\includegraphics[width=12pc,angle=0]{pb3/pb3_40.png}\includegraphics[width=12pc,angle=0]{pb3/pb3_80.png}
\end{center}
\caption{A comparison of different implicit schemes for solving \eqref{eq:pb3_eqn} for 10 time-steps with $\dt=0.01$ and various values of $\lambda \dt$. Also shown is the exact solution (blue line) evaluated at the discrete time-steps.}
\label{fig:pb3}
\end{figure}

\end{document}


