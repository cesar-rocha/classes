\documentclass[11pt]{article}


%% WRY has commented out some unused packages %%
%% If needed, activate these by uncommenting
\usepackage{geometry}                % See geometry.pdf to learn the layout options. There are lots.
%\geometry{letterpaper}                   % ... or a4paper or a5paper or ... 
\geometry{a4paper,left=2.5cm,right=2.5cm,top=2.5cm,bottom=2.5cm}
%\geometry{landscape}                % Activate for rotated page geometry
%\usepackage[parfill]{parskip}    % Activate to begin paragraphs with an empty line rather than an indent

%for figures
%\usepackage{graphicx}

\usepackage{color}
\definecolor{mygreen}{RGB}{28,172,0} % color values Red, Green, Blue
\definecolor{mylilas}{RGB}{170,55,241}
%% for graphics this one is also OK:
\usepackage{epsfig}

%% AMS mathsymbols are enabled with
\usepackage{amssymb,amsmath}

%% more options in enumerate
\usepackage{enumerate}
\usepackage{enumitem}

%% insert code
\usepackage{listings}

\usepackage[utf8]{inputenc}

\usepackage{hyperref}

%% To make really wide whats that cover everything:
\usepackage{scalerel}
\usepackage{stackengine}
\stackMath
\def\hatgap{2pt}
\def\subdown{-2pt}
\newcommand\what[2][]{%
\renewcommand\stackalignment{l}%
\stackon[\hatgap]{#2}{%
\stretchto{%
    \scalerel*[\widthof{$#2$}]{\kern-.6pt\bigwedge\kern-.6pt}%
    {\rule[-\textheight/2]{1ex}{\textheight}}%WIDTH-LIMITED BIG WEDGE
}{0.5ex}% THIS SQUEEZES THE WEDGE TO 0.5ex HEIGHT
_{\smash{\belowbaseline[\subdown]{\scriptstyle#1}}}%
}}

% Default fixed font does not support bold face
\DeclareFixedFont{\ttb}{T1}{txtt}{bx}{n}{12} % for bold
\DeclareFixedFont{\ttm}{T1}{txtt}{m}{n}{12}  % for normal

% Custom colors
\usepackage{color}
\definecolor{deepblue}{rgb}{0,0,0.5}
\definecolor{deepred}{rgb}{0.6,0,0}
\definecolor{deepgreen}{rgb}{0,0.5,0}


% Python style for highlighting
\newcommand\pythonstyle{\lstset{
language=Python,
basicstyle=\ttm,
otherkeywords={self},             % Add keywords here
keywordstyle=\ttb\color{deepblue},
emph={MyClass,__init__},          % Custom highlighting
emphstyle=\ttb\color{deepred},    % Custom highlighting style
stringstyle=\color{deepgreen},
frame=tb,                         % Any extra options here
showstringspaces=false            % 
}}

% Python environment
\lstnewenvironment{python}[1][]
{
\pythonstyle
\lstset{#1}
}
{}

% Python for external files
\newcommand\pythonexternal[2][]{{
\pythonstyle
\lstinputlisting[#1]{#2}}}

% Python for inline
\newcommand\pythoninline[1]{{\pythonstyle\lstinline!#1!}}

%% colors
\usepackage{graphicx,xcolor,lipsum}


\usepackage{mathtools}

\usepackage{graphicx}
\newcommand*{\matminus}{%
  \leavevmode
  \hphantom{0}%
  \llap{%
    \settowidth{\dimen0 }{$0$}%
    \resizebox{1.1\dimen0 }{\height}{$-$}%
  }%
}


\title{MAE290C, Homework Assignment 3}
\author{Cesar B Rocha}
\date{\today}

\begin{document}

\newcommand{\com}{\, ,}
\newcommand{\per}{\, .}

%% Averages
% Use \bar to over line solo symbols

\newcommand{\av}[1]{\bar{#1}}
\newcommand{\avbg}[1]{\overline{#1}}
\newcommand{\avbgg}[1]{\overline{#1}}

% A nice definition
\newcommand{\defn}{\ensuremath{\stackrel{\mathrm{def}}{=}}}

% equations
\def\beq{\begin{equation}}
\def\eeq{\end{equation}}

% calculus

\newcommand{\p}{\partial}
\newcommand{\ii}{{\rm i}}
\newcommand{\dd}{{\rm d}}
\newcommand{\id}{{\, \rm d}}
\newcommand{\ee}{{\rm e}}
\newcommand{\DD}{{\rm D}}
\newcommand{\wavy}{\text{wavy}}
\newcommand{\qg}{\text{qg}}


\newcommand{\be}{\beta}

\newcommand{\al}{\alpha}
\newcommand{\bx}{\boldsymbol{x}}
\newcommand{\by}{\boldsymbol{y}}
\newcommand{\bu}{\boldsymbol{u}}
\newcommand{\bv}{\boldsymbol{v}}


\newcommand{\half}{\tfrac{1}{2}}
\newcommand{\halfrho}{\tfrac{1}{2}}
\newcommand{\rz}{{}}
\newcommand{\bn}{\boldsymbol{\hat n}}
\newcommand{\br}{\boldsymbol{r}}
\newcommand{\bR}{\boldsymbol{R}}
\newcommand{\bA}{\ensuremath {\boldsymbol {A}}}
\newcommand{\bB}{\ensuremath {\boldsymbol {B}}}
\newcommand{\bU}{\ensuremath {\boldsymbol {U}}}
\newcommand{\bE}{\ensuremath {\boldsymbol {E}}}
\newcommand{\bJ}{\ensuremath {\boldsymbol {J}}}
\newcommand{\bXX}{\ensuremath {\boldsymbol {\mathcal{X}}}}
\newcommand{\bFF}{\ensuremath {\boldsymbol {F}}}
\newcommand{\bF}{\ensuremath {\boldsymbol {F}^{\sharp}}}
\newcommand{\bG}{\ensuremath {\boldsymbol G}}
\newcommand{\bSigma}{\ensuremath {\boldsymbol {\Sigma}}}
\newcommand{\bvarphi}{\ensuremath {\boldsymbol {\varphi}}}
\newcommand{\bxi}{\ensuremath {\boldsymbol {\xi}}}
\newcommand{\avbxi}{\overline{\ensuremath {\boldsymbol {\xi}}}}

% math cal

\newcommand{\J}{\mathcal{J}}
\newcommand{\K}{\mathcal{K}}
\newcommand{\cG}{\mathcal{G}}
\newcommand{\cF}{\mathcal{F}}
\newcommand{\cN}{\mathcal{N}}
\newcommand{\cL}{\mathcal{L}}


% san serif for matrices and differential operators
%\newcommand{\helmn}{\mathsf{H}_n}
\newcommand{\helmm}{\triangle_m}
\newcommand{\helmn}{\triangle_n}
\newcommand{\helms}{\triangle_s}
\newcommand{\helm}{\triangle}
\newcommand{\sA}{\mathsf{A}}
\newcommand{\sB}{\mathsf{B}}
\newcommand{\sG}{\mathsf{G}}
\newcommand{\sI}{\mathsf{I}}
\newcommand{\sJ}{\mathsf{J}}
\newcommand{\sU}{\mathsf{U}}
\newcommand{\sP}{\mathsf{P}}
\newcommand{\sQ}{\mathsf{Q}}
\newcommand{\sR}{\mathsf{R}}
\newcommand{\sL}{\mathsf{L}}
\renewcommand{\sJ}{\mathsf{J}}
\renewcommand{\sI}{\mathsf{I}}
\renewcommand{\L}{\mathsf{L}}
\newcommand{\sM}{\mathsf{M}}
\newcommand{\sT}{\mathsf{T}}
\newcommand{\sGamma}{\mathsf{\Gamma}}
\newcommand{\sOmega}{\mathsf{\Omega}}
\newcommand{\sSigma}{\mathsf{\Omega}}
\newcommand{\sbeta}{\mathsf{\beta}}
\newcommand{\sPi}{\mathsf{\Pi}}
\newcommand{\sC}{\mathsf{C}}
\newcommand{\sQy}{\mathsf{Q}}


% angle brackets

\def\la{\langle}
\def\ra{\rangle}
\def\laa{\left \langle}
\def\raa{\right \rangle}


%grads and div's
\newcommand{\bcdot}{\hspace{-0.1em} \boldsymbol{\cdot} \hspace{-0.12em}}
\newcommand{\bnabla}{\boldsymbol{\nabla}}
\newcommand{\bnablaH}{\bnabla_{\! \mathrm{h}}}
\newcommand{\grad}{\bnabla}
\newcommand{\gradH}{\bnablaH}
\newcommand{\curl}{\bnabla \!\times\!}
\newcommand{\diver}{\bnabla \bcdot }
\newcommand{\cross}{\times}
\newcommand{\lap}{\triangle}


%varthetas and thetas
\newcommand{\vth}{\vartheta}
\newcommand{\psii}{\psi^{\mathrm{i}}}
\newcommand{\thb}{\theta^{\mathrm{-}}}
\newcommand{\vthb}{\vartheta^{\mathrm{-}}}
\newcommand{\vthbhat}{{\hat{\vartheta}}^{\mathrm{-}}}
\newcommand{\vThb}{\varTheta^{\mathrm{-}}}
\newcommand{\psib}{\psi^{\mathrm{-}}}
\newcommand{\tht}{\theta^{\mathrm{+}}}
\newcommand{\vtht}{\vartheta^{\mathrm{+}}}
\newcommand{\vththat}{{\hat{\vartheta}}^{\mathrm{+}}}
\newcommand{\vthtbhat}{{\hat{\vartheta}}^{\pm}}
\newcommand{\vTht}{\varTheta^{\mathrm{+}}}
\newcommand{\vthtb}{\vartheta^{\pm}}
\newcommand{\vThtb}{\varTheta^{\pm}}

% nondimensional numbers
\renewcommand{\Re}{\mathrm{Re}}
\newcommand{\Ro}{\mathrm{Ro}}
\newcommand{\Ri}{\mathrm{Ri}}

%psi's
%Galerking coefficient for psi:
\newcommand{\gpsi}{\breve \psi}
\newcommand{\gphi}{\breve \phi}
\newcommand{\gq}{\breve q}
\newcommand{\gU}{\breve U}
\newcommand{\gQ}{\breve Q}
\newcommand{\gsigma}{\breve \sigma}


\newcommand{\psit}{\psi^{\mathrm{+}}}
\newcommand{\psithat}{{\hat{\psi}}^{\mathrm{+}}}
\newcommand{\psibhat}{{\hat{\psi}}^{\mathrm{-}}}
\newcommand{\psitb}{\psi^{\pm}}
\newcommand{\psitbhat}{{\hat{\psi}}^\pm}
\newcommand{\St}{S^{\mathrm{+}}}
\newcommand{\Sb}{S^{\mathrm{-}}}
\newcommand{\phb}{\phi^{\mathrm{-}}}
\newcommand{\pht}{\phi^{\mathrm{+}}}
\newcommand{\tautb}{\tau^{\pm}}
\newcommand{\sigmatb}{\sigma^{\pm}}


\newcommand{\bur}{\left(\tfrac{f_0}{N}\right)^2}
\newcommand{\ibur}{\left(\tfrac{N}{f_0}\right)^2}
\newcommand{\Nm}{N_{\mathrm{mix}}}
\newcommand{\xim}{\xi_{\mathrm{mix}}}
\newcommand{\hs}{h_*}
\renewcommand{\sp}{\mathsf{p}}
\newcommand{\se}{\mathsf{e}}
\newcommand{\sptb}{\mathsf{p}^\pm}


%nmax is a problem:
%\newcommand{\nmax}{n_{\mathrm{max}}}
\newcommand{\nmax}{\mathrm{N}}

\newcommand{\WKB}{\mathrm{WKB}}
\newcommand{\Lam}{\Lambda}
\newcommand{\tha}{\theta}
\newcommand{\kap}{\kappa}
\newcommand{\bphi}{\boldsymbol{\phi}}
\newcommand{\third}{\tfrac{1}{3}}
\newcommand{\cs}{c^\star}
\newcommand{\nt}{n^{\mathrm{trnc}}}
\newcommand{\sDp}{\mathsf{D}^1_{\nmax}}
\newcommand{\sDpp}{\mathsf{D}^2_{\nmax}}
\newcommand{\sD}{\mathsf{D_2}}
\newcommand{\sK}{\mathsf{K_2}}
\newcommand{\stheta}{\mathsf{\theta}}
\newcommand{\sphi}{\mathsf{\phi}}
\newcommand{\sq}{\mathsf{q}}
\newcommand{\cosech}{\text{csch}\,}
\newcommand{\sinc}{\text{sinc}\,}

%%%%%%%%% %%%%
\newcommand{\zp}{z^+}
\newcommand{\zm}{z^-}
\newcommand{\qA}{q^A_{\nmax}}
\newcommand{\psiB}{\psi^B_{\nmax}}
\newcommand{\phiB}{\phi^B_{\nmax}}
\newcommand{\eye}{\boldsymbol{\hat{i}}}
\newcommand{\jay}{\boldsymbol{\hat{j}}}
\newcommand{\kay}{\boldsymbol{\hat{k}}}
\newcommand{\psiG}{\psi^{\mathrm{G}}}
\newcommand{\qG}{q^{\mathrm{G}}}
\newcommand{\uG}{u^{\mathrm{G}}}
\newcommand{\UG}{U^{\mathrm{G}}}
\newcommand{\UGN}{U^{\mathrm{G}}_{\nmax}}
\newcommand{\QGN}{Q^{\mathrm{G}}_{\nmax}}
\newcommand{\sumoddn}{\sum_{n = 1, n~ \text{odd}}^{\nmax}}

% bretherton 
\newcommand{\qBr}{q_{\mathrm{Br}}}
\newcommand{\psiBr}{\psi_{\mathrm{Br}}}


\maketitle

\section*{Problem 1}


The non-dimensional Navier-Stokes equations, with constant density, are
\beq
\label{eq:ns1}
\p_t \bu = - \underbrace{\bu\cdot\nabla\bu}_{\defn \bN(\bu)}  -\nabla p + \frac{1}{\Re}\nabla\cdot\nabla\bu\com 
\eeq
and
\beq
\label{eq:ns2}
\nabla\cdot\bu = 0\com
\eeq
where $\bu$ is the velocity, $p$ is the dynamic pressure, and $\Re$ is the Reynolds number. We use the following
 notation for the discrete linear operators
 \beq
 \label{eq:defns}
    \sD \approx \nabla\cdot\com\qquad  \sG \approx\nabla \com\qqand\sL \approx \nabla\cdot\nabla\com
 \eeq   
 where $\sD$, $\sG$ and $\sL$ should be understood as matrices.

\begin{enumerate}


\item Time-discretizing \eqref{eq:ns1} and \eqref{eq:ns2} we obtain using explicit Euler for the non-linear term
 and implicit Euler for the linear terms, we obtain
 \beq
    \label{eq:ns1_d}
    \frac{\bu^{n+1}-\bu^{n}}{\Delta t} \approx \bN(\bu^{n}) - \sG p^{n+1} + \frac{1}{\Re} \sD\sG \bu^{n+1}\com 
 \eeq
and
\beq
    \label{eq:ns2_d}
    \sD u^{n+1} = 0\per
\eeq
We can re-cast \eqref{eq:ns1_d} and \eqref{eq:ns2_d} in block-matrix form
\beq
    \label{eq:ns_d}
    \underbrace{
    \begin{bmatrix*} 
        \sA & \sG \dt\\
        \sD & {\bf 0}
    \end{bmatrix*}}_{\defn \sM} 
    \begin{bmatrix*} 
        \bu^{n+1}\\
        p^{n+1}
    \end{bmatrix*} = \begin{bmatrix*} 
        \bu^{n}-\bN(\bu^n)\dt\\
        \bf{0}\\
    \end{bmatrix*}\com
\eeq
where
\beq
\sA \defn \sI - \frac{\dt}{\Re} \sL\per
\eeq

\item We have
    \beq
        \sA \sG  = \sG - \frac{\dt}{\Re} \sL\per
    \eeq
    Thus, to within an error of $\mathcal{O}(\dt)$, we can approximate \ref{eq:ns_d} as
\beq
    \label{eq:ns_d_2}
    \underbrace{
    \begin{bmatrix*} 
        \sA & \sA \sG \dt\\
        \sD & {\bf 0}
    \end{bmatrix*}}_{\defn \sM_2} 
    \begin{bmatrix*} 
        \bu^{n+1}\\
        p^{n+1}
    \end{bmatrix*} = \begin{bmatrix*} 
        \bu^{n}-\bN(\bu^n)\dt\\
        \bf{0}\\
    \end{bmatrix*}\per
\eeq

    \item To obtain the LU-factorization, we note that we can operate on the block-matrix $\sM_2$ as in
         standard matrices. Hence, we can perform a block Gauss elimination  on $\sM_2$. Noticing that the 
          block multiplier is simply $\sD \sA^{-1}$, we obtain

\begin{align}
    \sM_2 = 
    \begin{bmatrix*} 
        \sA & \sA \sG \dt\\
        \sD & {\bf 0}
    \end{bmatrix*} = & 
    \begin{bmatrix*} 
        \sI & 0\\
        \sD \sA^{-1} & \sI
    \end{bmatrix*}
\begin{bmatrix*} 
        \sA & \sA \sG \dt\\
    {\bf 0} & -\sL\dt 
    \end{bmatrix*}\\
    = &    \begin{bmatrix*} 
        \sI & \bf{0}\\
        \sD \sA^{-1} & \sI
 \end{bmatrix*}
 \begin{bmatrix*} 
     \sA & \bf{0}\\
     \bf{0} & -\sL \dt
    \end{bmatrix*}
\begin{bmatrix*} 
        \sI &  \sG \dt\\
    {\bf 0} & - \sI 
    \end{bmatrix*}\\
    = &
\begin{bmatrix*} 
        \sA & \bf{0}\\
        \sD  & -\sL \dt
 \end{bmatrix*}
\begin{bmatrix*} 
        \sI &  \sG \dt\\
    {\bf 0} & \sI 
    \end{bmatrix*}\per
\end{align}
Hence
\beq
    \label{eq:lu}
\begin{bmatrix*} 
        \sA & \bf{0}\\
        \sD  & -\sL \dt
 \end{bmatrix*}
\begin{bmatrix*} 
        \sI &  \sG \dt\\
    {\bf 0} & \sI 
    \end{bmatrix*}
    \begin{bmatrix*} 
        \bu^{n+1}\\
        p^{n+1}
    \end{bmatrix*} = \begin{bmatrix*} 
        \bu^{n}-\bN(\bu^n)\dt\\
        \bf{0}\\
    \end{bmatrix*}\com
\eeq
%and therefore the approximate solution is
%\begin{align}
%    \begin{bmatrix*} 
%        \bu^{n+1}\\
%        p^{n+1}
%    \end{bmatrix*} =& 
%    \begin{bmatrix*} 
%        \sI &  \sG \dt\\
%    {\bf 0} & \sI 
%    \end{bmatrix*}^{-1}
%    \begin{bmatrix*} 
%        \sA & \bf{0}\\
%        \sD  & -\sL \dt
%    \end{bmatrix*}^{-1}
%    \begin{bmatrix*} 
%        \bu^{n}-\bN(\bu^n)\dt\\
%        \bf{0}\\
%    \end{bmatrix*} \\
%    = & 
%     \begin{bmatrix*} 
%        \sI &  -\sG \dt\\
%    {\bf 0} & \sI 
%    \end{bmatrix*}
%     \begin{bmatrix*} 
%         \sA^{-1} &  \bf{0}\\
%         \frac{\sL^{-1}\sD\sA^{-1}}{\dt} & -\frac{\sL^{-1}}{\dt} 
%    \end{bmatrix*}
%        \begin{bmatrix*} 
%        \bu^{n}-\bN(\bu^n)\dt\\
%        \bf{0}\\
%    \end{bmatrix*}
%    \per
%\end{align}
%Of course, the expression above assume that all matrices are invertible, which may not be true. To show the equivalence  to the fractional time step method, we work with \eqref{eq:lu}. 

The LU factorization  makes clear that we can solve for $\bu^{n+1}$ and $p^{n+1}$ in two steps
\beq
\label{eq:lu2}
    \begin{bmatrix*} 
        \sI &  \sG \dt\\
    {\bf 0} & \sI 
    \end{bmatrix*}
    \begin{bmatrix*} 
        \bu^{n+1}\\
        p^{n+1}\\
    \end{bmatrix*} 
    =
    \begin{bmatrix*} 
     \bu^*\\
     p^{n+1}
    \end{bmatrix*}\com
\eeq
where
\beq
\label{eq:lu3}
       \begin{bmatrix*} 
        \sA & \bf{0}\\
        \sD  & -\sL \dt
\end{bmatrix*}
\begin{bmatrix*}  
    \bu^{*}\\
        p^{n+1}\\
\end{bmatrix*} 
    =
    \begin{bmatrix*} 
        \bu^n -\dt\bN(\bu^n)\\
         0
    \end{bmatrix*}\per
\eeq
Clearly, \eqref{eq:lu3} is equivalent to the first substep of the fractional method:
The first substep of the fractional method solves the problem without the pressure term
\beq
    \frac{\bu^{*}-\bu^{n}}{\Delta t} =- \bN(\bu^{n}) + \frac{1}{\Re} \sL \bu^{*}\per
\eeq
Hence
\beq
\bu^{*}  =   \sA^{-1}(\bu^n-\dt\bN(\bu^n))\com
\eeq
Then \eqref{eq:lu2} calculates the pressure and updates the velocity 
\beq
\label{eq:pequation}
p^{n+1} = \frac{\sL^{-1}\sD\bu^*}{\dt} = \frac{\sG^{-1}\bu^*}{\dt}\com
\eeq
which is the second row of \eqref{eq:lu}, and
\beq
\label{eq:un1}
 \bu^{n+1} = \bu^* - \dt \sG p^{n+1}\per 
\eeq

\item The fractional step method applies no-slip and no-normal flow conditions in the first step (i.e., on $\bu^*$)
     and no-normal flow condition on the second step (i.e., on $\bu^{n+1}$). From the LU decomposition shown above 
     it is clear that $\bu^*$ doesn't need to satisfy boundary conditions, because it is merely an intermediate variable, introduced to easily solve the system. Using the LU-factorization approach, boundary conditions (no-slip and no-normal flow) must only be enforced on $\bu^{n+1}$.

    \item The analogy does not hold if $\sL\neq\sD\sG$. 

        In particular, in the implication for the pressure equation is
        \beq
                p^{n+1} = \frac{\sL^{-1}\sD\bu^*}{\dt} \neq \frac{\sG^{-1}\bu^*}{\dt}\com
        \eeq
        and consequently
        \beq
            \sL p^{n+1}\neq \frac{\sD \sG \bu^*}{\dt}\per
        \eeq
        For the NS equation, $\sL\neq\sD\sG$ implies mass non-conservation, as discussed below.

 \item  If $\sL \neq \sD \sG$, then $\bu^{n+1}$ does not satisfy mass conservation. Taking the divergence of 
      \eqref{eq:un1}, we obtain
      \beq
      \label{eq:nondiv}
      \sD \bu^{n+1} = \sD \bu^* - \dt \sD\sG p^{n+1}\per
      \eeq
      The right-hand side of \eqref{eq:nondiv} is only zero if $\sL = \sD\sG$. Operating on \eqref{eq:pequation} 
       with $\sL$ and using the result to eliminate $\bu^*$ in \eqref{eq:nondiv}, we obtain
       \beq
       \sD \bu^{n+1} = \dt (\sL - \sD\sG)p^{n+1}\per
       \eeq
        Clearly  $(\sL - \sD\sG)p^{n+1}$ is the rate of volume gain/loss per time-step.

    \item In this case, the LU-decomposition leads to
        \beq
\label{eq:lu4}
    \begin{bmatrix*} 
        \sI &  \sG \dt\\
    {\bf 0} & \sI 
    \end{bmatrix*}
    \begin{bmatrix*} 
        \bu^{n+1}\\
        p^{n+1}\\
    \end{bmatrix*} 
    =
    \begin{bmatrix*} 
     \bu^*\\
     \phi^{n+1}
    \end{bmatrix*}\com
\eeq
where
\beq
\label{eq:lu5}
       \begin{bmatrix*} 
        \sA & \bf{0}\\
        \sD  & -\sL \dt
\end{bmatrix*}
\begin{bmatrix*}  
    \bu^{*}\\
        \phi^{n+1}\\
\end{bmatrix*} 
    =
    \begin{bmatrix*} 
        \bu^n -\dt\bN(\bu^n)-\dt\sG p^n\\
         0
    \end{bmatrix*}\per
\eeq
From \eqref{eq:lu5} we have
\beq
\sD \bu^* = \dt \frac \sL \phi^{n+1}\per
\eeq
Recalling that $\sA = \sI - \tfrac{\dt}{\Re}\sL$. Thus from \eqref{eq:lu4}, for $\sA\sG\dt$ to be consistent with the
 NS equations to second-order in time, we must have
 \beq
 \label{eq:sign}
 p^{n+1} = p^{n} + \phi^{n+1} -\frac{\dt}{\Re}\sL\phi^{n+1} = p^{n} + \phi^{n+1} -\frac{1}{\Re}\sD \bu^*\per
 \eeq
\end{enumerate}
    

\section*{Problem 2}

    \begin{enumerate}
        
            \item We have
                \beq
                    u_{i-1} - 2u_{i} + u_{i+1} = f_i\com
                \eeq
                where $i=1, 2, 3, \ldots, 1024$, and we used $\dx = 1$. 

            \item To solve the problem by relaxation we add a time dependence
                \beq
                    \p_t u_i = \sL u_i - f_i\com
                \eeq
                where $\sL$ is the second-order discrete operator above, and discretize it
                using Crank-Nicolson
                \beq
                \left(\sI - \tfrac{\dt}{2}\sL\right)u_i^{n+1} = \left(\sI + \tfrac{\dt}{2}\sL\right)u_i^{n} - \dt f_i\per
                \eeq
                We then iterate prescribing a $\dt$ and using the Thomas' algorithm to solve for the tridiagonal system
                at every iteration. With $\dt 100$, the solution converges within $10^{-4}$ (as measure by the infinite norm of the residual) after  $12766$ iterations. With $\dt 500$, the convergence is much faster ($5929$ iterations), and again slower at $\dt = 1000$ ($11858$ iterations).

            \item Figure \ref{fig:exps} shows the residual for the different $\dt$s every 100 iterations for the first
                 600 iterations. Clearly, $\dt=100$ converges very fast for highwavenumbers, and painfully slowly for
                 low wavenumbers. On the other hand, the largest timestep $dt=1000$ converges extremely fast for low wavenumbers but very slowly for large wavenumbers. Of course, $\dt=500$ is in between those extrema, and convergence for both large and small wavenumbers is relatively similar. This is the reason $\dt=500$ converges faster (see item a).

            \item The optimum time step that balances convergence for low and large wavenumbers is

                \beq
                \dt_{\text{opt}} = 2 \frac{\dx^2}{\pi^2} \nmax = 2\frac{1024}{\pi^2}\approx 208.
                \eeq
            This number is simply a geometric mean between the optimal time step for convergence of the  largest wavenumber and optimal time step for convergence of the lowest wavenumber. Using this timestep, the solution converges to within $10^{-4}$ after $6527$ iterations, which is the same order of magnitude (but larger) than using $\dt=500$. The structure of the residual is clearly a compromise between that for $\dt=100$ and $\dt=500$.

        \item  Repeating the solution changing the time step every iteration, $\dt=2000\dd \xi$, where $\dd \xi$ is a random variable uniformly distributed between 0 and 1, converges much faster than all other values of $\dt$ explored above. The solution converges to within $10^{-4}$ after $2926$ iterations.   The reason for this faster convergence is that, as shown above, each wavenumber converges faster at different time step, so by randomly choosing, convergence is more rapidly achieved. Figure \ref{fig:exps2} shows the residual every 100 iterations for the first 600 iterations. Clearly, the residual is damped randomly. There is no bias towards high or low wavenumbers decaying faster. For example, at iterations 600, the residual has both low and high wavenumbers.

            Multi-grid methods take advantage of the fact that wavenumbers converge  at different rates. After the residual of the largest wavenumbers has converged, the remaining residual is iterated successively in coarser grids, and then prolonged back successfully into finner grids to find the solution. Repeating this cycle leads to convergence rates much faster than other iterative methods. 

    \end{enumerate}

    \begin{figure}[ht]
    \begin{center}
    \includegraphics[width=10pc,angle=0]{figs/err100_0.png}  
    \includegraphics[width=10pc,angle=0]{figs/err500_0.png}  
    \includegraphics[width=10pc,angle=0]{figs/err1000_0.png}  \\ 
        \includegraphics[width=10pc,angle=0]{figs/err100_1.png}  
    \includegraphics[width=10pc,angle=0]{figs/err500_1.png}  
    \includegraphics[width=10pc,angle=0]{figs/err1000_1.png}  \\
        \includegraphics[width=10pc,angle=0]{figs/err100_2.png}  
    \includegraphics[width=10pc,angle=0]{figs/err500_2.png}  
    \includegraphics[width=10pc,angle=0]{figs/err1000_2.png}  \\
        \includegraphics[width=10pc,angle=0]{figs/err100_3.png}  
    \includegraphics[width=10pc,angle=0]{figs/err500_3.png}  
    \includegraphics[width=10pc,angle=0]{figs/err1000_3.png}  \\
        \includegraphics[width=10pc,angle=0]{figs/err100_4.png}  
    \includegraphics[width=10pc,angle=0]{figs/err500_4.png}  
    \includegraphics[width=10pc,angle=0]{figs/err1000_4.png}  \\
        \includegraphics[width=10pc,angle=0]{figs/err100_5.png}  
    \includegraphics[width=10pc,angle=0]{figs/err500_5.png}  
    \includegraphics[width=10pc,angle=0]{figs/err1000_5.png}  \\ 
    \end{center}
    \caption{The residual for the first 600 iterations. Clearly, 
        $\dt=100$ converges faster with for high wavenumbers (and convergence of low wavenumber is painfully slow), whereas $\dt=1000$ converges faster for low wavenumbers (and painfully slowly for large wavenumbers).}
    \label{fig:exps}
    \end{figure}


    \begin{figure}[ht]
    \begin{center}
    \includegraphics[width=10pc,angle=0]{figs/erropt_0.png} 
        \includegraphics[width=10pc,angle=0]{figs/erropt_1.png}  
        \includegraphics[width=10pc,angle=0]{figs/erropt_2.png}\\  
        \includegraphics[width=10pc,angle=0]{figs/erropt_3.png} 
        \includegraphics[width=10pc,angle=0]{figs/erropt_4.png}  
        \includegraphics[width=10pc,angle=0]{figs/erropt_5.png}  
    \end{center}
    \caption{The residual for the first 600 iterations with $\dt = \dt_{\text{opt}}\approx 208$.}
    \label{fig:exps1}
    \end{figure}


    \begin{figure}[ht]
    \begin{center}
    \includegraphics[width=10pc,angle=0]{figs/errrand_0.png} 
        \includegraphics[width=10pc,angle=0]{figs/errrand_1.png}  
        \includegraphics[width=10pc,angle=0]{figs/errrand_2.png}\\  
        \includegraphics[width=10pc,angle=0]{figs/errrand_3.png} 
        \includegraphics[width=10pc,angle=0]{figs/errrand_4.png}  
        \includegraphics[width=10pc,angle=0]{figs/errrand_5.png}  
    \end{center}
    \caption{The residual for the first 600 iterations with random $\dt$ between 0 and 2000.}
    \label{fig:exps2}
    \end{figure}

\end{document}
