\documentclass[11pt]{article}


%% WRY has commented out some unused packages %%
%% If needed, activate these by uncommenting
\usepackage{geometry}                % See geometry.pdf to learn the layout options. There are lots.
%\geometry{letterpaper}                   % ... or a4paper or a5paper or ... 
\geometry{a4paper,left=2.5cm,right=2.5cm,top=2.5cm,bottom=2.5cm}
%\geometry{landscape}                % Activate for rotated page geometry
%\usepackage[parfill]{parskip}    % Activate to begin paragraphs with an empty line rather than an indent

%for figures
%\usepackage{graphicx}

\usepackage{color}
\definecolor{mygreen}{RGB}{28,172,0} % color values Red, Green, Blue
\definecolor{mylilas}{RGB}{170,55,241}
%% for graphics this one is also OK:
\usepackage{epsfig}

%% AMS mathsymbols are enabled with
\usepackage{amssymb,amsmath}

%% more options in enumerate
\usepackage{enumerate}
\usepackage{enumitem}

%% insert code
\usepackage{listings}

\usepackage[utf8]{inputenc}

\usepackage{hyperref}

%% To make really wide whats that cover everything:
\usepackage{scalerel}
\usepackage{stackengine}
\stackMath
\def\hatgap{2pt}
\def\subdown{-2pt}
\newcommand\what[2][]{%
\renewcommand\stackalignment{l}%
\stackon[\hatgap]{#2}{%
\stretchto{%
    \scalerel*[\widthof{$#2$}]{\kern-.6pt\bigwedge\kern-.6pt}%
    {\rule[-\textheight/2]{1ex}{\textheight}}%WIDTH-LIMITED BIG WEDGE
}{0.5ex}% THIS SQUEEZES THE WEDGE TO 0.5ex HEIGHT
_{\smash{\belowbaseline[\subdown]{\scriptstyle#1}}}%
}}

% Default fixed font does not support bold face
\DeclareFixedFont{\ttb}{T1}{txtt}{bx}{n}{12} % for bold
\DeclareFixedFont{\ttm}{T1}{txtt}{m}{n}{12}  % for normal

% Custom colors
\usepackage{color}
\definecolor{deepblue}{rgb}{0,0,0.5}
\definecolor{deepred}{rgb}{0.6,0,0}
\definecolor{deepgreen}{rgb}{0,0.5,0}


% Python style for highlighting
\newcommand\pythonstyle{\lstset{
language=Python,
basicstyle=\ttm,
otherkeywords={self},             % Add keywords here
keywordstyle=\ttb\color{deepblue},
emph={MyClass,__init__},          % Custom highlighting
emphstyle=\ttb\color{deepred},    % Custom highlighting style
stringstyle=\color{deepgreen},
frame=tb,                         % Any extra options here
showstringspaces=false            % 
}}

% Python environment
\lstnewenvironment{python}[1][]
{
\pythonstyle
\lstset{#1}
}
{}

% Python for external files
\newcommand\pythonexternal[2][]{{
\pythonstyle
\lstinputlisting[#1]{#2}}}

% Python for inline
\newcommand\pythoninline[1]{{\pythonstyle\lstinline!#1!}}

%% colors
\usepackage{graphicx,xcolor,lipsum}


\usepackage{mathtools}

\usepackage{graphicx}
\newcommand*{\matminus}{%
  \leavevmode
  \hphantom{0}%
  \llap{%
    \settowidth{\dimen0 }{$0$}%
    \resizebox{1.1\dimen0 }{\height}{$-$}%
  }%
}


\title{MAE290C, Homework Assignment 3}
\author{Cesar B Rocha}
\date{\today}

\begin{document}

\newcommand{\com}{\, ,}
\newcommand{\per}{\, .}

%% Averages
% Use \bar to over line solo symbols

\newcommand{\av}[1]{\bar{#1}}
\newcommand{\avbg}[1]{\overline{#1}}
\newcommand{\avbgg}[1]{\overline{#1}}

% A nice definition
\newcommand{\defn}{\ensuremath{\stackrel{\mathrm{def}}{=}}}

% equations
\def\beq{\begin{equation}}
\def\eeq{\end{equation}}

% calculus

\newcommand{\p}{\partial}
\newcommand{\ii}{{\rm i}}
\newcommand{\dd}{{\rm d}}
\newcommand{\id}{{\, \rm d}}
\newcommand{\ee}{{\rm e}}
\newcommand{\DD}{{\rm D}}
\newcommand{\wavy}{\text{wavy}}
\newcommand{\qg}{\text{qg}}


\newcommand{\be}{\beta}

\newcommand{\al}{\alpha}
\newcommand{\bx}{\boldsymbol{x}}
\newcommand{\by}{\boldsymbol{y}}
\newcommand{\bu}{\boldsymbol{u}}
\newcommand{\bv}{\boldsymbol{v}}


\newcommand{\half}{\tfrac{1}{2}}
\newcommand{\halfrho}{\tfrac{1}{2}}
\newcommand{\rz}{{}}
\newcommand{\bn}{\boldsymbol{\hat n}}
\newcommand{\br}{\boldsymbol{r}}
\newcommand{\bR}{\boldsymbol{R}}
\newcommand{\bA}{\ensuremath {\boldsymbol {A}}}
\newcommand{\bB}{\ensuremath {\boldsymbol {B}}}
\newcommand{\bU}{\ensuremath {\boldsymbol {U}}}
\newcommand{\bE}{\ensuremath {\boldsymbol {E}}}
\newcommand{\bJ}{\ensuremath {\boldsymbol {J}}}
\newcommand{\bXX}{\ensuremath {\boldsymbol {\mathcal{X}}}}
\newcommand{\bFF}{\ensuremath {\boldsymbol {F}}}
\newcommand{\bF}{\ensuremath {\boldsymbol {F}^{\sharp}}}
\newcommand{\bG}{\ensuremath {\boldsymbol G}}
\newcommand{\bSigma}{\ensuremath {\boldsymbol {\Sigma}}}
\newcommand{\bvarphi}{\ensuremath {\boldsymbol {\varphi}}}
\newcommand{\bxi}{\ensuremath {\boldsymbol {\xi}}}
\newcommand{\avbxi}{\overline{\ensuremath {\boldsymbol {\xi}}}}

% math cal

\newcommand{\J}{\mathcal{J}}
\newcommand{\K}{\mathcal{K}}
\newcommand{\cG}{\mathcal{G}}
\newcommand{\cF}{\mathcal{F}}
\newcommand{\cN}{\mathcal{N}}
\newcommand{\cL}{\mathcal{L}}


% san serif for matrices and differential operators
%\newcommand{\helmn}{\mathsf{H}_n}
\newcommand{\helmm}{\triangle_m}
\newcommand{\helmn}{\triangle_n}
\newcommand{\helms}{\triangle_s}
\newcommand{\helm}{\triangle}
\newcommand{\sA}{\mathsf{A}}
\newcommand{\sB}{\mathsf{B}}
\newcommand{\sG}{\mathsf{G}}
\newcommand{\sI}{\mathsf{I}}
\newcommand{\sJ}{\mathsf{J}}
\newcommand{\sU}{\mathsf{U}}
\newcommand{\sP}{\mathsf{P}}
\newcommand{\sQ}{\mathsf{Q}}
\newcommand{\sR}{\mathsf{R}}
\newcommand{\sL}{\mathsf{L}}
\renewcommand{\sJ}{\mathsf{J}}
\renewcommand{\sI}{\mathsf{I}}
\renewcommand{\L}{\mathsf{L}}
\newcommand{\sM}{\mathsf{M}}
\newcommand{\sT}{\mathsf{T}}
\newcommand{\sGamma}{\mathsf{\Gamma}}
\newcommand{\sOmega}{\mathsf{\Omega}}
\newcommand{\sSigma}{\mathsf{\Omega}}
\newcommand{\sbeta}{\mathsf{\beta}}
\newcommand{\sPi}{\mathsf{\Pi}}
\newcommand{\sC}{\mathsf{C}}
\newcommand{\sQy}{\mathsf{Q}}


% angle brackets

\def\la{\langle}
\def\ra{\rangle}
\def\laa{\left \langle}
\def\raa{\right \rangle}


%grads and div's
\newcommand{\bcdot}{\hspace{-0.1em} \boldsymbol{\cdot} \hspace{-0.12em}}
\newcommand{\bnabla}{\boldsymbol{\nabla}}
\newcommand{\bnablaH}{\bnabla_{\! \mathrm{h}}}
\newcommand{\grad}{\bnabla}
\newcommand{\gradH}{\bnablaH}
\newcommand{\curl}{\bnabla \!\times\!}
\newcommand{\diver}{\bnabla \bcdot }
\newcommand{\cross}{\times}
\newcommand{\lap}{\triangle}


%varthetas and thetas
\newcommand{\vth}{\vartheta}
\newcommand{\psii}{\psi^{\mathrm{i}}}
\newcommand{\thb}{\theta^{\mathrm{-}}}
\newcommand{\vthb}{\vartheta^{\mathrm{-}}}
\newcommand{\vthbhat}{{\hat{\vartheta}}^{\mathrm{-}}}
\newcommand{\vThb}{\varTheta^{\mathrm{-}}}
\newcommand{\psib}{\psi^{\mathrm{-}}}
\newcommand{\tht}{\theta^{\mathrm{+}}}
\newcommand{\vtht}{\vartheta^{\mathrm{+}}}
\newcommand{\vththat}{{\hat{\vartheta}}^{\mathrm{+}}}
\newcommand{\vthtbhat}{{\hat{\vartheta}}^{\pm}}
\newcommand{\vTht}{\varTheta^{\mathrm{+}}}
\newcommand{\vthtb}{\vartheta^{\pm}}
\newcommand{\vThtb}{\varTheta^{\pm}}

% nondimensional numbers
\renewcommand{\Re}{\mathrm{Re}}
\newcommand{\Ro}{\mathrm{Ro}}
\newcommand{\Ri}{\mathrm{Ri}}

%psi's
%Galerking coefficient for psi:
\newcommand{\gpsi}{\breve \psi}
\newcommand{\gphi}{\breve \phi}
\newcommand{\gq}{\breve q}
\newcommand{\gU}{\breve U}
\newcommand{\gQ}{\breve Q}
\newcommand{\gsigma}{\breve \sigma}


\newcommand{\psit}{\psi^{\mathrm{+}}}
\newcommand{\psithat}{{\hat{\psi}}^{\mathrm{+}}}
\newcommand{\psibhat}{{\hat{\psi}}^{\mathrm{-}}}
\newcommand{\psitb}{\psi^{\pm}}
\newcommand{\psitbhat}{{\hat{\psi}}^\pm}
\newcommand{\St}{S^{\mathrm{+}}}
\newcommand{\Sb}{S^{\mathrm{-}}}
\newcommand{\phb}{\phi^{\mathrm{-}}}
\newcommand{\pht}{\phi^{\mathrm{+}}}
\newcommand{\tautb}{\tau^{\pm}}
\newcommand{\sigmatb}{\sigma^{\pm}}


\newcommand{\bur}{\left(\tfrac{f_0}{N}\right)^2}
\newcommand{\ibur}{\left(\tfrac{N}{f_0}\right)^2}
\newcommand{\Nm}{N_{\mathrm{mix}}}
\newcommand{\xim}{\xi_{\mathrm{mix}}}
\newcommand{\hs}{h_*}
\renewcommand{\sp}{\mathsf{p}}
\newcommand{\se}{\mathsf{e}}
\newcommand{\sptb}{\mathsf{p}^\pm}


%nmax is a problem:
%\newcommand{\nmax}{n_{\mathrm{max}}}
\newcommand{\nmax}{\mathrm{N}}

\newcommand{\WKB}{\mathrm{WKB}}
\newcommand{\Lam}{\Lambda}
\newcommand{\tha}{\theta}
\newcommand{\kap}{\kappa}
\newcommand{\bphi}{\boldsymbol{\phi}}
\newcommand{\third}{\tfrac{1}{3}}
\newcommand{\cs}{c^\star}
\newcommand{\nt}{n^{\mathrm{trnc}}}
\newcommand{\sDp}{\mathsf{D}^1_{\nmax}}
\newcommand{\sDpp}{\mathsf{D}^2_{\nmax}}
\newcommand{\sD}{\mathsf{D_2}}
\newcommand{\sK}{\mathsf{K_2}}
\newcommand{\stheta}{\mathsf{\theta}}
\newcommand{\sphi}{\mathsf{\phi}}
\newcommand{\sq}{\mathsf{q}}
\newcommand{\cosech}{\text{csch}\,}
\newcommand{\sinc}{\text{sinc}\,}

%%%%%%%%% %%%%
\newcommand{\zp}{z^+}
\newcommand{\zm}{z^-}
\newcommand{\qA}{q^A_{\nmax}}
\newcommand{\psiB}{\psi^B_{\nmax}}
\newcommand{\phiB}{\phi^B_{\nmax}}
\newcommand{\eye}{\boldsymbol{\hat{i}}}
\newcommand{\jay}{\boldsymbol{\hat{j}}}
\newcommand{\kay}{\boldsymbol{\hat{k}}}
\newcommand{\psiG}{\psi^{\mathrm{G}}}
\newcommand{\qG}{q^{\mathrm{G}}}
\newcommand{\uG}{u^{\mathrm{G}}}
\newcommand{\UG}{U^{\mathrm{G}}}
\newcommand{\UGN}{U^{\mathrm{G}}_{\nmax}}
\newcommand{\QGN}{Q^{\mathrm{G}}_{\nmax}}
\newcommand{\sumoddn}{\sum_{n = 1, n~ \text{odd}}^{\nmax}}

% bretherton 
\newcommand{\qBr}{q_{\mathrm{Br}}}
\newcommand{\psiBr}{\psi_{\mathrm{Br}}}


\maketitle

\section*{Problem 1}


The non-dimensional Navier-Stokes equations, with constant density, are
\beq
\label{eq:ns1}
\p_t \bu = - \underbrace{\bu\cdot\nabla\bu}_{\defn \bN(\bu)}  -\nabla p + \frac{1}{\Re}\nabla\cdot\nabla\bu\com 
\eeq
and
\beq
\label{eq:ns2}
\nabla\cdot\bu = 0\com
\eeq
where $\bu$ is the velocity, $p$ is the dynamic pressure, and $\Re$ is the Reynolds number. We use the following
 notation for the discrete linear operators
 \beq
    \sD \approx \nabla\cdot\com\qquad  \sG \approx\nabla \com\qqand\sL =  \sD \sG\com
 \eeq   
 where $\sD$, $\sG$ and $\sL$ should be understood as matrices.

\begin{enumerate}


\item Time-discretizing \eqref{eq:ns1} and \eqref{eq:ns2} we obtain using explicit Euler for the non-linear term
 and implicit Euler for the linear terms, we obtain
 \beq
    \label{eq:ns1_d}
    \frac{\bu^{n+1}-\bu^{n}}{\Delta t} \approx \bN(\bu^{n}) - \sG \bu^{n+1} + \frac{1}{\Re} \sD\sG \bu^{n+1}\com 
 \eeq
and
\beq
    \label{eq:ns2_d}
    \sD u^{n+1} = 0\per
\eeq
We can re-cast \eqref{eq:ns1_d} and \eqref{eq:ns2_d} in block-matrix form
\beq
    \label{eq:ns_d}
    \underbrace{
    \begin{bmatrix*} 
        \sA & \sG \dt\\
        \sD & {\bf 0}
    \end{bmatrix*}}_{\defn \sM} 
    \begin{bmatrix*} 
        \bu^{n+1}\\
        p^{n+1}
    \end{bmatrix*} = \begin{bmatrix*} 
        \bu^{n}-\bN(\bu^n)\dt\\
        \bf{0}\\
    \end{bmatrix*}\com
\eeq
where
\beq
\sA \defn \sI - \frac{\dt}{\Re} \sL\per
\eeq

\item We have
    \beq
        \sA \sG  = \sG - \frac{\dt}{\Re} \sL\per
    \eeq
    Thus, to within an error of $\mathcal{O}(\dt)$, we can approximate \ref{eq:ns_d} as
\beq
    \label{eq:ns_d_2}
    \underbrace{
    \begin{bmatrix*} 
        \sA & \sA \sG \dt\\
        \sD & {\bf 0}
    \end{bmatrix*}}_{\defn \sM_2} 
    \begin{bmatrix*} 
        \bu^{n+1}\\
        p^{n+1}
    \end{bmatrix*} = \begin{bmatrix*} 
        \bu^{n}-\bN(\bu^n)\dt\\
        \bf{0}\\
    \end{bmatrix*}\per
\eeq

    \item To obtain the LU-factorization, we note that we can operate on the block-matrix $\sM_2$ as in
         standard matrices. Hence, we can perform a block Gauss elimination  on $\sM_2$. Noticing that the 
          block multiplier is simply $\sD \sA^{-1}$, we obtain

\begin{align}
    \sM_2 = 
    \begin{bmatrix*} 
        \sA & \sA \sG \dt\\
        \sD & {\bf 0}
    \end{bmatrix*} = & 
    \begin{bmatrix*} 
        \sI & 0\\
        \sD \sA^{-1} & \sI
    \end{bmatrix*}
\begin{bmatrix*} 
        \sA & \sA \sG \dt\\
    {\bf 0} & -\sL\dt 
    \end{bmatrix*}\\
    = &    \begin{bmatrix*} 
        \sI & \bf{0}\\
        \sD \sA^{-1} & \sI
 \end{bmatrix*}
 \begin{bmatrix*} 
     \sA & \bf{0}\\
     \bf{0} & -\sL \dt
    \end{bmatrix*}
\begin{bmatrix*} 
        \sI &  \sG \dt\\
    {\bf 0} & - \sI 
    \end{bmatrix*}\\
    = &
\begin{bmatrix*} 
        \sA & \bf{0}\\
        \sD  & -\sL \dt
 \end{bmatrix*}
\begin{bmatrix*} 
        \sI &  \sG \dt\\
    {\bf 0} & \sI 
    \end{bmatrix*}\per
\end{align}
Hence
\beq
    \label{eq:ns_d_3}
\begin{bmatrix*} 
        \sA & \bf{0}\\
        \sD  & -\sL \dt
 \end{bmatrix*}
\begin{bmatrix*} 
        \sI &  \sG \dt\\
    {\bf 0} & \sI 
    \end{bmatrix*}
    \begin{bmatrix*} 
        \bu^{n+1}\\
        p^{n+1}
    \end{bmatrix*} = \begin{bmatrix*} 
        \bu^{n}-\bN(\bu^n)\dt\\
        \bf{0}\\
    \end{bmatrix*}\com
\eeq
and therefore the approximate solution is
\begin{align}
    \label{eq:ns_d_4}
    \begin{bmatrix*} 
        \bu^{n+1}\\
        p^{n+1}
    \end{bmatrix*} =& 
    \begin{bmatrix*} 
        \sI &  \sG \dt\\
    {\bf 0} & \sI 
    \end{bmatrix*}^{-1}
    \begin{bmatrix*} 
        \sA & \bf{0}\\
        \sD  & -\sL \dt
    \end{bmatrix*}^{-1}
    \begin{bmatrix*} 
        \bu^{n}-\bN(\bu^n)\dt\\
        \bf{0}\\
    \end{bmatrix*} \\
    = & 
     \begin{bmatrix*} 
        \sI &  -\sG \dt\\
    {\bf 0} & \sI 
    \end{bmatrix*}
     \begin{bmatrix*} 
         \sA^{-1} &  \bf{0}\\
         \frac{\sL^{-1}\sD^{-1}\sA^{-1}}{\dt} & -\frac{\sL^{-1}}{\dt} 
    \end{bmatrix*}
        \begin{bmatrix*} 
        \bu^{n}-\bN(\bu^n)\dt\\
        \bf{0}\\
    \end{bmatrix*}
    \per
\end{align}




\end{enumerate}
    


%    \begin{figure}[ht]
%    \begin{center}
%    \includegraphics[width=17pc,angle=0]{pb3/experiments/pb3_3/figs/burgers_phys_16.png} 
%    \includegraphics[width=17pc,angle=0]{pb3/experiments/pb3_3/figs/burgers_phys_64.png}\\
%    \includegraphics[width=17pc,angle=0]{pb3/experiments/pb3_3/figs/burgers_phys_128.png}
%    \includegraphics[width=17pc,angle=0]{pb3/experiments/pb3_3/figs/burgers_phys_2048.png}
%    \end{center}
%    \caption{Results in physical domain from simulation of 1D Burgers equations with $\nu = 10^{-3}$ and various N.}
%    \label{fig:pb3_3_phys}
%    \end{figure}

\end{document}
