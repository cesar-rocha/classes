\documentclass[11pt]{article}


%% WRY has commented out some unused packages %%
%% If needed, activate these by uncommenting
\usepackage{geometry}                % See geometry.pdf to learn the layout options. There are lots.
%\geometry{letterpaper}                   % ... or a4paper or a5paper or ... 
\geometry{a4paper,left=2.5cm,right=2.5cm,top=2.5cm,bottom=2.5cm}
%\geometry{landscape}                % Activate for rotated page geometry
%\usepackage[parfill]{parskip}    % Activate to begin paragraphs with an empty line rather than an indent

%for figures
%\usepackage{graphicx}

\usepackage{color}
\definecolor{mygreen}{RGB}{28,172,0} % color values Red, Green, Blue
\definecolor{mylilas}{RGB}{170,55,241}
%% for graphics this one is also OK:
\usepackage{epsfig}

%% AMS mathsymbols are enabled with
\usepackage{amssymb,amsmath}

%% more options in enumerate
\usepackage{enumerate}
\usepackage{enumitem}

%% insert code
\usepackage{listings}

\usepackage[utf8]{inputenc}

\usepackage{hyperref}

%% To make really wide whats that cover everything:
\usepackage{scalerel}
\usepackage{stackengine}
\stackMath
\def\hatgap{2pt}
\def\subdown{-2pt}
\newcommand\what[2][]{%
\renewcommand\stackalignment{l}%
\stackon[\hatgap]{#2}{%
\stretchto{%
    \scalerel*[\widthof{$#2$}]{\kern-.6pt\bigwedge\kern-.6pt}%
    {\rule[-\textheight/2]{1ex}{\textheight}}%WIDTH-LIMITED BIG WEDGE
}{0.5ex}% THIS SQUEEZES THE WEDGE TO 0.5ex HEIGHT
_{\smash{\belowbaseline[\subdown]{\scriptstyle#1}}}%
}}

% Default fixed font does not support bold face
\DeclareFixedFont{\ttb}{T1}{txtt}{bx}{n}{12} % for bold
\DeclareFixedFont{\ttm}{T1}{txtt}{m}{n}{12}  % for normal

% Custom colors
\usepackage{color}
\definecolor{deepblue}{rgb}{0,0,0.5}
\definecolor{deepred}{rgb}{0.6,0,0}
\definecolor{deepgreen}{rgb}{0,0.5,0}


% Python style for highlighting
\newcommand\pythonstyle{\lstset{
language=Python,
basicstyle=\ttm,
otherkeywords={self},             % Add keywords here
keywordstyle=\ttb\color{deepblue},
emph={MyClass,__init__},          % Custom highlighting
emphstyle=\ttb\color{deepred},    % Custom highlighting style
stringstyle=\color{deepgreen},
frame=tb,                         % Any extra options here
showstringspaces=false            % 
}}

% Python environment
\lstnewenvironment{python}[1][]
{
\pythonstyle
\lstset{#1}
}
{}

% Python for external files
\newcommand\pythonexternal[2][]{{
\pythonstyle
\lstinputlisting[#1]{#2}}}

% Python for inline
\newcommand\pythoninline[1]{{\pythonstyle\lstinline!#1!}}

%% colors
\usepackage{graphicx,xcolor,lipsum}


\usepackage{mathtools}

\usepackage{graphicx}
\newcommand*{\matminus}{%
  \leavevmode
  \hphantom{0}%
  \llap{%
    \settowidth{\dimen0 }{$0$}%
    \resizebox{1.1\dimen0 }{\height}{$-$}%
  }%
}


\title{MAE290C, Homework Assignment 3}
\author{Cesar B Rocha}
\date{\today}

\begin{document}

\include{mysymbols}
\maketitle

\section*{Problem 1}


The non-dimensional Navier-Stokes equations, with constant density, are
\beq
\label{eq:ns1}
\p_t \bu = - \underbrace{\bu\cdot\nabla\bu}_{\defn \bN(\bu)}  -\nabla p + \frac{1}{\Re}\nabla\cdot\nabla\bu\com 
\eeq
and
\beq
\label{eq:ns2}
\nabla\cdot\bu = 0\com
\eeq
where $\bu$ is the velocity, $p$ is the dynamic pressure, and $\Re$ is the Reynolds number. We use the following
 notation for the discrete linear operators
 \beq
    \sD \approx \nabla\cdot\com\qquad  \sG \approx\nabla \com\qqand\sL =  \sD \sG\com
 \eeq   
 where $\sD$, $\sG$ and $\sL$ should be understood as matrices.

\begin{enumerate}


\item Time-discretizing \eqref{eq:ns1} and \eqref{eq:ns2} we obtain using explicit Euler for the non-linear term
 and implicit Euler for the linear terms, we obtain
 \beq
    \label{eq:ns1_d}
    \frac{\bu^{n+1}-\bu^{n}}{\Delta t} \approx \bN(\bu^{n}) - \sG \bu^{n+1} + \frac{1}{\Re} \sD\sG \bu^{n+1}\com 
 \eeq
and
\beq
    \label{eq:ns2_d}
    \sD u^{n+1} = 0\per
\eeq
We can re-cast \eqref{eq:ns1_d} and \eqref{eq:ns2_d} in block-matrix form
\beq
    \label{eq:ns_d}
    \underbrace{
    \begin{bmatrix*} 
        \sA & \sG \dt\\
        \sD & {\bf 0}
    \end{bmatrix*}}_{\defn \sM} 
    \begin{bmatrix*} 
        \bu^{n+1}\\
        p^{n+1}
    \end{bmatrix*} = \begin{bmatrix*} 
        \bu^{n}-\bN(\bu^n)\dt\\
        \bf{0}\\
    \end{bmatrix*}\com
\eeq
where
\beq
\sA \defn \sI - \frac{\dt}{\Re} \sL\per
\eeq

\item We have
    \beq
        \sA \sG  = \sG - \frac{\dt}{\Re} \sL\per
    \eeq
    Thus, to within an error of $\mathcal{O}(\dt)$, we can approximate \ref{eq:ns_d} as
\beq
    \label{eq:ns_d_2}
    \underbrace{
    \begin{bmatrix*} 
        \sA & \sA \sG \dt\\
        \sD & {\bf 0}
    \end{bmatrix*}}_{\defn \sM_2} 
    \begin{bmatrix*} 
        \bu^{n+1}\\
        p^{n+1}
    \end{bmatrix*} = \begin{bmatrix*} 
        \bu^{n}-\bN(\bu^n)\dt\\
        \bf{0}\\
    \end{bmatrix*}\per
\eeq

    \item To obtain the LU-factorization, we note that we can operate on the block-matrix $\sM_2$ as in
         standard matrices. Hence, we can perform a block Gauss elimination  on $\sM_2$. Noticing that the 
          block multiplier is simply $\sD \sA^{-1}$, we obtain

\begin{align}
    \sM_2 = 
    \begin{bmatrix*} 
        \sA & \sA \sG \dt\\
        \sD & {\bf 0}
    \end{bmatrix*} = & 
    \begin{bmatrix*} 
        \sI & 0\\
        \sD \sA^{-1} & \sI
    \end{bmatrix*}
\begin{bmatrix*} 
        \sA & \sA \sG \dt\\
    {\bf 0} & -\sL\dt 
    \end{bmatrix*}\\
    = &    \begin{bmatrix*} 
        \sI & \bf{0}\\
        \sD \sA^{-1} & \sI
 \end{bmatrix*}
 \begin{bmatrix*} 
     \sA & \bf{0}\\
     \bf{0} & -\sL \dt
    \end{bmatrix*}
\begin{bmatrix*} 
        \sI &  \sG \dt\\
    {\bf 0} & - \sI 
    \end{bmatrix*}\\
    = &
\begin{bmatrix*} 
        \sA & \bf{0}\\
        \sD  & -\sL \dt
 \end{bmatrix*}
\begin{bmatrix*} 
        \sI &  \sG \dt\\
    {\bf 0} & \sI 
    \end{bmatrix*}\per
\end{align}
Hence
\beq
    \label{eq:ns_d_3}
\begin{bmatrix*} 
        \sA & \bf{0}\\
        \sD  & -\sL \dt
 \end{bmatrix*}
\begin{bmatrix*} 
        \sI &  \sG \dt\\
    {\bf 0} & \sI 
    \end{bmatrix*}
    \begin{bmatrix*} 
        \bu^{n+1}\\
        p^{n+1}
    \end{bmatrix*} = \begin{bmatrix*} 
        \bu^{n}-\bN(\bu^n)\dt\\
        \bf{0}\\
    \end{bmatrix*}\com
\eeq
and therefore the approximate solution is
\begin{align}
    \label{eq:ns_d_4}
    \begin{bmatrix*} 
        \bu^{n+1}\\
        p^{n+1}
    \end{bmatrix*} =& 
    \begin{bmatrix*} 
        \sI &  \sG \dt\\
    {\bf 0} & \sI 
    \end{bmatrix*}^{-1}
    \begin{bmatrix*} 
        \sA & \bf{0}\\
        \sD  & -\sL \dt
    \end{bmatrix*}^{-1}
    \begin{bmatrix*} 
        \bu^{n}-\bN(\bu^n)\dt\\
        \bf{0}\\
    \end{bmatrix*} \\
    = & 
     \begin{bmatrix*} 
        \sI &  -\sG \dt\\
    {\bf 0} & \sI 
    \end{bmatrix*}
     \begin{bmatrix*} 
         \sA^{-1} &  \bf{0}\\
         \frac{\sL^{-1}\sD^{-1}\sA^{-1}}{\dt} & -\frac{\sL^{-1}}{\dt} 
    \end{bmatrix*}
        \begin{bmatrix*} 
        \bu^{n}-\bN(\bu^n)\dt\\
        \bf{0}\\
    \end{bmatrix*}
    \per
\end{align}




\end{enumerate}
    


%    \begin{figure}[ht]
%    \begin{center}
%    \includegraphics[width=17pc,angle=0]{pb3/experiments/pb3_3/figs/burgers_phys_16.png} 
%    \includegraphics[width=17pc,angle=0]{pb3/experiments/pb3_3/figs/burgers_phys_64.png}\\
%    \includegraphics[width=17pc,angle=0]{pb3/experiments/pb3_3/figs/burgers_phys_128.png}
%    \includegraphics[width=17pc,angle=0]{pb3/experiments/pb3_3/figs/burgers_phys_2048.png}
%    \end{center}
%    \caption{Results in physical domain from simulation of 1D Burgers equations with $\nu = 10^{-3}$ and various N.}
%    \label{fig:pb3_3_phys}
%    \end{figure}

\end{document}
