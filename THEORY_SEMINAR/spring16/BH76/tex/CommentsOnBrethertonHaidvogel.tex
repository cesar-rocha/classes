\documentclass[11pt]{article}


%% WRY has commented out some unused packages %%
%% If needed, activate these by uncommenting
\usepackage{geometry}                % See geometry.pdf to learn the layout options. There are lots.
%\geometry{letterpaper}                   % ... or a4paper or a5paper or ... 
\geometry{a4paper,left=2.5cm,right=2.5cm,top=2.5cm,bottom=2.5cm}
%\geometry{landscape}                % Activate for rotated page geometry
%\usepackage[parfill]{parskip}    % Activate to begin paragraphs with an empty line rather than an indent

\usepackage{natbib}

%for figures
%\usepackage{graphicx}

\usepackage{color}
\definecolor{mygreen}{RGB}{28,172,0} % color values Red, Green, Blue
\definecolor{mylilas}{RGB}{170,55,241}
%% for graphics this one is also OK:
\usepackage{epsfig}

%% AMS mathsymbols are enabled with
\usepackage{amssymb,amsmath}

%% more options in enumerate
\usepackage{enumerate}
\usepackage{enumitem}

%% insert code
\usepackage{listings}

\usepackage[utf8]{inputenc}

\usepackage{hyperref}

%% To make really wide whats that cover everything:
\usepackage{scalerel}
\usepackage{stackengine}
\stackMath
\def\hatgap{2pt}
\def\subdown{-2pt}
\newcommand\what[2][]{%
\renewcommand\stackalignment{l}%
\stackon[\hatgap]{#2}{%
\stretchto{%
    \scalerel*[\widthof{$#2$}]{\kern-.6pt\bigwedge\kern-.6pt}%
    {\rule[-\textheight/2]{1ex}{\textheight}}%WIDTH-LIMITED BIG WEDGE
}{0.5ex}% THIS SQUEEZES THE WEDGE TO 0.5ex HEIGHT
_{\smash{\belowbaseline[\subdown]{\scriptstyle#1}}}%
}}

% Default fixed font does not support bold face
\DeclareFixedFont{\ttb}{T1}{txtt}{bx}{n}{12} % for bold
\DeclareFixedFont{\ttm}{T1}{txtt}{m}{n}{12}  % for normal

% Custom colors
\usepackage{color}
\definecolor{deepblue}{rgb}{0,0,0.5}
\definecolor{deepred}{rgb}{0.6,0,0}
\definecolor{deepgreen}{rgb}{0,0.5,0}


% Python style for highlighting
\newcommand\pythonstyle{\lstset{
language=Python,
basicstyle=\ttm,
otherkeywords={self},             % Add keywords here
keywordstyle=\ttb\color{deepblue},
emph={MyClass,__init__},          % Custom highlighting
emphstyle=\ttb\color{deepred},    % Custom highlighting style
stringstyle=\color{deepgreen},
frame=tb,                         % Any extra options here
showstringspaces=false            % 
}}

% Python environment
\lstnewenvironment{python}[1][]
{
\pythonstyle
\lstset{#1}
}
{}

% Python for external files
\newcommand\pythonexternal[2][]{{
\pythonstyle
\lstinputlisting[#1]{#2}}}

% Python for inline
\newcommand\pythoninline[1]{{\pythonstyle\lstinline!#1!}}

%% colors
\usepackage{graphicx,xcolor,lipsum}


\usepackage{mathtools}

\usepackage{graphicx}
\newcommand*{\matminus}{%
  \leavevmode
  \hphantom{0}%
  \llap{%
    \settowidth{\dimen0 }{$0$}%
    \resizebox{1.1\dimen0 }{\height}{$-$}%
  }%
}

\usepackage{import}
\import{/Users/crocha/Dropbox/tex/}{symbols}

\title{Notes on \\ {\it Two-dimensional turbulence above topography, JFM 1976 }\\ by Bretherton and Haidvogel}
\date{}
\author{Cesar B Rocha}
\date{\today}

\begin{document}

\renewcommand{\lap}{\nabla^2}
\renewcommand{\mu}{k_0^{2}}

\maketitle

\newcommand{\BH}{\text{Bretherton and Haidvogel }}

\section{Phenomenology of two-dimensional turbulence}

The two invariants of the two-dimensional system are (kinetic) energy
\beq
\label{energy}
E = \half \iint |\nabla \psi|^2  \dd x \, \dd y \per
\eeq
and enstrophy
\beq
\label{energy}
Q = \half \iint q^2  \dd x \, \dd y \com
\eeq
where
\beq
q = \lap \psi + h\per
\eeq

\section{A minimum enstrophy principle}

\BH start by asking: ``what is the flow pattern which minimizes Q for a given E?''.
That is, one minimizes the
functional Q given the constrain E to obtain (see Appendix A)

\begin{align}
\label{minimize}
\delta Q + \mu \delta E &=  \iint \left[ q \lap \sigma  + \mu (\nabla \psi \cdot \nabla \sigma)\right]\, \dd x\, \dd y \\
\label{mini2}
                                 &=  \iint \lap \left( \lap \psi  + h - \mu \psi \right) \sigma \, \dd x\, \dd y = 0\com
\end{align}
where $\lambda^{-2}$ is a Lagrange multiplier and $\sigma$ is an arbitrary function. To obtain \eqref{mini2} from \eqref{minimize}
we use integration by parts and harmless boundary conditions (doubly periodic or no flux), e.g.,
\beq
\iint \nabla \psi \cdot \nabla \sigma \dd x \dd y = \underbrace{\iint \nabla \cdot \left( 
\sigma \nabla \psi \right) \dd x \, \dd y}_{=0} - \iint \sigma \nabla^2 \psi \dd x \, \dd y \per 
\eeq
Given the arbitrariness of $\sigma$ and assuming a doubly periodic domain, \eqref{mini2} reduces to the 
elliptic problem
\beq
\label{ellip}
\left(\mu - \lap \right)\psi = h \per
\eeq
In Fourier space, \eqref{ellip} reduces to
\beq
\label{ellip_fourier}
\hat \psi = \frac{\hat{h}}{\mu +  k^2 + l^2}\per
\eeq

\clearpage

\appendix 
\section{Algebra leading to \eqref{minimize}}
Let
\beq
\psi \to \psi + \alpha \sigma \com
\eeq
so that
\beq
q \to q + \alpha \lap \sigma \com
\eeq
where $\alpha$ is a small parameter and $\sigma$ is an arbitrary solution for the streamfunction.
We now minimize the functional
\beq
S = Q + \mu E\com
\eeq
with respect to $\alpha$. That is, we require 
\beq
\frac{\dd S}{\dd \alpha} = 0\com \text{for all } \sigma\per
\eeq
We obtain
\beq
\iint \left[ q \lap \sigma  + \mu (\nabla \psi \cdot \nabla \sigma)\right]\, \dd x\, \dd y + \alpha 
\iint \left[ (\lap \sigma )^2 + |\nabla \sigma|^2\right]\dd x \, \dd y = 0 \per
\eeq
Thus, the leading order (in $\alpha$) constrain is
\beq
\delta S = \delta (Q + \mu E) =  \iint \left[ q \lap \sigma  + \mu (\nabla \psi \cdot \nabla \sigma)\right]\, \dd x\, \dd y = 0\per
\eeq


%\begin{figure}[ht]
%\begin{center}
%\includegraphics[width=15pc,angle=0]{../src/figs/95perc.png} 
%    \includegraphics[width=15pc,angle=0]{../src/figs/90perc.png} \\
%    \includegraphics[width=15pc,angle=0]{../src/figs/87perc.png}     
%    \includegraphics[width=15pc,angle=0]{../src/figs/85perc.png} \\
%    \includegraphics[width=15pc,angle=0]{../src/figs/83perc.png}     
%    \includegraphics[width=15pc,angle=0]{../src/figs/82perc.png} 
%\end{center}
%\caption{The evolution of the vorticity field. After about $t=65$, the field consists of only two, opposite sign vortices which advect one another and decay very slowly. The colorbar limits are the same for all plots.}
%\label{fig:vort}
%\end{figure}
%
%\newpage
%\bibliographystyle{ametsoc2014}
%\bibliography{mae290c.bib}

\end{document}
