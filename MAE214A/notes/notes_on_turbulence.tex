
\documentclass[11pt]{article}

%% WRY has commented out some unused packages %%
%% If needed, activate these by uncommenting
\usepackage{geometry}                % See geometry.pdf to learn the layout options. There are lots.
%\geometry{letterpaper}                   % ... or a4paper or a5paper or ... 
\geometry{a4paper,left=2.5cm,right=2.5cm,top=2.5cm,bottom=2.5cm}
%\geometry{landscape}                % Activate for rotated page geometry
%\usepackage[parfill]{parskip}    % Activate to begin paragraphs with an empty line rather than an indent

%for figures
%\usepackage{graphicx}

\usepackage{color}
\definecolor{mygreen}{RGB}{28,172,0} % color values Red, Green, Blue
\definecolor{mylilas}{RGB}{170,55,241}
%% for graphics this one is also OK:
\usepackage{epsfig}

%% AMS mathsymbols are enabled with
\usepackage{amssymb,amsmath}

%% more options in enumerate
\usepackage{enumerate}
\usepackage{enumitem}

%% insert code
\usepackage{listings}

\usepackage[utf8]{inputenc}

\usepackage{hyperref}

%% colors
\usepackage{graphicx,xcolor,lipsum}


\usepackage{mathtools}

\usepackage{graphicx}
\newcommand*{\matminus}{%
  \leavevmode
  \hphantom{0}%
  \llap{%
    \settowidth{\dimen0 }{$0$}%
    \resizebox{1.1\dimen0 }{\height}{$-$}%
  }%
}

\title{Notes on turbulence and mixing}
\author{Cesar B Rocha}
\date{\today}

\begin{document}

\newcommand{\com}{\, ,}
\newcommand{\per}{\, .}

%% Averages
% Use \bar to over line solo symbols

\newcommand{\av}[1]{\bar{#1}}
\newcommand{\avbg}[1]{\overline{#1}}
\newcommand{\avbgg}[1]{\overline{#1}}

% A nice definition
\newcommand{\defn}{\ensuremath{\stackrel{\mathrm{def}}{=}}}

% equations
\def\beq{\begin{equation}}
\def\eeq{\end{equation}}

% calculus

\newcommand{\p}{\partial}
\newcommand{\ii}{{\rm i}}
\newcommand{\dd}{{\rm d}}
\newcommand{\id}{{\, \rm d}}
\newcommand{\ee}{{\rm e}}
\newcommand{\DD}{{\rm D}}
\newcommand{\wavy}{\text{wavy}}
\newcommand{\qg}{\text{qg}}


\newcommand{\be}{\beta}

\newcommand{\al}{\alpha}
\newcommand{\bx}{\boldsymbol{x}}
\newcommand{\by}{\boldsymbol{y}}
\newcommand{\bu}{\boldsymbol{u}}
\newcommand{\bv}{\boldsymbol{v}}


\newcommand{\half}{\tfrac{1}{2}}
\newcommand{\halfrho}{\tfrac{1}{2}}
\newcommand{\rz}{{}}
\newcommand{\bn}{\boldsymbol{\hat n}}
\newcommand{\br}{\boldsymbol{r}}
\newcommand{\bR}{\boldsymbol{R}}
\newcommand{\bA}{\ensuremath {\boldsymbol {A}}}
\newcommand{\bB}{\ensuremath {\boldsymbol {B}}}
\newcommand{\bU}{\ensuremath {\boldsymbol {U}}}
\newcommand{\bE}{\ensuremath {\boldsymbol {E}}}
\newcommand{\bJ}{\ensuremath {\boldsymbol {J}}}
\newcommand{\bXX}{\ensuremath {\boldsymbol {\mathcal{X}}}}
\newcommand{\bFF}{\ensuremath {\boldsymbol {F}}}
\newcommand{\bF}{\ensuremath {\boldsymbol {F}^{\sharp}}}
\newcommand{\bG}{\ensuremath {\boldsymbol G}}
\newcommand{\bSigma}{\ensuremath {\boldsymbol {\Sigma}}}
\newcommand{\bvarphi}{\ensuremath {\boldsymbol {\varphi}}}
\newcommand{\bxi}{\ensuremath {\boldsymbol {\xi}}}
\newcommand{\avbxi}{\overline{\ensuremath {\boldsymbol {\xi}}}}

% math cal

\newcommand{\J}{\mathcal{J}}
\newcommand{\K}{\mathcal{K}}
\newcommand{\cG}{\mathcal{G}}
\newcommand{\cF}{\mathcal{F}}
\newcommand{\cN}{\mathcal{N}}
\newcommand{\cL}{\mathcal{L}}


% san serif for matrices and differential operators
%\newcommand{\helmn}{\mathsf{H}_n}
\newcommand{\helmm}{\triangle_m}
\newcommand{\helmn}{\triangle_n}
\newcommand{\helms}{\triangle_s}
\newcommand{\helm}{\triangle}
\newcommand{\sA}{\mathsf{A}}
\newcommand{\sB}{\mathsf{B}}
\newcommand{\sG}{\mathsf{G}}
\newcommand{\sI}{\mathsf{I}}
\newcommand{\sJ}{\mathsf{J}}
\newcommand{\sU}{\mathsf{U}}
\newcommand{\sP}{\mathsf{P}}
\newcommand{\sQ}{\mathsf{Q}}
\newcommand{\sR}{\mathsf{R}}
\newcommand{\sL}{\mathsf{L}}
\renewcommand{\sJ}{\mathsf{J}}
\renewcommand{\sI}{\mathsf{I}}
\renewcommand{\L}{\mathsf{L}}
\newcommand{\sM}{\mathsf{M}}
\newcommand{\sT}{\mathsf{T}}
\newcommand{\sGamma}{\mathsf{\Gamma}}
\newcommand{\sOmega}{\mathsf{\Omega}}
\newcommand{\sSigma}{\mathsf{\Omega}}
\newcommand{\sbeta}{\mathsf{\beta}}
\newcommand{\sPi}{\mathsf{\Pi}}
\newcommand{\sC}{\mathsf{C}}
\newcommand{\sQy}{\mathsf{Q}}


% angle brackets

\def\la{\langle}
\def\ra{\rangle}
\def\laa{\left \langle}
\def\raa{\right \rangle}


%grads and div's
\newcommand{\bcdot}{\hspace{-0.1em} \boldsymbol{\cdot} \hspace{-0.12em}}
\newcommand{\bnabla}{\boldsymbol{\nabla}}
\newcommand{\bnablaH}{\bnabla_{\! \mathrm{h}}}
\newcommand{\grad}{\bnabla}
\newcommand{\gradH}{\bnablaH}
\newcommand{\curl}{\bnabla \!\times\!}
\newcommand{\diver}{\bnabla \bcdot }
\newcommand{\cross}{\times}
\newcommand{\lap}{\triangle}


%varthetas and thetas
\newcommand{\vth}{\vartheta}
\newcommand{\psii}{\psi^{\mathrm{i}}}
\newcommand{\thb}{\theta^{\mathrm{-}}}
\newcommand{\vthb}{\vartheta^{\mathrm{-}}}
\newcommand{\vthbhat}{{\hat{\vartheta}}^{\mathrm{-}}}
\newcommand{\vThb}{\varTheta^{\mathrm{-}}}
\newcommand{\psib}{\psi^{\mathrm{-}}}
\newcommand{\tht}{\theta^{\mathrm{+}}}
\newcommand{\vtht}{\vartheta^{\mathrm{+}}}
\newcommand{\vththat}{{\hat{\vartheta}}^{\mathrm{+}}}
\newcommand{\vthtbhat}{{\hat{\vartheta}}^{\pm}}
\newcommand{\vTht}{\varTheta^{\mathrm{+}}}
\newcommand{\vthtb}{\vartheta^{\pm}}
\newcommand{\vThtb}{\varTheta^{\pm}}

% nondimensional numbers
\renewcommand{\Re}{\mathrm{Re}}
\newcommand{\Ro}{\mathrm{Ro}}
\newcommand{\Ri}{\mathrm{Ri}}

%psi's
%Galerking coefficient for psi:
\newcommand{\gpsi}{\breve \psi}
\newcommand{\gphi}{\breve \phi}
\newcommand{\gq}{\breve q}
\newcommand{\gU}{\breve U}
\newcommand{\gQ}{\breve Q}
\newcommand{\gsigma}{\breve \sigma}


\newcommand{\psit}{\psi^{\mathrm{+}}}
\newcommand{\psithat}{{\hat{\psi}}^{\mathrm{+}}}
\newcommand{\psibhat}{{\hat{\psi}}^{\mathrm{-}}}
\newcommand{\psitb}{\psi^{\pm}}
\newcommand{\psitbhat}{{\hat{\psi}}^\pm}
\newcommand{\St}{S^{\mathrm{+}}}
\newcommand{\Sb}{S^{\mathrm{-}}}
\newcommand{\phb}{\phi^{\mathrm{-}}}
\newcommand{\pht}{\phi^{\mathrm{+}}}
\newcommand{\tautb}{\tau^{\pm}}
\newcommand{\sigmatb}{\sigma^{\pm}}


\newcommand{\bur}{\left(\tfrac{f_0}{N}\right)^2}
\newcommand{\ibur}{\left(\tfrac{N}{f_0}\right)^2}
\newcommand{\Nm}{N_{\mathrm{mix}}}
\newcommand{\xim}{\xi_{\mathrm{mix}}}
\newcommand{\hs}{h_*}
\renewcommand{\sp}{\mathsf{p}}
\newcommand{\se}{\mathsf{e}}
\newcommand{\sptb}{\mathsf{p}^\pm}


%nmax is a problem:
%\newcommand{\nmax}{n_{\mathrm{max}}}
\newcommand{\nmax}{\mathrm{N}}

\newcommand{\WKB}{\mathrm{WKB}}
\newcommand{\Lam}{\Lambda}
\newcommand{\tha}{\theta}
\newcommand{\kap}{\kappa}
\newcommand{\bphi}{\boldsymbol{\phi}}
\newcommand{\third}{\tfrac{1}{3}}
\newcommand{\cs}{c^\star}
\newcommand{\nt}{n^{\mathrm{trnc}}}
\newcommand{\sDp}{\mathsf{D}^1_{\nmax}}
\newcommand{\sDpp}{\mathsf{D}^2_{\nmax}}
\newcommand{\sD}{\mathsf{D_2}}
\newcommand{\sK}{\mathsf{K_2}}
\newcommand{\stheta}{\mathsf{\theta}}
\newcommand{\sphi}{\mathsf{\phi}}
\newcommand{\sq}{\mathsf{q}}
\newcommand{\cosech}{\text{csch}\,}
\newcommand{\sinc}{\text{sinc}\,}

%%%%%%%%% %%%%
\newcommand{\zp}{z^+}
\newcommand{\zm}{z^-}
\newcommand{\qA}{q^A_{\nmax}}
\newcommand{\psiB}{\psi^B_{\nmax}}
\newcommand{\phiB}{\phi^B_{\nmax}}
\newcommand{\eye}{\boldsymbol{\hat{i}}}
\newcommand{\jay}{\boldsymbol{\hat{j}}}
\newcommand{\kay}{\boldsymbol{\hat{k}}}
\newcommand{\psiG}{\psi^{\mathrm{G}}}
\newcommand{\qG}{q^{\mathrm{G}}}
\newcommand{\uG}{u^{\mathrm{G}}}
\newcommand{\UG}{U^{\mathrm{G}}}
\newcommand{\UGN}{U^{\mathrm{G}}_{\nmax}}
\newcommand{\QGN}{Q^{\mathrm{G}}_{\nmax}}
\newcommand{\sumoddn}{\sum_{n = 1, n~ \text{odd}}^{\nmax}}

% bretherton 
\newcommand{\qBr}{q_{\mathrm{Br}}}
\newcommand{\psiBr}{\psi_{\mathrm{Br}}}



\maketitle

This are notes I've been writing for my self study as part of the class ``Turbulence and Mixing'' (MAE 214A), taught by Prof. Sutanu Sarkar in Winter 2015. The class covers essentially homogeneous three-dimensional turbulence but I'm also self-studying two-dimensional and quasi-two-dimensional turbulence. 

I claim no originality to the content of these notes.  In particular, I'm loosely following Sutanu's class notes, and the following books and notes

\begin{itemize}

    \item \textit{Turbulent flows} by Pope;

    \item \textit{A first course in turbulence} by Tennekes and Lumley;

    \item \textit{Turbulence} by Frisch;

    \item \textit{The theory of homogeneous turbulence} by Batchelor;

    \item \textit{Notes on turbulence} by McWilliams.

\end{itemize}

\section{Introduction}

\subsection{What is turbulence?}

Turbulence is certainly a field in which the maxims ``To define it is to limit it'' is valid. Indeed, we must avoid the temptation of formally defining turbulence. Instead, we shall collect some characteristics common to turbulent flows:

\begin{enumerate}

    \item {\bf Large Reynolds number}
        \subitem We will define the Reynolds number ($\Re$), and contrast different interpretations. Here it suffices the loose scaling definition as the ration of the non-linear advective terms to viscous term. This readily suggests that turbulent flows are a consequence of non-linearities of the equations of motion. In the opposite limit, that of of low $\Re$, the equations of motions become asymptotically linear, and one can typically make progress analytically. In contrasts, we rarely can solve for turbulent flows analytically, and studies on turbulence almost always resort to experiments both in the laboratory or on computers. An important implication of non-linearity is a remarkable sensitivity to initial and boundary conditions.
    \item {\bf Irregularity}
        \subitem Turbulent flows are spatially irregular and evolve chaotically. This characteristic is frequently observed in natural phenomena. For example, measurements of velocity the turbulent planetary boundary layer at a fixed point display irregular temporal behavior. The same is true for chimney flows that often one can see; the regular flow eventually breaks into a chaotic motion.

    \item {\bf Three-dimensional vorticity}
        \subitem Turbulent flows are typically three-dimensionally rotational. The absence of vorticity can certainly be used to characterize non-turbulent flows. For example, quasi-linear surface gravity waves in the ocean can combine to give rise to patterns that are irregular. But these flows are definitely not turbulent (they lack vorticity).
    
    \item {\bf Wide range of  scales}
        \subitem It is really hard to find particular scales consistent with turbulent motions. Indeed, turbulent flows span a wide range of temporal and spatial scales: from the very large to tiny dissipative spatial scales; from very short to significantly long time scales.

    \item {\bf Transport and mixing}
        \subitem Turbulent flows effectively transport and mixes mass, momentum, and scalars. Transport and mixing in turbulent flows are typically orders of magnitude larger than purely molecular processes.


    \item {\bf Coherent structures}
        \subitem There is some order in the midst of disorder. For that reason, statistical theories are not always successful. 
\end{enumerate}


\subsection{Methods of study}

\begin{enumerate}

        \item Statistical theory of idealized turbulence (isotropic, homogeneous).

        \item Phenomenological theory of turbulence. This is a semi-empirical approach also known as one-point modeling.

        \item Characterization of coherent structures.

        \item Two-point statistical models (correlations and spectra).

        \item Larger eddy simulations. These are computational simulations in which only the largest scales of the turbulent flow are explicitly resolved.

        \item Direct numerical simulations (DNS). This is the gold standard of study of turbulence, in which computational simulations are used to resolved all scales of motions explicitly. Needless to say, DNSs can be very computationally expensive.

\end{enumerate}

\subsection{Application}
Applications of turbulence studies are every where. Here are some examples

\begin{enumerate}
    \item Plane design: Estimation of drag on a wing;
    \item Mixing rates in a chemical reactor;
    \item Parameterization of small-scale motions in climate models.
\end{enumerate}

\section{The Navier-Stokes Equations}
We will typically use index notation since its compact and shrinks the algebra. We will eventually rewrite some terms in vector notation to facilitate physical interpretation. Throughout we use summation convention: when two repeated indices appear together, summation over that index in implied unless otherwise stated. 

The momentum equation is
\beq
\label{eq:momentum}
\p_t u_i + u_j\p_{x_j} u_i = \rho^{-1} \p_{x_j} \sigma_{ij} - \rho^{-1}\p_{x_3}\Psi\com
\eeq
where the stress tensor is given by
\beq
\label{eq:stress}
\sigma_{ij} = -p \delta_{ij} + 2\mu \,s_{ij}\com
\eeq
with $p$ the thermodynamic pressure, and the stain rate
\beq
\label{eq:strain}
s_{ij} = \frac{1}{2}\left(\p_{x_j}u_i + \p_{x_i}u_j\right)\per
\eeq
Also in \eqref{eq:stress} $\mu$ is the dynamic viscosity, and the gravity force $\Psi = \rho g x_3$.  Thus, the more usual form of the NS equations is
\beq
\label{eq:ns_momentum}
\p_t u_i + u_j\p_{x_j} u_i = \rho^{-1} \p_{x_j} p + \nu \p_{x_j}\p_{x_j} u_i\com
\eeq
where $\nu \defn \mu/\rho$ is the kinematic viscosity. We will deal with incompressible fluids. Thus the mass conservation reduces to the solenoidal condition
\beq
\label{eq:mass}
\p_{x_j} u_j = 0\per
\eeq
With \eqref{eq:mass} we note that then nonlinear advective term can be rewritten as 
\beq
\label{eq:reynolds_decomp}
u_j \p_{x_j} u_i = \p_{x_j} (u_i u_j) \per
\eeq

It is convenient define the modified pressure
\beq
\label{eq:mod_pressure}
p = -p\delta_{ij} + \Psi\per
\eeq


\subsection*{Reynolds decomposition}
We introduce the decomposition
\beq
\label{eq:decomp}
u_i = U_i + u'_i\com
\eeq
where the mean velocity is
\beq
\label{eq:mean_u}
U_i = <u_i>  \defn \frac{1}{T}\int_0^T u_i \dd t\com
\eeq
with the arbitrary period $T$. Defining a good average is an art. Sometimes it is better to think in terms of ensemble mean. Notice that by definition $<u'_i> = 0$

We introduce similar decompositions for the pressure. The idea is to introduce the decomposition \eqref{eq:decomp} into  the NS equations, and average over time. The linear terms are easy. For instance, mass conservations gives
\beq
\label{eq:mass_avg}
\p_{x_j} U_j = 0\com\qqand \p_{x_j} u_j = 0\per
\eeq
The average of the nonlinear term is
\begin{align}
\label{eq:avg_nl}
<(U_i + u'_i)(U_j + u_j')> &= <U_i\,U_j> + <U_i \,u'_j> + <U_j \,u'_i> + <u'_i\,u'_j> \nonumber \\
    & =U_i\,U_j +  <u'_i\,u'_j>\per
\end{align}
Thus, the average momentum equation is
\beq
\label{eq:momentum_avg}
\p_t U_i + U_j \p_{x_j} U_i =  \rho^{-1} \p_{x_j} P + \nu \p_{x_j}\p_{x_j} U_i - \p_{x_j}<u'_i\,u'_j>\per
\eeq
The last term in  represents the mean eddy transport of fluctuating momentum. It is generally nonzero because the velocity fluctuations are typically correlated. The velocity correlation is termed the Reynolds stress
\beq
\label{eq:rey_stress}
\sigma'_{ij} \defn - \rho <u'_i\,u'_j>\per
\eeq
This decomposition reveals a problem that is in the heart of turbulence. Because of the nonlinearity of the advective term in \eqref{eq:ns_momentum} one cannot write a closed equation for average fields. Indeed, this is termed the closure problem.  If one tries to write equations for $<u'_i\,u'_j>$, one finds that it depends on higher order terms, say $<u'_i\,u'_j\,u'_k>$.

We can form an equation for the turbulent velocity $\tu_i$ by subtracting the equation for the total flow  \eqref{eq:ns_momentum} from the average momentum equation \eqref{eq:momentum_avg}
\beq
\label{eq:ns_turb}
\p_t \tu_i + u_j \p_{x_j} \tu_i = -\rho^{-1}\p_{x_i}p' + \nu \p_{x_j}\p_{x_j}\tu_i -\tu_j \p_{x_j} U_i
+ \p_{x_j}<\tu_i\tu_j>\per
\eeq

\subsection*{Eddy viscosity}
To make progress, we will have to close the system. The most used method is to write the Reynolds stresses as a function of $U_i$. This is inspired by the apparent similarity between the viscous and Reynolds stresses. It is important to realize that this is a crude representation, and that $\tau'_{ij}$ depend fundamentally on the properties of the flows, whereas viscous stresses do not. The parameterize expression is 
\beq
\label{eq:eddy_viscosity}
\tau'_{ij} = K \p_{x_i} U_j\com
\eeq
and is called eddy viscosity. The eddy viscous coefficient $K$ is typically orders of magnitude larger than the molecular viscosity $\nu$.

\subsection*{Passive scalar}
We can also write an equation for the passive scale $\phi$
\beq
\label{eq:pass_scalar}
\p_t \phi + u_j \p_{x_j} \phi = \gamma \p_{x_jx_j}^2\phi\com
\eeq
where $\gamma$ is the scaler diffusivity. Similarly to the procedure used above in the equations of motions, we can introduce a Reynolds-like decomposition, to obtain the mean passive scalar equation
\beq
\label{eq:pass_scalar_mean}
\p_t \Phi + U_j \p_{x_j} \Phi = \gamma \p_{x_jx_j}^2\Phi - \p_{x_j}{<u'_j\,\phi'>}\per
\eeq
Again, assumptions have to be made to close the system.

\subsection*{The role of pressure}
Taking the divergence of the momentum equation \eqref{eq:ns_momentum} we obtain a Poisson problem for pressure
\beq
\label{eq:pressure}
\p_{x_j}\p_{x_j}p = - \rho \p_{x_j} u_i \p_{x_i} u_j\per
\eeq
Hence the pressure can be diagnosed from the flow $u_i$, provided boundary conditions are prescribed.

\subsection*{The vorticity equation}
The vorticity is defined as the curl of the velocity
\beq
\label{eq:vort_defn}
\omega_i = \eijk \p_{x_j} u_k\per
\eeq
We can form an equation for $\omega_i$ by taking the curl of the momentum equation \eqref{eq:ns_momentum}. The final result is
\beq
\label{eq:vort_eqn}
\p_t \omega_i + u_j \p_{x_j}\omega_i = \omega_j\p_{x_j}u_i + \nu \p_{x_j x_j} \omega_i\per
\eeq
We can form an enstrophy equation by dotting the equation above with $\omega_i$, to obtain
\beq
\label{eq:enstrophy_eqn}
\p_t \omega_i^2 + u_j \p_{x_j}\omega_i ^2 = 2\omega_i\omega_j\p_{x_j}u_i + \nu \p_{x_j x_j} \omega_i^2 - 2 \nu \left(\p_{x_j} \omega_i\right)^2\per
\eeq
The last term is negative definite, and it represents the viscous dissipation of enstrophy. The second last term is the divergence of a viscous flux, which redistribute enstrophy across the domain. Because the tensor $\omega_i \omega_j$ is symmetric, we can rewrite the first term as 
\beq
\label{eq:enstrophy_1st_term}
\omega_i\omega_j \p_{x_j}u_i = \omega_i\omega_j s_{ij}\com
\eeq
where $s_{ij}$ is the rate of strain tensor defined in \eqref{eq:strain}. Thus this term represents the generation of enstrophy through  straining by the flow.

\section{Kinetic Energy}
Kinetic energy is a diagnostic of fluid flows. Thus, a descriptions of its transfers and dissipation plays a fundamental role in the study of turbulence. The kinetic energy per unit mass is defined as
\beq
\label{eq:tot_ke}
E(x_i,t) \defn \frac{u_i u_i}{2}\com
\eeq
and its average can be decomposed in two parts
\beq
\label{eq:ke_decomp}
<E(x_i,t)>= \bar{E}(x_i,t) + e(x_i,t)\com
\eeq
where the kinetic energy of the average flow is
\beq
\label{eq:ke_avg}
\bar{E}(x_i,t) \defn \frac{<\au_i\au_i>}{2}\com
\eeq
and the turbulence kinetic energy is
\beq
\label{eq:ke_turb}
e(x_i,t) \defn \frac{<\tu_i\tu_i>}{2}\per
\eeq
The decomposition \eqref{eq:ke_decomp}  is easily verifiable: just recall the Reynolds decomposition $u_i \defn \au_i + \tu_i$. Also, the turbulent kinetic energy $e(x_i,t)$ is determined by the isotropic part of the Reynolds stress tensor. The anisotropic  part of that tensor also scales with $e(x_i,t)$; experiments show that $|<u,v>|\sim 0.27 e$. Mathematically, the turbulent kinetic energy is a bound on the anisotropic part of the Reynolds stress tensor is
\beq
\label{eq:bond_reynolds_stress}
|<uv>| \le e\per
\eeq


\subsection*{The instantaneous kinetic energy equation}
We begin by noticing that for incompressible flows
\beq
u_i \frac{D u_i}{D t}  =\frac{D E }{D t}  = \p_t E + \p_{x_j}\left(u_j E\right)\per
\eeq
Hence, multiplying the momentum equation \eqref{eq:ns_momentum} by $u_i$, and using the definition of the rate of strain tensor \eqref{eq:strain},  we obtain the instantaneous kinetic energy equation
\beq
\label{eq:inst_ke_eqn}
\p_t E + \p_{x_j}\left(u_j E\right) + \p_{x_j}T_{j} = -2\nu s_{ij}s_{ij}\com
\eeq
where the kinetic energy flux is
\beq
\label{eq:ke_flux}
T_j \defn u_j p /\rho - 2\nu u_k s_{ik}\per
\eeq
Integrating the energy equation \eqref{eq:inst_ke_eqn} over a fixed control volume $V$ with surface areas $S$ we obtain the instantaneous energy budget 
\beq
\label{eq:inst_ke_budget}
\frac{\dd}{\dd t}\int\! E \,\dd V + \int \left(u_j E + T_j\right)_k\hat{n}_k\dd S = -2 \int \!\nu s_{ij}s_{ij}\, \dd V\com 
\eeq
where $\hat{n}_k$ is a unit vector normal to $S$. The surface term in \eqref{eq:inst_ke_budget} accounts for the surface energy flux into the volume $V$ and the work done by the pressure and viscous stress on the surface $S$. The term on the right of  \eqref{eq:inst_ke_budget} is negative definite and it represents the kinetic energy viscous dissipation, i.e., the conversion of kinetic energy into internal energy.

Notice that
\beq
\label{eq:ke_adv_identity}
\p_{x_j}\p_{x_j}\half u_i u_i = u_i \p_{x_j}\p_{x_j} u_i + \p_{x_j}u_i \p_{x_j}u_i\per
\eeq
Hence an alternate form for the instantaneous kinetic energy equation is
\beq
\label{eq:inst_ke_eqn_2}
\p_t E + \p_{x_j}\left(u_j E\right) + \p_{x_j}\tilde{T}_{j} = -\nu  \p_{x_j}u_i \p_{x_j}u_i\com
\eeq
where the energy flux is
\beq
\label{eq:ke_flux_2}
\tilde{T}_j = u_i p/\rho -\nu \p_{x_j}E\per
\eeq

\subsection*{The average kinetic energy equation}
Taking the average of the instantaneous kinetic energy equation \eqref{eq:inst_ke_eqn} we obtain
\beq
\label{eq:avg_ke}
\p_t <E> + \p_{x_j}\left(U_j <E>\right) + \p_{x_j}\left(<u_j E> + <T_{j}>\right) = -\vep - \bar{\vep}\com
\eeq
where  $<u_j E>$ is the turbulence transport of kinetic energy, and the terms on the right of \eqref{eq:avg_ke} are the viscous disspation of mean kinetic energy
\beq
\label{eq:mean_ep}
\bar{\vep}\defn 2 \nu S_{ij} S_{ij}\com
\eeq
and the viscous dissipation of turbulent kinetic energy
\beq
\label{eq:turb_ep}
\vep\defn 2 \nu s'_{ij} s'_{ij}\com
\eeq
with mean and turbulent rate of strain tensors
\beq
\label{eq:mean_strain}
S_{ij} = <s_{ij}> \defn \frac{1}{2}\left(\p_{x_j}\au_i + \p_{x_i}\au_j\right)\com
\eeq
and
\beq
\label{eq:turb_strain}
s_{ij} = s_{ij} - S_{ij} \defn \frac{1}{2}\left(\p_{x_j}u_i + \p_{x_i}u_j\right)\per
\eeq
Pope note that the dissipation associated with the mean flow scales  as $\Re^{-1}$, and is typically negligible. 

\subsection*{The equation for the kinetic energy of the mean flow}
We can form an equation for the kinetic energy of the mean flow by multiplying \eqref{eq:momentum_avg} by $U_i$
\beq
\label{eq:ke_eqn_mean_flow}
\p_t \bar{E} + \p_{x_j} \au_j \bar{E}  + \p_{x_j}<T_j> = -\cP -\bar\vep\com
\eeq
where the average kinetic energy transport is
\beq
\label{eq:average_transp}
<T_j> = \au_j <\tu_i\tu_j> + \au_j P/\rho - 2\nu \au_j S_{ij}\per
\eeq
The production term $\cP$ is
\beq
\label{eq:prod_term}
\cP \defn -<\tu_i\tu_j> \p_{x_j}U_i\per
\eeq

\subsection*{The equation for the kinetic energy of the turbulent flow}
We form an equation for the kinetic energy of the turbulent flow by multiplying \eqref{eq:ns_turb} by $\tu_i$ and averaging
\beq
\label{eq:ke_eqn_mean_flow}
\p_t e + \pxj U_j e  + \pxj T'_j = \cP - \vep\com
\eeq
where the flux of turbulent kinetic energy is
\beq
\label{eq:turb_ke_flux}
T'_i \defn <\tu_i p'>/\rho - 2\nu <\tu_i s'_{ij}> +\half <\tu_i\tu_i\tu_j>\per
\eeq
Notice that the production $\cP$ term appears in both mean and turbulent kinetic energy equations with signs changed. Thus this term represents a conversion between mean and turbulent flows. This term is associated with the interaction of the Reynolds stress with the background velocity shear. Because this term is typically positive, the kinetic energy is transferred from mean to turbulent flow. Pope note some important properties of the production term

\begin{enumerate}
    \item  Because $<\tu_i\tu_j>$ is symmetric, only the symmetric part of the velocity gradient tensor
        affects the production term
        \beq
        \cP = -<\tu_i\tu_j>S_{ij};
        \eeq
    \item Only the anisotropic part of the Reynolds stress tensor affects the production term
        \beq
        \cP = -a_{ij}S_{ij}\com
        \eeq
        $a_{ij} = <\tu_i\tu_j>-\frac{2 e}{3}\delta_{ij}$;
    \item Assuming the eddy viscosity hypothesis $a_{ij} = -2 \nu_T S_{ij}$ the production term has the form
        \beq
        \cP  = 2\nu_T S_{ij} S_{ij} \ge 0\per
        \eeq
        Hence $\cP$ assumes the form of the dissipation of mean kinetic energy $\bar{\vep}$ with the eddy
        viscosity $\nu_T$ in place of the molecular viscosity $\nu$.

\end{enumerate}

\end{document}




