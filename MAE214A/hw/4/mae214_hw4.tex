\documentclass[11pt]{article}

%% WRY has commented out some unused packages %%
%% If needed, activate these by uncommenting
\usepackage{geometry}                % See geometry.pdf to learn the layout options. There are lots.
%\geometry{letterpaper}                   % ... or a4paper or a5paper or ... 
\geometry{a4paper,left=2.5cm,right=2.5cm,top=2.5cm,bottom=2.5cm}
%\geometry{landscape}                % Activate for rotated page geometry
%\usepackage[parfill]{parskip}    % Activate to begin paragraphs with an empty line rather than an indent

%for figures
%\usepackage{graphicx}

\usepackage{color}
\definecolor{mygreen}{RGB}{28,172,0} % color values Red, Green, Blue
\definecolor{mylilas}{RGB}{170,55,241}
%% for graphics this one is also OK:
\usepackage{epsfig}

%% AMS mathsymbols are enabled with
\usepackage{amssymb,amsmath}

%% more options in enumerate
\usepackage{enumerate}
\usepackage{enumitem}

%% insert code
\usepackage{listings}

\usepackage[utf8]{inputenc}

\usepackage{hyperref}


% Default fixed font does not support bold face
\DeclareFixedFont{\ttb}{T1}{txtt}{bx}{n}{12} % for bold
\DeclareFixedFont{\ttm}{T1}{txtt}{m}{n}{12}  % for normal

% Custom colors
\usepackage{color}
\definecolor{deepblue}{rgb}{0,0,0.5}
\definecolor{deepred}{rgb}{0.6,0,0}
\definecolor{deepgreen}{rgb}{0,0.5,0}


% Python style for highlighting
\newcommand\pythonstyle{\lstset{
language=Python,
basicstyle=\ttm,
otherkeywords={self},             % Add keywords here
keywordstyle=\ttb\color{deepblue},
emph={MyClass,__init__},          % Custom highlighting
emphstyle=\ttb\color{deepred},    % Custom highlighting style
stringstyle=\color{deepgreen},
frame=tb,                         % Any extra options here
showstringspaces=false            % 
}}

% Python environment
\lstnewenvironment{python}[1][]
{
\pythonstyle
\lstset{#1}
}
{}

% Python for external files
\newcommand\pythonexternal[2][]{{
\pythonstyle
\lstinputlisting[#1]{#2}}}

% Python for inline
\newcommand\pythoninline[1]{{\pythonstyle\lstinline!#1!}}

%\usepackage{epstopdf}
%\DeclareGraphicsRule{.tif}{png}{.png}{`convert #1 `dirname #1`/`basename #1 .tif`.png}



%% colors
\usepackage{graphicx,xcolor,lipsum}


\usepackage{mathtools}

\usepackage{graphicx}
\newcommand*{\matminus}{%
  \leavevmode
  \hphantom{0}%
  \llap{%
    \settowidth{\dimen0 }{$0$}%
    \resizebox{1.1\dimen0 }{\height}{$-$}%
  }%
}


\title{MAE214, Homework Assignment 4}
\author{Cesar B Rocha}
\date{\today}

\begin{document}

\newcommand{\com}{\, ,}
\newcommand{\per}{\, .}

%% Averages
% Use \bar to over line solo symbols

\newcommand{\av}[1]{\bar{#1}}
\newcommand{\avbg}[1]{\overline{#1}}
\newcommand{\avbgg}[1]{\overline{#1}}

% A nice definition
\newcommand{\defn}{\ensuremath{\stackrel{\mathrm{def}}{=}}}

% equations
\def\beq{\begin{equation}}
\def\eeq{\end{equation}}

% calculus

\newcommand{\p}{\partial}
\newcommand{\ii}{{\rm i}}
\newcommand{\dd}{{\rm d}}
\newcommand{\id}{{\, \rm d}}
\newcommand{\ee}{{\rm e}}
\newcommand{\DD}{{\rm D}}
\newcommand{\wavy}{\text{wavy}}
\newcommand{\qg}{\text{qg}}


\newcommand{\be}{\beta}

\newcommand{\al}{\alpha}
\newcommand{\bx}{\boldsymbol{x}}
\newcommand{\by}{\boldsymbol{y}}
\newcommand{\bu}{\boldsymbol{u}}
\newcommand{\bv}{\boldsymbol{v}}


\newcommand{\half}{\tfrac{1}{2}}
\newcommand{\halfrho}{\tfrac{1}{2}}
\newcommand{\rz}{{}}
\newcommand{\bn}{\boldsymbol{\hat n}}
\newcommand{\br}{\boldsymbol{r}}
\newcommand{\bR}{\boldsymbol{R}}
\newcommand{\bA}{\ensuremath {\boldsymbol {A}}}
\newcommand{\bB}{\ensuremath {\boldsymbol {B}}}
\newcommand{\bU}{\ensuremath {\boldsymbol {U}}}
\newcommand{\bE}{\ensuremath {\boldsymbol {E}}}
\newcommand{\bJ}{\ensuremath {\boldsymbol {J}}}
\newcommand{\bXX}{\ensuremath {\boldsymbol {\mathcal{X}}}}
\newcommand{\bFF}{\ensuremath {\boldsymbol {F}}}
\newcommand{\bF}{\ensuremath {\boldsymbol {F}^{\sharp}}}
\newcommand{\bG}{\ensuremath {\boldsymbol G}}
\newcommand{\bSigma}{\ensuremath {\boldsymbol {\Sigma}}}
\newcommand{\bvarphi}{\ensuremath {\boldsymbol {\varphi}}}
\newcommand{\bxi}{\ensuremath {\boldsymbol {\xi}}}
\newcommand{\avbxi}{\overline{\ensuremath {\boldsymbol {\xi}}}}

% math cal

\newcommand{\J}{\mathcal{J}}
\newcommand{\K}{\mathcal{K}}
\newcommand{\cG}{\mathcal{G}}
\newcommand{\cF}{\mathcal{F}}
\newcommand{\cN}{\mathcal{N}}
\newcommand{\cL}{\mathcal{L}}


% san serif for matrices and differential operators
%\newcommand{\helmn}{\mathsf{H}_n}
\newcommand{\helmm}{\triangle_m}
\newcommand{\helmn}{\triangle_n}
\newcommand{\helms}{\triangle_s}
\newcommand{\helm}{\triangle}
\newcommand{\sA}{\mathsf{A}}
\newcommand{\sB}{\mathsf{B}}
\newcommand{\sG}{\mathsf{G}}
\newcommand{\sI}{\mathsf{I}}
\newcommand{\sJ}{\mathsf{J}}
\newcommand{\sU}{\mathsf{U}}
\newcommand{\sP}{\mathsf{P}}
\newcommand{\sQ}{\mathsf{Q}}
\newcommand{\sR}{\mathsf{R}}
\newcommand{\sL}{\mathsf{L}}
\renewcommand{\sJ}{\mathsf{J}}
\renewcommand{\sI}{\mathsf{I}}
\renewcommand{\L}{\mathsf{L}}
\newcommand{\sM}{\mathsf{M}}
\newcommand{\sT}{\mathsf{T}}
\newcommand{\sGamma}{\mathsf{\Gamma}}
\newcommand{\sOmega}{\mathsf{\Omega}}
\newcommand{\sSigma}{\mathsf{\Omega}}
\newcommand{\sbeta}{\mathsf{\beta}}
\newcommand{\sPi}{\mathsf{\Pi}}
\newcommand{\sC}{\mathsf{C}}
\newcommand{\sQy}{\mathsf{Q}}


% angle brackets

\def\la{\langle}
\def\ra{\rangle}
\def\laa{\left \langle}
\def\raa{\right \rangle}


%grads and div's
\newcommand{\bcdot}{\hspace{-0.1em} \boldsymbol{\cdot} \hspace{-0.12em}}
\newcommand{\bnabla}{\boldsymbol{\nabla}}
\newcommand{\bnablaH}{\bnabla_{\! \mathrm{h}}}
\newcommand{\grad}{\bnabla}
\newcommand{\gradH}{\bnablaH}
\newcommand{\curl}{\bnabla \!\times\!}
\newcommand{\diver}{\bnabla \bcdot }
\newcommand{\cross}{\times}
\newcommand{\lap}{\triangle}


%varthetas and thetas
\newcommand{\vth}{\vartheta}
\newcommand{\psii}{\psi^{\mathrm{i}}}
\newcommand{\thb}{\theta^{\mathrm{-}}}
\newcommand{\vthb}{\vartheta^{\mathrm{-}}}
\newcommand{\vthbhat}{{\hat{\vartheta}}^{\mathrm{-}}}
\newcommand{\vThb}{\varTheta^{\mathrm{-}}}
\newcommand{\psib}{\psi^{\mathrm{-}}}
\newcommand{\tht}{\theta^{\mathrm{+}}}
\newcommand{\vtht}{\vartheta^{\mathrm{+}}}
\newcommand{\vththat}{{\hat{\vartheta}}^{\mathrm{+}}}
\newcommand{\vthtbhat}{{\hat{\vartheta}}^{\pm}}
\newcommand{\vTht}{\varTheta^{\mathrm{+}}}
\newcommand{\vthtb}{\vartheta^{\pm}}
\newcommand{\vThtb}{\varTheta^{\pm}}

% nondimensional numbers
\renewcommand{\Re}{\mathrm{Re}}
\newcommand{\Ro}{\mathrm{Ro}}
\newcommand{\Ri}{\mathrm{Ri}}

%psi's
%Galerking coefficient for psi:
\newcommand{\gpsi}{\breve \psi}
\newcommand{\gphi}{\breve \phi}
\newcommand{\gq}{\breve q}
\newcommand{\gU}{\breve U}
\newcommand{\gQ}{\breve Q}
\newcommand{\gsigma}{\breve \sigma}


\newcommand{\psit}{\psi^{\mathrm{+}}}
\newcommand{\psithat}{{\hat{\psi}}^{\mathrm{+}}}
\newcommand{\psibhat}{{\hat{\psi}}^{\mathrm{-}}}
\newcommand{\psitb}{\psi^{\pm}}
\newcommand{\psitbhat}{{\hat{\psi}}^\pm}
\newcommand{\St}{S^{\mathrm{+}}}
\newcommand{\Sb}{S^{\mathrm{-}}}
\newcommand{\phb}{\phi^{\mathrm{-}}}
\newcommand{\pht}{\phi^{\mathrm{+}}}
\newcommand{\tautb}{\tau^{\pm}}
\newcommand{\sigmatb}{\sigma^{\pm}}


\newcommand{\bur}{\left(\tfrac{f_0}{N}\right)^2}
\newcommand{\ibur}{\left(\tfrac{N}{f_0}\right)^2}
\newcommand{\Nm}{N_{\mathrm{mix}}}
\newcommand{\xim}{\xi_{\mathrm{mix}}}
\newcommand{\hs}{h_*}
\renewcommand{\sp}{\mathsf{p}}
\newcommand{\se}{\mathsf{e}}
\newcommand{\sptb}{\mathsf{p}^\pm}


%nmax is a problem:
%\newcommand{\nmax}{n_{\mathrm{max}}}
\newcommand{\nmax}{\mathrm{N}}

\newcommand{\WKB}{\mathrm{WKB}}
\newcommand{\Lam}{\Lambda}
\newcommand{\tha}{\theta}
\newcommand{\kap}{\kappa}
\newcommand{\bphi}{\boldsymbol{\phi}}
\newcommand{\third}{\tfrac{1}{3}}
\newcommand{\cs}{c^\star}
\newcommand{\nt}{n^{\mathrm{trnc}}}
\newcommand{\sDp}{\mathsf{D}^1_{\nmax}}
\newcommand{\sDpp}{\mathsf{D}^2_{\nmax}}
\newcommand{\sD}{\mathsf{D_2}}
\newcommand{\sK}{\mathsf{K_2}}
\newcommand{\stheta}{\mathsf{\theta}}
\newcommand{\sphi}{\mathsf{\phi}}
\newcommand{\sq}{\mathsf{q}}
\newcommand{\cosech}{\text{csch}\,}
\newcommand{\sinc}{\text{sinc}\,}

%%%%%%%%% %%%%
\newcommand{\zp}{z^+}
\newcommand{\zm}{z^-}
\newcommand{\qA}{q^A_{\nmax}}
\newcommand{\psiB}{\psi^B_{\nmax}}
\newcommand{\phiB}{\phi^B_{\nmax}}
\newcommand{\eye}{\boldsymbol{\hat{i}}}
\newcommand{\jay}{\boldsymbol{\hat{j}}}
\newcommand{\kay}{\boldsymbol{\hat{k}}}
\newcommand{\psiG}{\psi^{\mathrm{G}}}
\newcommand{\qG}{q^{\mathrm{G}}}
\newcommand{\uG}{u^{\mathrm{G}}}
\newcommand{\UG}{U^{\mathrm{G}}}
\newcommand{\UGN}{U^{\mathrm{G}}_{\nmax}}
\newcommand{\QGN}{Q^{\mathrm{G}}_{\nmax}}
\newcommand{\sumoddn}{\sum_{n = 1, n~ \text{odd}}^{\nmax}}

% bretherton 
\newcommand{\qBr}{q_{\mathrm{Br}}}
\newcommand{\psiBr}{\psi_{\mathrm{Br}}}



\maketitle

\section*{Problem 1}


\begin{enumerate}[label=(\alph*)]

    \item Figure \ref{fig:the_velocity}-top shows the different time-series at the grid-point $(65,2)$. 
        Of course, the instantaneous velocity at spanwise grid point 65 the red line in figure \ref{fig:the_velocity} presents by far the largest variance.
        The spanwise-averaged velocity presents a much smaller variance (the green line in figure \ref{fig:the_velocity} – it represents a relatively
        small variation about the time-and-spanwise mean velocity (the red line in figure \ref{fig:the_velocity}).

    \item We repeat the calculation of item (a) to grid points (65,8) and (65,20) – the results are presented in 
        middle and bottom panels of figure \ref{fig:the_velocity}. At each z level (distance from the wall), the
        relative magnitude and variance of the different curves are similar to the pattern described in item a, with
          \ref{fig:the_velocity}, with instantaneous velocities presenting the largest variance. But the magnitude of 
          the mean and variance vary significantly with the distance from the wall. Near the wall (top panel in figure \ref{fig:the_velocity}) the magnitudes of both mean and turbulent velocities are relatively small, significantly increasing at z-grid point 8. The largest mean flow is of course at the z-grid point 20. This is consistent with the shape of the
          mean flow which is, of course, largest in the middle of the channel, and vanished at the boundary. The magnitude of the turbulent velocities (the difference between the blue and red curves) appears to peak near grid point 8 (or perhaps between 8 and 20), consistent with the fact that the turbulence is largest where the mean velocity gradient is largest.

      \item Figure \ref{fig:the_mean_velocity} shows the time-and-spanwise mean velocity $<u^+>_{yt}$ as  a function
          of the distance form the wall in ``wall units'' ($z^+ \defn z/\delta_\nu$).


      \item Figure \ref{fig:the_mean_velocity2} shows the time-and-spanwise mean velocity $<u^+>$ as  a function
          of the distance form the wall in ``wall units'' from the mean profile given (Umean.dat). The dashed line indicates a log profile. Also identifies in  \ref{fig:the_mean_velocity2} are the viscous sublayer, the log-layer, and the
          law of the wake; the buffer layer spans the region between the viscous sublayer and the log-layer.



\end{enumerate}

\begin{figure}[ht]
\begin{center}
\includegraphics[width=27pc,angle=0]{pb1ab_2.png}\\
\includegraphics[width=27pc,angle=0]{pb1ab_8.png}\\
\includegraphics[width=27pc,angle=0]{pb1ab_20.png}
\end{center}
\caption{Time series of along-channel velocity: $u(t)$ instantaneous velocity at $y=65$, $<u>_y(t)$ is the 
instantaneous spanwise-averaged velocity, and $<u>_{yt}$ is the time-and-spanwise average velocity. The triplet
 of velocity timeseries are shown for three different $z$ grid points: 2 (top), 8 (middle), and 20 (bottom).}
\label{fig:the_velocity}
\end{figure}

\begin{figure}[ht]
\begin{center}
\includegraphics[width=27pc,angle=0]{pb1c.png}\\
\end{center}
\caption{The calculates time-and-spanwise mean velocity $<u+>_{yt}$ as a function of the distance from the wall.}
\label{fig:the_mean_velocity}
\end{figure}

\begin{figure}[ht]
\begin{center}
\includegraphics[width=27pc,angle=0]{pb1d.png}\\
\end{center}
\caption{The time-and-spanwise mean velocity $<u+>_{yt}$ as a function of the distance from the wall, from the given profile (Umean.dat).}
\label{fig:the_mean_velocity2}
\end{figure}


\section*{Problem 2}

\begin{enumerate}[label=(\alph*)]
    \item Figure \ref{fig:my_stress} compares the Reynolds stresses and production for the CFD Lab experiment compared against the results for Moser et al. Both experiments were performed at $\Re = 395$. Note that, for comparison, we changed the CFD lab experiment nomenclature coordinates from $z^+$ to $y^+$ (distance from the wall in wall units), and $v$ to $w$, and vise versa. The results compare favorably for all terms; the largest component is  $<uu>$, which peaks at about $y^+ \approx 12$. The production also peaks at about $y^+ \approx 12$.

  \item Figure \ref{fig:my_stress} compares the off-diagonal Reynolds stresses for the CFD Lab experiment compared against the results for Hoyas and Jimenez. The former was performed at $\Re = 395$ and the latter at much higher Reynolds number $\Re = 2000$. The magnitude of $<uv>$ is comparable in the both case, and $<uw>$ and $vw$ are at residual levels in both cases.  Of course, in the high Reynolds number experiment $<uv>$ extends farther, decaying almost linearly with $y^+$ from $y^+=200$ to $y^+=2000$.

\end{enumerate}

\begin{figure}[ht]
\begin{center}
\includegraphics[width=20pc,angle=0]{pb2a_res_comp.png}\\
\includegraphics[width=20pc,angle=0]{pb2a_prod_comp.png}
\end{center}
\caption{Reynolds stresses (upper panel) and production term (bottom panel) as a function of distance form the wall for the MAE UCSD CFD Lab channel flow experiment (solid) and the data from a similar experiment from Moser et al. (dashed) at Re$_\tau$ = 395.}
\label{fig:my_stress}
\end{figure}

\begin{figure}[ht]
\begin{center}
\includegraphics[width=20pc,angle=0]{pb2b_comp2.png}
\end{center}
\caption{Upper panel: Reynolds stresses and production term as a function of distance form the wall for the Hoyas and Jimenes experiment at Re$_\tau$ = 2000, and its comparison with the MAE CFD Lab at experiment at Re$_\tau$ = 395.}
\label{fig:stress2}
\end{figure}


\section*{Problem 3}



\begin{enumerate}[label=(\alph*)]

    \item The mean $x$-momentum equation in the boundary layer is
\beq
\label{eq:bl1}
<U>\frac{\p <U>}{\p x} + <V>\frac{\p <U>}{\p y} = \nu \frac{\p^2 <U>}{\p y^2} - \frac{\p <uv>}{\p y} -\frac{1}{\rho}\frac{\dd p_0}{\dd x}\com
\eeq
where the free stream pressure is
\beq
p_0(x) = <p> + <v^2>\com
\eeq
Noticing that the total shear stress is
\beq
\tau \defn \rho \nu \frac{\p <U>}{\p y} -\rho <uv>\com
\eeq
and using the mean $x$-momentum equation outside the boundary layer
\beq
U_0 \frac{\dd U_0}{\dd x} = -\frac{1}{\rho}\frac{\dd p_0}{\dd x}\com
\eeq
we can rewrite \eqref{eq:bl1}
\beq
<U>\frac{\p <U>}{\p x} + <V>\frac{\p <U>}{\p y} = \frac{1}{\rho}\frac{\p \tau}{\p y} + U_0 \frac{\dd U_0}{\dd x}\per
\eeq
Further using the continuity equation
\beq
\frac{\p <U>}{\p x} + \frac{\p <V>}{\p y} = 0\com
\eeq
we obtain 
\beq
\frac{\p <U><U>}{\p x} + \frac{\p <U><V>}{\p y} = \frac{1}{\rho}\frac{\p \tau}{\p y} + U_0 \frac{\dd U_0}{\dd x}\per
\eeq
Now multiplying the continuity equation by $U_0$ and combining with the momentum equation above we obtain
\beq
\frac{\p}{\p x}\left[<U>(U_0-<U>)\right] + \frac{\p}{\p y}
\left[<V>(U_0-<U>)\right] + (U_0-<U>)\frac{\dd U_0}{\dd x} = -\frac{1}{\rho}\frac{\p \tau}{\p y} \per
\eeq
Integrating in $y$ from $0$ to $\infty$ (outside the boundary layer), we obtain
\beq
\label{eq:bl_int}
-\frac{1}{\rho} \left(\underbrace{\tau^{\infty}}_{=0}-\tau^w\right) = \underbrace{\left[<V>(U_0-<U>)\right]^{\infty}_0}_{= 0} + \frac{\dd}{\dd x}\left(U_0^2 \theta\right) + \delta^*U_0 \frac{\dd U_0}{\dd x}\com
\eeq
where the momentum thickness is 
\beq
\theta(x) \defn \int_0^\infty \frac{<U>}{U_0}\left(1-\frac{<U>}{U_0}\right) \dd y\com
\eeq
and the displacement thickness
\beq
\delta^*(x) = \int_0^*\left(1 - \frac{<U>}{U_0}\right)\per
\eeq
Also, in \eqref{eq:bl_int} the total shear vanishes outside the boundary layer. The first term on the right of \eqref{eq:bl_int} also vanishes since at $y = 0$, $<U> = <V> = 0$, and, as $y\to\infty$, $<U>\to U_0$. Thus
\beq
\frac{\tau^w }{\rho} = \underbrace{\frac{\p}{\p x}\left(U_0^2 \theta \right)}_{U_0\frac{\dd \theta}{\dd x} + 2\theta U_0 \frac{\dd U_0}{\dd x}} + \delta^*U_0 \frac{\dd U_0}{\dd x}\per
\eeq
Thus, we obtain the von Karman integral momentum equation for a flat plate boundary layer
\beq
2 \frac{\dd \theta}{\dd x} + \frac{(4\theta + 2\delta^*)}{U_0}\frac{\dd U_0}{\dd x} = \frac{\tau^w}{\frac{1}{2}\rho U_0^2}\per
\eeq

\item The main differences between the turbulent channel flow and the boundary layer on a flat plate is that in the latter the boundary layer thickness grows with $x$ and, contrary to the channel flow where the mean where the given mean pressure gradients allows for the determination of the stress at the boundary, that stress at the wall is unknown a priori in the boundary layer problem. Pope also points out that, contrary to the channel flow problem, in the boundary layer problem, the flow outside the boundary layer consistent of intermittent turbulence. The main similarity is the behavior in the inner layer ($y/\delta(x)<1$).

\end{enumerate}

\end{document}


