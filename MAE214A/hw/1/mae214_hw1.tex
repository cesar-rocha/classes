\documentclass[11pt]{article}


%% WRY has commented out some unused packages %%
%% If needed, activate these by uncommenting
\usepackage{geometry}                % See geometry.pdf to learn the layout options. There are lots.
%\geometry{letterpaper}                   % ... or a4paper or a5paper or ... 
\geometry{a4paper,left=2.5cm,right=2.5cm,top=2.5cm,bottom=2.5cm}
%\geometry{landscape}                % Activate for rotated page geometry
%\usepackage[parfill]{parskip}    % Activate to begin paragraphs with an empty line rather than an indent

%for figures
%\usepackage{graphicx}

\usepackage{color}
\definecolor{mygreen}{RGB}{28,172,0} % color values Red, Green, Blue
\definecolor{mylilas}{RGB}{170,55,241}
%% for graphics this one is also OK:
\usepackage{epsfig}

%% AMS mathsymbols are enabled with
\usepackage{amssymb,amsmath}

%% more options in enumerate
\usepackage{enumerate}
\usepackage{enumitem}

%% insert code
\usepackage{listings}

\usepackage[utf8]{inputenc}

\usepackage{hyperref}


% Default fixed font does not support bold face
\DeclareFixedFont{\ttb}{T1}{txtt}{bx}{n}{12} % for bold
\DeclareFixedFont{\ttm}{T1}{txtt}{m}{n}{12}  % for normal

% Custom colors
\usepackage{color}
\definecolor{deepblue}{rgb}{0,0,0.5}
\definecolor{deepred}{rgb}{0.6,0,0}
\definecolor{deepgreen}{rgb}{0,0.5,0}


% Python style for highlighting
\newcommand\pythonstyle{\lstset{
language=Python,
basicstyle=\ttm,
otherkeywords={self},             % Add keywords here
keywordstyle=\ttb\color{deepblue},
emph={MyClass,__init__},          % Custom highlighting
emphstyle=\ttb\color{deepred},    % Custom highlighting style
stringstyle=\color{deepgreen},
frame=tb,                         % Any extra options here
showstringspaces=false            % 
}}

% Python environment
\lstnewenvironment{python}[1][]
{
\pythonstyle
\lstset{#1}
}
{}

% Python for external files
\newcommand\pythonexternal[2][]{{
\pythonstyle
\lstinputlisting[#1]{#2}}}

% Python for inline
\newcommand\pythoninline[1]{{\pythonstyle\lstinline!#1!}}

%\usepackage{epstopdf}
%\DeclareGraphicsRule{.tif}{png}{.png}{`convert #1 `dirname #1`/`basename #1 .tif`.png}



%% colors
\usepackage{graphicx,xcolor,lipsum}


\usepackage{mathtools}

\usepackage{graphicx}
\newcommand*{\matminus}{%
  \leavevmode
  \hphantom{0}%
  \llap{%
    \settowidth{\dimen0 }{$0$}%
    \resizebox{1.1\dimen0 }{\height}{$-$}%
  }%
}


\title{MAE214, Homework Assignment 1}
\author{Cesar B Rocha}
\date{\today}

\begin{document}

\include{mysymbols}

\maketitle

\section*{Problem 1}

\begin{enumerate}[label=(\alph*)]

    \item With constant density $\rho$ and uniform viscosity $\nu$, the vorticity equation is
        \beq
            \label{eq:vort}
            \frac{\p \omega_i}{\p t}   + u_j \frac{\p \omega_i}{\p x_j} = \underbrace{\omega_j\frac{\p u_i}{\p x_j}}_{(I)} \,+\,\underbrace{\nu \frac{\p^2 \omega_i}{\p x_j^2}}_{(II)}\com
        \eeq
        where the vorticity is defined as the curl of the velocity
        \beq
            \label{eq:curl_v}
            \omega_i \defn \eijk \frac{\p u_k}{\p x_j}\com
        \eeq
        where $\eijk$ is the alternating tensor. Also in \eqref{eq:vort} (I) represents the vorticity generation due to vortex stretching, and (II) is the viscous diffusion of vorticity. In two-dimensional flows, the stretching term is identically zero (I).  To see this, we note that in two dimensions the vorticity \eqref{eq:curl_v} is nonzero only in the direction normal to the plane of the flow. Suppose the plane is defined in the $x_1$-$x_2$ direction, then the only nonzero term in the vorticity is $\omega_3$. The stretching reduces to $\omega_3 \p_{x_3}u_i$, where $i=1,2$. But is independent of $x_3$ because the flow is two-dimensional, and therefore the stretching term vanishes identically. Thus in the absence of  viscosity the vorticity $\omega_3$ is materially conserved in two-dimensions. 

    To form an enstrophy equation we dot the vorticity equation \eqref{eq:vort} with $\omega_i$ to obtain
    \beq
        \label{ens1}
        \frac{\p}{\p t} \frac{\omega_i^2}{2} + u_j \frac{\p}{\p x_j}\frac{\omega_i^2}{2} = \omega_i \omega_j \frac{\p u_i}{\p x_j} + \nu \omega_i \frac{\p^2 \omega_i}{\p x_j^2} \com
    \eeq
    or after rearranging and working a bit on the last term
    \beq
        \label{ens2}
        \frac{\p \omega_i^2}{\p t} + u_j \frac{\p\omega_i^2}{\p x_j} = \underbrace{\omega_i \omega_j \frac{\p^2 u_i}{\p x_j}}_{(I)} + \underbrace{\nu \frac{\p \omega_i^2}{\p x_j^2}}_{(II)} \underbrace{- 2 \nu \left(\frac{\p \omega_i}{\p x_j}\right)^2}_{(III)}\per
    \eeq
    The first term on the right of \eqref{ens2} is the enstrophy generation due to the straining field. To see that we note that $\omega_i \omega_j$ is symmetric. Thus
    \beq
        \label{strain}
        \omega_i \omega_j \frac{\p u_i}{\p x_j} = \omega_i \omega_j s_{ij}\com
    \eeq
    where the rate of strain tensor is
    \beq
        s_{ij} \defn \frac{1}{2}\left(\frac{\p u_i}{\p x_j} + \frac{\p u_j}{\p x_i}\right)\per
    \eeq
    The second term (II) is the divergence of the viscous flux enstrophy. This term only redistributes enstrophy. To see this, we note that in a bounded region with no flux at the boundaries, this term integrates to zero. 

    Finally, the last term is negative definite (provided $\nu>0$; no negative viscosity here!), and it represents the viscous dissipation of enstrophy.

    Because there is no stretching term in two-dimensional flows, there is no enstrophy generation due to straining fields (I). In the absence of viscosity, the two-dimensional flow conserves total enstrophy defined as the area integral of the vorticity squared
    \beq
        \label{eq:tot_ens}
        Z \defn \int \int \omega_3^2 \dd S\com
    \eeq
    provided the is no flux of vorticity across the boundaries. This follows directly from the material conservation of vorticity in two dimensions with $\nu =0$.

\end{enumerate}



\end{document}


