\documentclass[11pt]{article}


%% WRY has commented out some unused packages %%
%% If needed, activate these by uncommenting
\usepackage{geometry}                % See geometry.pdf to learn the layout options. There are lots.
%\geometry{letterpaper}                   % ... or a4paper or a5paper or ... 
\geometry{a4paper,left=2.5cm,right=2.5cm,top=2.5cm,bottom=2.5cm}
%\geometry{landscape}                % Activate for rotated page geometry
%\usepackage[parfill]{parskip}    % Activate to begin paragraphs with an empty line rather than an indent

%for figures
%\usepackage{graphicx}

\usepackage{color}
\definecolor{mygreen}{RGB}{28,172,0} % color values Red, Green, Blue
\definecolor{mylilas}{RGB}{170,55,241}
%% for graphics this one is also OK:
\usepackage{epsfig}

%% AMS mathsymbols are enabled with
\usepackage{amssymb,amsmath}

%% more options in enumerate
\usepackage{enumerate}
\usepackage{enumitem}

%% insert code
\usepackage{listings}

\usepackage[utf8]{inputenc}

\usepackage{hyperref}


% Default fixed font does not support bold face
\DeclareFixedFont{\ttb}{T1}{txtt}{bx}{n}{12} % for bold
\DeclareFixedFont{\ttm}{T1}{txtt}{m}{n}{12}  % for normal

% Custom colors
\usepackage{color}
\definecolor{deepblue}{rgb}{0,0,0.5}
\definecolor{deepred}{rgb}{0.6,0,0}
\definecolor{deepgreen}{rgb}{0,0.5,0}


% Python style for highlighting
\newcommand\pythonstyle{\lstset{
language=Python,
basicstyle=\ttm,
otherkeywords={self},             % Add keywords here
keywordstyle=\ttb\color{deepblue},
emph={MyClass,__init__},          % Custom highlighting
emphstyle=\ttb\color{deepred},    % Custom highlighting style
stringstyle=\color{deepgreen},
frame=tb,                         % Any extra options here
showstringspaces=false            % 
}}

% Python environment
\lstnewenvironment{python}[1][]
{
\pythonstyle
\lstset{#1}
}
{}

% Python for external files
\newcommand\pythonexternal[2][]{{
\pythonstyle
\lstinputlisting[#1]{#2}}}

% Python for inline
\newcommand\pythoninline[1]{{\pythonstyle\lstinline!#1!}}

%\usepackage{epstopdf}
%\DeclareGraphicsRule{.tif}{png}{.png}{`convert #1 `dirname #1`/`basename #1 .tif`.png}



%% colors
\usepackage{graphicx,xcolor,lipsum}


\usepackage{mathtools}

\usepackage{graphicx}
\newcommand*{\matminus}{%
  \leavevmode
  \hphantom{0}%
  \llap{%
    \settowidth{\dimen0 }{$0$}%
    \resizebox{1.1\dimen0 }{\height}{$-$}%
  }%
}


\title{MAE214, Homework Assignment 1}
\author{Cesar B Rocha}
\date{\today}

\begin{document}

\newcommand{\com}{\, ,}
\newcommand{\per}{\, .}

%% Averages
% Use \bar to over line solo symbols

\newcommand{\av}[1]{\bar{#1}}
\newcommand{\avbg}[1]{\overline{#1}}
\newcommand{\avbgg}[1]{\overline{#1}}

% A nice definition
\newcommand{\defn}{\ensuremath{\stackrel{\mathrm{def}}{=}}}

% equations
\def\beq{\begin{equation}}
\def\eeq{\end{equation}}

% calculus

\newcommand{\p}{\partial}
\newcommand{\ii}{{\rm i}}
\newcommand{\dd}{{\rm d}}
\newcommand{\id}{{\, \rm d}}
\newcommand{\ee}{{\rm e}}
\newcommand{\DD}{{\rm D}}
\newcommand{\wavy}{\text{wavy}}
\newcommand{\qg}{\text{qg}}


\newcommand{\be}{\beta}

\newcommand{\al}{\alpha}
\newcommand{\bx}{\boldsymbol{x}}
\newcommand{\by}{\boldsymbol{y}}
\newcommand{\bu}{\boldsymbol{u}}
\newcommand{\bv}{\boldsymbol{v}}


\newcommand{\half}{\tfrac{1}{2}}
\newcommand{\halfrho}{\tfrac{1}{2}}
\newcommand{\rz}{{}}
\newcommand{\bn}{\boldsymbol{\hat n}}
\newcommand{\br}{\boldsymbol{r}}
\newcommand{\bR}{\boldsymbol{R}}
\newcommand{\bA}{\ensuremath {\boldsymbol {A}}}
\newcommand{\bB}{\ensuremath {\boldsymbol {B}}}
\newcommand{\bU}{\ensuremath {\boldsymbol {U}}}
\newcommand{\bE}{\ensuremath {\boldsymbol {E}}}
\newcommand{\bJ}{\ensuremath {\boldsymbol {J}}}
\newcommand{\bXX}{\ensuremath {\boldsymbol {\mathcal{X}}}}
\newcommand{\bFF}{\ensuremath {\boldsymbol {F}}}
\newcommand{\bF}{\ensuremath {\boldsymbol {F}^{\sharp}}}
\newcommand{\bG}{\ensuremath {\boldsymbol G}}
\newcommand{\bSigma}{\ensuremath {\boldsymbol {\Sigma}}}
\newcommand{\bvarphi}{\ensuremath {\boldsymbol {\varphi}}}
\newcommand{\bxi}{\ensuremath {\boldsymbol {\xi}}}
\newcommand{\avbxi}{\overline{\ensuremath {\boldsymbol {\xi}}}}

% math cal

\newcommand{\J}{\mathcal{J}}
\newcommand{\K}{\mathcal{K}}
\newcommand{\cG}{\mathcal{G}}
\newcommand{\cF}{\mathcal{F}}
\newcommand{\cN}{\mathcal{N}}
\newcommand{\cL}{\mathcal{L}}


% san serif for matrices and differential operators
%\newcommand{\helmn}{\mathsf{H}_n}
\newcommand{\helmm}{\triangle_m}
\newcommand{\helmn}{\triangle_n}
\newcommand{\helms}{\triangle_s}
\newcommand{\helm}{\triangle}
\newcommand{\sA}{\mathsf{A}}
\newcommand{\sB}{\mathsf{B}}
\newcommand{\sG}{\mathsf{G}}
\newcommand{\sI}{\mathsf{I}}
\newcommand{\sJ}{\mathsf{J}}
\newcommand{\sU}{\mathsf{U}}
\newcommand{\sP}{\mathsf{P}}
\newcommand{\sQ}{\mathsf{Q}}
\newcommand{\sR}{\mathsf{R}}
\newcommand{\sL}{\mathsf{L}}
\renewcommand{\sJ}{\mathsf{J}}
\renewcommand{\sI}{\mathsf{I}}
\renewcommand{\L}{\mathsf{L}}
\newcommand{\sM}{\mathsf{M}}
\newcommand{\sT}{\mathsf{T}}
\newcommand{\sGamma}{\mathsf{\Gamma}}
\newcommand{\sOmega}{\mathsf{\Omega}}
\newcommand{\sSigma}{\mathsf{\Omega}}
\newcommand{\sbeta}{\mathsf{\beta}}
\newcommand{\sPi}{\mathsf{\Pi}}
\newcommand{\sC}{\mathsf{C}}
\newcommand{\sQy}{\mathsf{Q}}


% angle brackets

\def\la{\langle}
\def\ra{\rangle}
\def\laa{\left \langle}
\def\raa{\right \rangle}


%grads and div's
\newcommand{\bcdot}{\hspace{-0.1em} \boldsymbol{\cdot} \hspace{-0.12em}}
\newcommand{\bnabla}{\boldsymbol{\nabla}}
\newcommand{\bnablaH}{\bnabla_{\! \mathrm{h}}}
\newcommand{\grad}{\bnabla}
\newcommand{\gradH}{\bnablaH}
\newcommand{\curl}{\bnabla \!\times\!}
\newcommand{\diver}{\bnabla \bcdot }
\newcommand{\cross}{\times}
\newcommand{\lap}{\triangle}


%varthetas and thetas
\newcommand{\vth}{\vartheta}
\newcommand{\psii}{\psi^{\mathrm{i}}}
\newcommand{\thb}{\theta^{\mathrm{-}}}
\newcommand{\vthb}{\vartheta^{\mathrm{-}}}
\newcommand{\vthbhat}{{\hat{\vartheta}}^{\mathrm{-}}}
\newcommand{\vThb}{\varTheta^{\mathrm{-}}}
\newcommand{\psib}{\psi^{\mathrm{-}}}
\newcommand{\tht}{\theta^{\mathrm{+}}}
\newcommand{\vtht}{\vartheta^{\mathrm{+}}}
\newcommand{\vththat}{{\hat{\vartheta}}^{\mathrm{+}}}
\newcommand{\vthtbhat}{{\hat{\vartheta}}^{\pm}}
\newcommand{\vTht}{\varTheta^{\mathrm{+}}}
\newcommand{\vthtb}{\vartheta^{\pm}}
\newcommand{\vThtb}{\varTheta^{\pm}}

% nondimensional numbers
\renewcommand{\Re}{\mathrm{Re}}
\newcommand{\Ro}{\mathrm{Ro}}
\newcommand{\Ri}{\mathrm{Ri}}

%psi's
%Galerking coefficient for psi:
\newcommand{\gpsi}{\breve \psi}
\newcommand{\gphi}{\breve \phi}
\newcommand{\gq}{\breve q}
\newcommand{\gU}{\breve U}
\newcommand{\gQ}{\breve Q}
\newcommand{\gsigma}{\breve \sigma}


\newcommand{\psit}{\psi^{\mathrm{+}}}
\newcommand{\psithat}{{\hat{\psi}}^{\mathrm{+}}}
\newcommand{\psibhat}{{\hat{\psi}}^{\mathrm{-}}}
\newcommand{\psitb}{\psi^{\pm}}
\newcommand{\psitbhat}{{\hat{\psi}}^\pm}
\newcommand{\St}{S^{\mathrm{+}}}
\newcommand{\Sb}{S^{\mathrm{-}}}
\newcommand{\phb}{\phi^{\mathrm{-}}}
\newcommand{\pht}{\phi^{\mathrm{+}}}
\newcommand{\tautb}{\tau^{\pm}}
\newcommand{\sigmatb}{\sigma^{\pm}}


\newcommand{\bur}{\left(\tfrac{f_0}{N}\right)^2}
\newcommand{\ibur}{\left(\tfrac{N}{f_0}\right)^2}
\newcommand{\Nm}{N_{\mathrm{mix}}}
\newcommand{\xim}{\xi_{\mathrm{mix}}}
\newcommand{\hs}{h_*}
\renewcommand{\sp}{\mathsf{p}}
\newcommand{\se}{\mathsf{e}}
\newcommand{\sptb}{\mathsf{p}^\pm}


%nmax is a problem:
%\newcommand{\nmax}{n_{\mathrm{max}}}
\newcommand{\nmax}{\mathrm{N}}

\newcommand{\WKB}{\mathrm{WKB}}
\newcommand{\Lam}{\Lambda}
\newcommand{\tha}{\theta}
\newcommand{\kap}{\kappa}
\newcommand{\bphi}{\boldsymbol{\phi}}
\newcommand{\third}{\tfrac{1}{3}}
\newcommand{\cs}{c^\star}
\newcommand{\nt}{n^{\mathrm{trnc}}}
\newcommand{\sDp}{\mathsf{D}^1_{\nmax}}
\newcommand{\sDpp}{\mathsf{D}^2_{\nmax}}
\newcommand{\sD}{\mathsf{D_2}}
\newcommand{\sK}{\mathsf{K_2}}
\newcommand{\stheta}{\mathsf{\theta}}
\newcommand{\sphi}{\mathsf{\phi}}
\newcommand{\sq}{\mathsf{q}}
\newcommand{\cosech}{\text{csch}\,}
\newcommand{\sinc}{\text{sinc}\,}

%%%%%%%%% %%%%
\newcommand{\zp}{z^+}
\newcommand{\zm}{z^-}
\newcommand{\qA}{q^A_{\nmax}}
\newcommand{\psiB}{\psi^B_{\nmax}}
\newcommand{\phiB}{\phi^B_{\nmax}}
\newcommand{\eye}{\boldsymbol{\hat{i}}}
\newcommand{\jay}{\boldsymbol{\hat{j}}}
\newcommand{\kay}{\boldsymbol{\hat{k}}}
\newcommand{\psiG}{\psi^{\mathrm{G}}}
\newcommand{\qG}{q^{\mathrm{G}}}
\newcommand{\uG}{u^{\mathrm{G}}}
\newcommand{\UG}{U^{\mathrm{G}}}
\newcommand{\UGN}{U^{\mathrm{G}}_{\nmax}}
\newcommand{\QGN}{Q^{\mathrm{G}}_{\nmax}}
\newcommand{\sumoddn}{\sum_{n = 1, n~ \text{odd}}^{\nmax}}

% bretherton 
\newcommand{\qBr}{q_{\mathrm{Br}}}
\newcommand{\psiBr}{\psi_{\mathrm{Br}}}



\maketitle

\section*{Problem 1}

\begin{enumerate}[label=(\alph*)]

    \item With constant density $\rho$ and uniform viscosity $\nu$, the vorticity equation is
        \beq
            \label{eq:vort}
            \frac{\p \omega_i}{\p t}   + u_j \frac{\p \omega_i}{\p x_j} = \underbrace{\omega_j\frac{\p u_i}{\p x_j}}_{(I)} \,+\,\underbrace{\nu \frac{\p^2 \omega_i}{\p x_j^2}}_{(II)}\com
        \eeq
        where the vorticity is defined as the curl of the velocity
        \beq
            \label{eq:curl_v}
            \omega_i \defn \eijk \frac{\p u_k}{\p x_j}\com
        \eeq
        where $\eijk$ is the alternating tensor. Also in \eqref{eq:vort} (I) represents the vorticity generation due to vortex stretching, and (II) is the viscous diffusion of vorticity. In two-dimensional flows, the stretching term is identically zero (I).  To see this, we note that in two dimensions the vorticity \eqref{eq:curl_v} is nonzero only in the direction normal to the plane of the flow. Suppose the plane is defined in the $x_1$-$x_2$ direction, then the only nonzero term in the vorticity is $\omega_3$. The stretching reduces to $\omega_3 \p_{x_3}u_i$, where $i=1,2$. But is independent of $x_3$ because the flow is two-dimensional, and therefore the stretching term vanishes identically. Thus in the absence of  viscosity the vorticity $\omega_3$ is materially conserved in two-dimensions. 

    To form an enstrophy equation we dot the vorticity equation \eqref{eq:vort} with $\omega_i$ to obtain
    \beq
        \label{ens1}
        \frac{\p}{\p t} \frac{\omega_i^2}{2} + u_j \frac{\p}{\p x_j}\frac{\omega_i^2}{2} = \omega_i \omega_j \frac{\p u_i}{\p x_j} + \nu \omega_i \frac{\p^2 \omega_i}{\p x_j^2} \com
    \eeq
    or after rearranging and working a bit on the last term
    \beq
        \label{ens2}
        \frac{\p \omega_i^2}{\p t} + u_j \frac{\p\omega_i^2}{\p x_j} = \underbrace{\omega_i \omega_j \frac{\p^2 u_i}{\p x_j}}_{(I)} + \underbrace{\nu \frac{\p \omega_i^2}{\p x_j^2}}_{(II)} \underbrace{- 2 \nu \left(\frac{\p \omega_i}{\p x_j}\right)^2}_{(III)}\per
    \eeq
    The first term on the right of \eqref{ens2} is the enstrophy generation due to the straining field. To see that we note that $\omega_i \omega_j$ is symmetric. Thus
    \beq
        \label{strain}
        \omega_i \omega_j \frac{\p u_i}{\p x_j} = \omega_i \omega_j s_{ij}\com
    \eeq
    where the rate of strain tensor is
    \beq
        s_{ij} \defn \frac{1}{2}\left(\frac{\p u_i}{\p x_j} + \frac{\p u_j}{\p x_i}\right)\per
    \eeq
    The second term (II) is the divergence of the viscous flux enstrophy. This term only redistributes enstrophy. To see this, we note that in a bounded region with no flux at the boundaries, this term integrates to zero. 

    Finally, the last term is negative definite (provided $\nu>0$; no negative viscosity here!), and it represents the viscous dissipation of enstrophy.

    Because there is no stretching term in two-dimensional flows, there is no enstrophy generation due to straining fields (I). In the absence of viscosity, the two-dimensional flow conserves total enstrophy defined as the area integral of the vorticity squared
    \beq
        \label{eq:tot_ens}
        Z \defn \int \int \omega_3^2 \dd S\com
    \eeq
    provided the is no flux of vorticity across the boundaries. This follows directly from the material conservation of vorticity in two dimensions with $\nu =0$.

\end{enumerate}



\end{document}


