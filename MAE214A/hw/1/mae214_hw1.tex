\documentclass[11pt]{article}


%% WRY has commented out some unused packages %%
%% If needed, activate these by uncommenting
\usepackage{geometry}                % See geometry.pdf to learn the layout options. There are lots.
%\geometry{letterpaper}                   % ... or a4paper or a5paper or ... 
\geometry{a4paper,left=2.5cm,right=2.5cm,top=2.5cm,bottom=2.5cm}
%\geometry{landscape}                % Activate for rotated page geometry
%\usepackage[parfill]{parskip}    % Activate to begin paragraphs with an empty line rather than an indent

%for figures
%\usepackage{graphicx}

\usepackage{color}
\definecolor{mygreen}{RGB}{28,172,0} % color values Red, Green, Blue
\definecolor{mylilas}{RGB}{170,55,241}
%% for graphics this one is also OK:
\usepackage{epsfig}

%% AMS mathsymbols are enabled with
\usepackage{amssymb,amsmath}

%% more options in enumerate
\usepackage{enumerate}
\usepackage{enumitem}

%% insert code
\usepackage{listings}

\usepackage[utf8]{inputenc}

\usepackage{hyperref}


% Default fixed font does not support bold face
\DeclareFixedFont{\ttb}{T1}{txtt}{bx}{n}{12} % for bold
\DeclareFixedFont{\ttm}{T1}{txtt}{m}{n}{12}  % for normal

% Custom colors
\usepackage{color}
\definecolor{deepblue}{rgb}{0,0,0.5}
\definecolor{deepred}{rgb}{0.6,0,0}
\definecolor{deepgreen}{rgb}{0,0.5,0}


% Python style for highlighting
\newcommand\pythonstyle{\lstset{
language=Python,
basicstyle=\ttm,
otherkeywords={self},             % Add keywords here
keywordstyle=\ttb\color{deepblue},
emph={MyClass,__init__},          % Custom highlighting
emphstyle=\ttb\color{deepred},    % Custom highlighting style
stringstyle=\color{deepgreen},
frame=tb,                         % Any extra options here
showstringspaces=false            % 
}}

% Python environment
\lstnewenvironment{python}[1][]
{
\pythonstyle
\lstset{#1}
}
{}

% Python for external files
\newcommand\pythonexternal[2][]{{
\pythonstyle
\lstinputlisting[#1]{#2}}}

% Python for inline
\newcommand\pythoninline[1]{{\pythonstyle\lstinline!#1!}}

%\usepackage{epstopdf}
%\DeclareGraphicsRule{.tif}{png}{.png}{`convert #1 `dirname #1`/`basename #1 .tif`.png}



%% colors
\usepackage{graphicx,xcolor,lipsum}


\usepackage{mathtools}

\usepackage{graphicx}
\newcommand*{\matminus}{%
  \leavevmode
  \hphantom{0}%
  \llap{%
    \settowidth{\dimen0 }{$0$}%
    \resizebox{1.1\dimen0 }{\height}{$-$}%
  }%
}


\title{MAE214, Homework Assignment 1}
\author{Cesar B Rocha}
\date{\today}

\begin{document}

\newcommand{\com}{\, ,}
\newcommand{\per}{\, .}

%% Averages
% Use \bar to over line solo symbols

\newcommand{\av}[1]{\bar{#1}}
\newcommand{\avbg}[1]{\overline{#1}}
\newcommand{\avbgg}[1]{\overline{#1}}

% A nice definition
\newcommand{\defn}{\ensuremath{\stackrel{\mathrm{def}}{=}}}

% equations
\def\beq{\begin{equation}}
\def\eeq{\end{equation}}

% calculus

\newcommand{\p}{\partial}
\newcommand{\ii}{{\rm i}}
\newcommand{\dd}{{\rm d}}
\newcommand{\id}{{\, \rm d}}
\newcommand{\ee}{{\rm e}}
\newcommand{\DD}{{\rm D}}
\newcommand{\wavy}{\text{wavy}}
\newcommand{\qg}{\text{qg}}


\newcommand{\be}{\beta}

\newcommand{\al}{\alpha}
\newcommand{\bx}{\boldsymbol{x}}
\newcommand{\by}{\boldsymbol{y}}
\newcommand{\bu}{\boldsymbol{u}}
\newcommand{\bv}{\boldsymbol{v}}


\newcommand{\half}{\tfrac{1}{2}}
\newcommand{\halfrho}{\tfrac{1}{2}}
\newcommand{\rz}{{}}
\newcommand{\bn}{\boldsymbol{\hat n}}
\newcommand{\br}{\boldsymbol{r}}
\newcommand{\bR}{\boldsymbol{R}}
\newcommand{\bA}{\ensuremath {\boldsymbol {A}}}
\newcommand{\bB}{\ensuremath {\boldsymbol {B}}}
\newcommand{\bU}{\ensuremath {\boldsymbol {U}}}
\newcommand{\bE}{\ensuremath {\boldsymbol {E}}}
\newcommand{\bJ}{\ensuremath {\boldsymbol {J}}}
\newcommand{\bXX}{\ensuremath {\boldsymbol {\mathcal{X}}}}
\newcommand{\bFF}{\ensuremath {\boldsymbol {F}}}
\newcommand{\bF}{\ensuremath {\boldsymbol {F}^{\sharp}}}
\newcommand{\bG}{\ensuremath {\boldsymbol G}}
\newcommand{\bSigma}{\ensuremath {\boldsymbol {\Sigma}}}
\newcommand{\bvarphi}{\ensuremath {\boldsymbol {\varphi}}}
\newcommand{\bxi}{\ensuremath {\boldsymbol {\xi}}}
\newcommand{\avbxi}{\overline{\ensuremath {\boldsymbol {\xi}}}}

% math cal

\newcommand{\J}{\mathcal{J}}
\newcommand{\K}{\mathcal{K}}
\newcommand{\cG}{\mathcal{G}}
\newcommand{\cF}{\mathcal{F}}
\newcommand{\cN}{\mathcal{N}}
\newcommand{\cL}{\mathcal{L}}


% san serif for matrices and differential operators
%\newcommand{\helmn}{\mathsf{H}_n}
\newcommand{\helmm}{\triangle_m}
\newcommand{\helmn}{\triangle_n}
\newcommand{\helms}{\triangle_s}
\newcommand{\helm}{\triangle}
\newcommand{\sA}{\mathsf{A}}
\newcommand{\sB}{\mathsf{B}}
\newcommand{\sG}{\mathsf{G}}
\newcommand{\sI}{\mathsf{I}}
\newcommand{\sJ}{\mathsf{J}}
\newcommand{\sU}{\mathsf{U}}
\newcommand{\sP}{\mathsf{P}}
\newcommand{\sQ}{\mathsf{Q}}
\newcommand{\sR}{\mathsf{R}}
\newcommand{\sL}{\mathsf{L}}
\renewcommand{\sJ}{\mathsf{J}}
\renewcommand{\sI}{\mathsf{I}}
\renewcommand{\L}{\mathsf{L}}
\newcommand{\sM}{\mathsf{M}}
\newcommand{\sT}{\mathsf{T}}
\newcommand{\sGamma}{\mathsf{\Gamma}}
\newcommand{\sOmega}{\mathsf{\Omega}}
\newcommand{\sSigma}{\mathsf{\Omega}}
\newcommand{\sbeta}{\mathsf{\beta}}
\newcommand{\sPi}{\mathsf{\Pi}}
\newcommand{\sC}{\mathsf{C}}
\newcommand{\sQy}{\mathsf{Q}}


% angle brackets

\def\la{\langle}
\def\ra{\rangle}
\def\laa{\left \langle}
\def\raa{\right \rangle}


%grads and div's
\newcommand{\bcdot}{\hspace{-0.1em} \boldsymbol{\cdot} \hspace{-0.12em}}
\newcommand{\bnabla}{\boldsymbol{\nabla}}
\newcommand{\bnablaH}{\bnabla_{\! \mathrm{h}}}
\newcommand{\grad}{\bnabla}
\newcommand{\gradH}{\bnablaH}
\newcommand{\curl}{\bnabla \!\times\!}
\newcommand{\diver}{\bnabla \bcdot }
\newcommand{\cross}{\times}
\newcommand{\lap}{\triangle}


%varthetas and thetas
\newcommand{\vth}{\vartheta}
\newcommand{\psii}{\psi^{\mathrm{i}}}
\newcommand{\thb}{\theta^{\mathrm{-}}}
\newcommand{\vthb}{\vartheta^{\mathrm{-}}}
\newcommand{\vthbhat}{{\hat{\vartheta}}^{\mathrm{-}}}
\newcommand{\vThb}{\varTheta^{\mathrm{-}}}
\newcommand{\psib}{\psi^{\mathrm{-}}}
\newcommand{\tht}{\theta^{\mathrm{+}}}
\newcommand{\vtht}{\vartheta^{\mathrm{+}}}
\newcommand{\vththat}{{\hat{\vartheta}}^{\mathrm{+}}}
\newcommand{\vthtbhat}{{\hat{\vartheta}}^{\pm}}
\newcommand{\vTht}{\varTheta^{\mathrm{+}}}
\newcommand{\vthtb}{\vartheta^{\pm}}
\newcommand{\vThtb}{\varTheta^{\pm}}

% nondimensional numbers
\renewcommand{\Re}{\mathrm{Re}}
\newcommand{\Ro}{\mathrm{Ro}}
\newcommand{\Ri}{\mathrm{Ri}}

%psi's
%Galerking coefficient for psi:
\newcommand{\gpsi}{\breve \psi}
\newcommand{\gphi}{\breve \phi}
\newcommand{\gq}{\breve q}
\newcommand{\gU}{\breve U}
\newcommand{\gQ}{\breve Q}
\newcommand{\gsigma}{\breve \sigma}


\newcommand{\psit}{\psi^{\mathrm{+}}}
\newcommand{\psithat}{{\hat{\psi}}^{\mathrm{+}}}
\newcommand{\psibhat}{{\hat{\psi}}^{\mathrm{-}}}
\newcommand{\psitb}{\psi^{\pm}}
\newcommand{\psitbhat}{{\hat{\psi}}^\pm}
\newcommand{\St}{S^{\mathrm{+}}}
\newcommand{\Sb}{S^{\mathrm{-}}}
\newcommand{\phb}{\phi^{\mathrm{-}}}
\newcommand{\pht}{\phi^{\mathrm{+}}}
\newcommand{\tautb}{\tau^{\pm}}
\newcommand{\sigmatb}{\sigma^{\pm}}


\newcommand{\bur}{\left(\tfrac{f_0}{N}\right)^2}
\newcommand{\ibur}{\left(\tfrac{N}{f_0}\right)^2}
\newcommand{\Nm}{N_{\mathrm{mix}}}
\newcommand{\xim}{\xi_{\mathrm{mix}}}
\newcommand{\hs}{h_*}
\renewcommand{\sp}{\mathsf{p}}
\newcommand{\se}{\mathsf{e}}
\newcommand{\sptb}{\mathsf{p}^\pm}


%nmax is a problem:
%\newcommand{\nmax}{n_{\mathrm{max}}}
\newcommand{\nmax}{\mathrm{N}}

\newcommand{\WKB}{\mathrm{WKB}}
\newcommand{\Lam}{\Lambda}
\newcommand{\tha}{\theta}
\newcommand{\kap}{\kappa}
\newcommand{\bphi}{\boldsymbol{\phi}}
\newcommand{\third}{\tfrac{1}{3}}
\newcommand{\cs}{c^\star}
\newcommand{\nt}{n^{\mathrm{trnc}}}
\newcommand{\sDp}{\mathsf{D}^1_{\nmax}}
\newcommand{\sDpp}{\mathsf{D}^2_{\nmax}}
\newcommand{\sD}{\mathsf{D_2}}
\newcommand{\sK}{\mathsf{K_2}}
\newcommand{\stheta}{\mathsf{\theta}}
\newcommand{\sphi}{\mathsf{\phi}}
\newcommand{\sq}{\mathsf{q}}
\newcommand{\cosech}{\text{csch}\,}
\newcommand{\sinc}{\text{sinc}\,}

%%%%%%%%% %%%%
\newcommand{\zp}{z^+}
\newcommand{\zm}{z^-}
\newcommand{\qA}{q^A_{\nmax}}
\newcommand{\psiB}{\psi^B_{\nmax}}
\newcommand{\phiB}{\phi^B_{\nmax}}
\newcommand{\eye}{\boldsymbol{\hat{i}}}
\newcommand{\jay}{\boldsymbol{\hat{j}}}
\newcommand{\kay}{\boldsymbol{\hat{k}}}
\newcommand{\psiG}{\psi^{\mathrm{G}}}
\newcommand{\qG}{q^{\mathrm{G}}}
\newcommand{\uG}{u^{\mathrm{G}}}
\newcommand{\UG}{U^{\mathrm{G}}}
\newcommand{\UGN}{U^{\mathrm{G}}_{\nmax}}
\newcommand{\QGN}{Q^{\mathrm{G}}_{\nmax}}
\newcommand{\sumoddn}{\sum_{n = 1, n~ \text{odd}}^{\nmax}}

% bretherton 
\newcommand{\qBr}{q_{\mathrm{Br}}}
\newcommand{\psiBr}{\psi_{\mathrm{Br}}}



\maketitle

\section*{Problem 1}

\begin{enumerate}[label=(\alph*)]

    \item With constant density $\rho$ and uniform viscosity $\nu$, the vorticity equation is
        \beq
            \label{eq:vort}
            \frac{\p \omega_i}{\p t}   + u_j \frac{\p \omega_i}{\p x_j} = \underbrace{\omega_j\frac{\p u_i}{\p x_j}}_{(I)} \,+\,\underbrace{\nu \frac{\p^2 \omega_i}{\p x_j^2}}_{(II)}\com
        \eeq
        where the vorticity is defined as the curl of the velocity
        \beq
            \label{eq:curl_v}
            \omega_i \defn \eijk \frac{\p u_k}{\p x_j}\com
        \eeq
        where $\eijk$ is the alternating tensor. Also in \eqref{eq:vort}, (I) represents the vorticity generation due to vortex stretching, and (II) is the viscous diffusion of vorticity. In two-dimensional flows, the stretching term (I) is identically zero.  To see this, we note that in two dimensions the vorticity \eqref{eq:curl_v} is nonzero only in the direction normal to the plane of the flow. Suppose the plane is defined in the $x_1$-$x_2$ direction, then the only nonzero term in the vorticity is $\omega_3$. The stretching reduces to $\omega_3 \p_{x_3}u_i$, where $i=1,2$. But is independent of $x_3$ because the flow is two-dimensional, and therefore the stretching term vanishes identically. Thus in the absence of  viscosity the vorticity $\omega_3$ is materially conserved in two-dimensions. 

    To form an enstrophy equation we dot the vorticity equation \eqref{eq:vort} with $\omega_i$ to obtain
    \beq
        \label{ens1}
        \frac{\p}{\p t} \frac{\omega_i^2}{2} + u_j \frac{\p}{\p x_j}\frac{\omega_i^2}{2} = \omega_i \omega_j \frac{\p u_i}{\p x_j} + \nu \omega_i \frac{\p^2 \omega_i}{\p x_j^2} \com
    \eeq
    or after rearranging and working a bit on the last term
    \beq
        \label{ens2}
        \frac{\p \omega_i^2}{\p t} + u_j \frac{\p\omega_i^2}{\p x_j} = \underbrace{\omega_i \omega_j \frac{\p^2 u_i}{\p x_j}}_{(I)} + \underbrace{\nu \frac{\p \omega_i^2}{\p x_j^2}}_{(II)} \underbrace{- 2 \nu \left(\frac{\p \omega_i}{\p x_j}\right)^2}_{(III)}\per
    \eeq
    The first term (I) on the right of \eqref{ens2} is the enstrophy generation due to the straining field. To see that we note that $\omega_i \omega_j$ is symmetric. Thus
    \beq
        \label{strain}
        \omega_i \omega_j \frac{\p u_i}{\p x_j} = \omega_i \omega_j s_{ij}\com
    \eeq
    where the rate of strain tensor is
    \beq
        s_{ij} \defn \frac{1}{2}\left(\frac{\p u_i}{\p x_j} + \frac{\p u_j}{\p x_i}\right)\per
    \eeq
    The second term (II) is the divergence of the viscous flux of enstrophy. This term only redistributes enstrophy. That is, in a bounded region with no flux at the boundaries, this term integrates to zero. 

    Finally, the last term is negative definite (provided $\nu>0$; no negative viscosity here!), and it represents the viscous dissipation of enstrophy.

    Because there is no stretching term in two-dimensional flows, there is no enstrophy generation due to straining fields (I). In the absence of viscosity, the two-dimensional flow conserves total enstrophy defined as the area integral of the vorticity squared
    \beq
        \label{eq:tot_ens}
        Z \defn \int \int \omega_3^2 \dd S\com
    \eeq
    provided the is no flux of vorticity across the boundaries. This follows directly from the material conservation of vorticity in two dimensions with $\nu =0$.

\item The buoyancy term is $-g\rho/\rho_0 \hat{x}_3$, or in index notation $-g\rho/\rho_0 \delta_{i3}$, where $\delta$ is the Kronecker delta. The curl of this term is
    \beq
        \label{eq:curl_buoy}
        \eijk \frac{\p}{\p x_j}\left(-\frac{\rho}{\rho_0}g \delta_{k3}\right) = \varepsilon_{ij3}  \frac{\p}{\p x_j}\left(-\frac{\rho}{\rho_0}g \right)\per
    \eeq
     The nonzero components of \eqref{eq:curl_buoy} are
     \begin{align}
     i=1,j=2:\qquad -\frac{g}{\rho_0} \frac{\p \rho}{\p x_2} \hat{x}_2\nonumber \\
     i=2,j=1:\qquad  \frac{g}{\rho_0} \frac{\p \rho}{\p x_1}\hat{x}_1\per
    \end{align}
    Thus this term is, in vector notation,
    \beq
        \frac{g}{\rho_0} \hat{x}_3 \times \nabla_{H} \rho\com 
    \eeq
    where $\nabla_H$ represents the horizontal gradient operator. Notice that to include the ``gravity'' in the form of $-g \rho/\rho_0 $ we are already making an approximation. This term accounts for the generation of vorticity due to the existence of horizontal buoyancy gradients.
 .  The enstrophy equation also has an extra term that represents the generation of enstrophy due to horizontal buoyancy gradients
    \beq
        -2 \omega_i \varepsilon_{ij3}  \frac{\p}{\p x_j}\left(\frac{\rho}{\rho_0}g \right)\per
    \eeq

    We are saying that $\rho << \rho_0$, and removing the hydrostatic component of the pressure field from the $x_3$-momentum equation (this is  known as the Boussinesq approximation). A different approach is to consider the full NS equations. Then the acceleration term is $g$, and its curl is zero (body forces do not generate vorticity). But because of the nonuniform density $\rho$ we have an additional term stemming from the pressure gradient force when computing the curl of the momentum equation:
    \beq
    \nabla \times \left(\frac{\nabla p}{\rho}\right) = -\frac{\nabla \rho \times \nabla p}{\rho^2} = -\frac{1}{\rho^2}\eijk \frac{\p \rho}{\p x_j}\frac{\p p}{\p x_k} \neq 0
    \eeq
    for nonuniform density. This term represents the production of vorticity due to the baroclinicity of the flow (that is, the misalignment between density and pressure surfaces). The associated enstrophy production due to baroclinicity is
    \beq
       -\frac{\omega_i}{\rho^2}\eijk \frac{\p \rho}{\p x_j}\frac{\p p}{\p x_k} \neq 0\per
    \eeq
     
    \end{enumerate}


\section*{Problem 2 -- Pope's exercise 4.5}
In this problem we consider the results from an experiment for a flow in which $\p <U_1>/\p_{x_2}$ is the only nonzero component of the velocity gradient. The measured Reynolds stress tensor normalized by the eddy kinetic energy is
\beq
            \label{eq:reynolds}
\frac{<u_i u_j>}{k} =  \left[ \begin{matrix}
        1.08 & -0.32 & 0. \\
        -0.32 & 0.40 & 0 \\
        0 & 0 & 0.52
\end{matrix} \right]\per
\eeq

\begin{enumerate}[label=(\alph*)]
    \item The normalized anisotropy (a.k.a the deviatoric tensor)  is
        \begin{align}
            b_{ij} \defn \frac{<u_i u_j>}{k} - \frac{1}{3}\delta_{ij} &=  \left[ \begin{matrix}
        1.08 & -0.32 & 0. \\
        -0.32 & 0.40 & 0 \\
        0 & 0 & 0.52
\end{matrix} \right] - \frac{1}{3} \left[ \begin{matrix}
        1 & 0 & 0. \\
        0 & 1 & 0 \\
        0 & 0 & 1
\end{matrix} \right] \nonumber \\  &= \left[ \begin{matrix}
        0.74666667 & -0.32 & 0 \\
        -0.32 &  0.06666667 &  0   \\
0 &  0 &0.18666667\end{matrix} \right]\per
    \end{align}

    \item The correlation between the eddy velocity components $u_1$ and $u_2$ is
        \beq
            r_{12} = \frac{<u_1 u_2>}{\sqrt{<u_1^2><u_2^2>}} = -0.48686449556014766\per
            \eeq


    \item First, we note that because the last row of \eqref{eq:reynolds} is already in diagonal form, we do not need to consider a third dimension. Instead, we can work with the matrix
        \beq
      \sU_2 \defn        \left[ \begin{matrix}
        1.08 & -0.32 \\
        -0.32 & 0.40 
\end{matrix} \right]\com
        \eeq
    and look for the rotation matrix
    \beq
    \sR \defn
        \left[ \begin{matrix}
        \cos \theta & \sin \theta \\
        -\sin \theta & \cos \theta 
\end{matrix} \right]
    \eeq
    that diagonalizes the problem. That is $\sR \sU_2 = \sD \sR$, where $\sD$ is a diagonal matrix. To find the angle $\theta$, we simply demand that the off-diagonal terms of $\sD = \sR \sU_2 \sR^\sT$ be zero. This requires
    \beq
        -0.32 \cos^2\theta + 0.68 \cos\theta\sin\theta + 0.32 \sin^2\theta = 0\per
    \eeq
    We solve the transcendental equation above using Newton's method. With an initial guess of $\pi/4$\footnote{Other guesses lead to converged solutions that are shifted by $\pi$. This only changes the position of the diagonal terms in $\sD$}, the solution converge to $\theta = 1.119  \,(68.37^\circ)$. The diagonalized Reynolds stress is
    \beq
    \left[ \begin{matrix}
        0.23 & 0 & 0. \\
        0 & 1.21 & 0 \\
        0 & 0 & 0.52
\end{matrix} \right]\per            
    \eeq

    \item The mean rate of strain tensor is
        \beq
        \sS =  \left[ \begin{matrix}
        0 & \p U_1/\p x_2 & 0. \\
         \p U_1/\p x_2 & 0 & 0 \\
        0 & 0 & 0
\end{matrix} \right]
        \eeq
        Using the same procedure employed above we obtain the condition for digitalization
        \beq
            \frac{\p U_1}{\p x_2}\left(\cos^2\theta - \sin^2\theta \right) = 0\per
        \eeq
        Here we can solve analytically to obtain $\theta= \pi/4 \, (45^\circ)$. The diagonalized mean rate of strain tensor is
        \beq
        \sS =  \left[ \begin{matrix}
        \p U_1/\p x_2 & 0 & 0. \\
         0 & -\p U_1/\p x_2 & 0 \\
            0 & 0 & 0 
\end{matrix} \right] \per
        \eeq


    \item Clearly, one of the eigenvalues of \eqref{eq:reynolds} is $0.52$. To find the other two we solve the quadratic equation
        \beq
            (1.08 - \lambda)(0.4 - \lambda) - 0.32^2 = 0\per
        \eeq
        We obtain $\lambda^+ \approx 0.27309$ and $\lambda^- \approx 1.20690$. Thus the eigenvalues of the Reynolds tensor   \eqref{eq:reynolds}  are $\{0.27309, 0.52000,  1.20690\}$.

\end{enumerate}

\section*{Problem 3}

\begin{enumerate}[label=(\alph*)]
    
    \item We begin with the observation that this is a two-dimensional problem. The horizontal coordinate is $x$, and the vertical coordinate is $y$. The Navier-Stokes equations are
        \begin{align}
            \label{ns}
            \p_t u + u\p_x u + v\p_y u &= -\rho^{-1}\p_x p + \nu  \left(\p^2_{xx}u +\p^2_{yy}u\right)\com\nonumber\\
            \p_t v + u\p_x v + v\p_y v &= -\rho^{-1}\p_y p + \nu  \left(\p^2_{xx}v +\p^2_{yy}v\right)\com\nonumber\\
            \p_x u + \p_y v& = 0\per
        \end{align}
        We will look for steady solution, so the first assumption is that $(i)$ \textit{the steady state can be achieve}. This assumption is easily verified by the existence of solutions to the steady equations. The second assumption is that $(ii)$ \textit{the Reynolds number is large}
        \beq
        \Re \defn \frac{U L}{\nu} >>> 1\per
        \eeq
        This loosely mean that the viscosity $\nu$ is very small, and we might be tempted to ignore the viscous terms  in \eqref{ns}. But clearly those terms have the highest order spatial derivatives, and $\nu = 0$ is a singular limit. Hence, there must be a region next to the boundary where the viscous term contributes to the leading order momentum balance. In this region the scale of variations is $\mathcal{O(\delta)}$, and we might anticipate that viscous terms balance the inertial terms. Using $U$ as the characteristic scale of the velocity $u$, L for the variations in the $x$-direction and $\delta$ for the spatial scales of variations in the $y$-direction, we obtain
        \beq
        \label{delta}
        \delta \sim \sqrt{\frac{\nu L}{U}}\com \qquad \text{and hence} \qquad\frac{\delta}{L} \sim \frac{1}{\sqrt{\Re}}\per
        \eeq
    That is, the boundary layer thickness is inversely proportion to the square-root of the Reynolds number. We use this information to rescale the Navier-Stokes equations to obtain the boundary layer approximate equations. To obtain this approximation systematically, we use nondimensional variables. First we notice that outside the boundary layer the dominant balance is between inertia and the pressure gradient force. Thus, an estimate for the pressure scale is
    \beq
    \label{P}
        P = \rho U^2\per
    \eeq
    We now introduce nondimensional dependent variables 
    \beq
    \label{scales}
    \ndu \defn \frac{u}{U}\com\qquad \ndv \defn \frac{v}{U}\com\qqand \np \defn \frac{p}{P}\com
    \eeq
    where we use $\breve .$ to denote nondimensional quantity. Introducing the scales \eqref{scales} in the steady version of \eqref{ns}, and using  \eqref{delta} and \eqref{P}, we obtain the nondimensional steady NS equations 
        \begin{align}
            \label{ns_nd}
            \ndu\p_{\nx} \ndu + \ndv\p_{\ny} \ndu &= -\p_{\nx} \np + \Re^{-1} \p^2_{\nx\nx}\ndu +\p^2_{\ny\ny}\ndu\com\nonumber\\
            \Re^{-1} \left(\ndu\p_{\nx} \ndv + \ndv\p_{\ny} \ndv\right) &= -\p_{\ny} \np + \Re^{-2} \p^2_{\nx\nx}\ndv +  \Re^{-1} \p^2_{\nx\nx}\ndv\com\nonumber\\
                                                                        &\p_{\nx} \ndu + \p_{\ny} \ndv = 0\per
        \end{align}
        Thus using assumption $(ii)$ we obtain the boundary layer equations
        \begin{align}
            \label{ns__bl}
            \ndu\p_{\nx} \ndu + \ndv\p_{\ny} \ndu &= -\p_{\nx} \np + \p^2_{\ny\ny}\ndu\com\nonumber\\
                                                  & 0 = -\p_{\ny} \np \nonumber \com\\
             &\p_{\nx} \ndu + \p_{\ny} \ndv = 0\com
        \end{align}
        or in keeping the same terms in dimensional the equation \eqref{ns} 
\begin{align}
            \label{ns_bl}
           u\p_x u + v\p_y u &= -\rho^{-1}\p_x p + \nu  \p^2_{yy}u\com\nonumber\\
         0 &= -\rho^{-1}\p_y p \nonumber\com \\
           &\p_x u + \p_y v= 0\per
\end{align}

\item A solution (dependent variable) is self-similar if it is independent of a particular combination of dependent parameters. For instance the nondimensional profile of a jet is self-similar if the is independent of down-stream direction; the profile of the jet collapse to a single, self-similar curve under a particular nondimensionalization. It is possible to find self-similar solutions whenever there is no external length scales imposed to the problem. For the boundary layer initial value problem we can imagine that the flow will become self-similar provided we are a distance significantly away from the initial contact location. While the thickness of the boundary layer may increase with downstream distance $x$, the shape of the solution with appropriate scaling may collapse to a single curve.

We assume the self-similar velocity
\beq
\label{eq:self_similar_vel}
g(\eta) \defn \frac{u(x,y)}{U}\com\qquad \text{with} \qquad \eta \defn \frac{y}{\delta(x)}\com
\eeq
and streamfunction
\beq
\label{eq:self_similar_sf}
f(\eta) \defn \frac{\psi(x,y)}{U \delta(x)}\per
\eeq
If self-similar solutions exist, we must be able to derive a differential equation for $g$ or $f$ which is only depends on the similarity independent variable $\eta$.

The definition of the streamfunction follows from the two-dimensional nondivergence  in \eqref{ns_bl}
\beq
\label{sf}
u \defn \frac{\p \psi}{\p y}\com\qqand v \defn -\frac{\p \psi}{\p x}\per
\eeq
Using \eqref{sf} the $x$-momentum equation \eqref{ns__bl} becomes
\beq
\label{blasius1}
\p_y\psi \p_x\p_y\psi - \p_x\psi\p_y\p_y\psi = - \nu \p^3_{yyy} \psi\per
\eeq
The boundary conditions are no-slip at the boundary
\beq
\label{eq:bc}
u = \p_y \psi = 0\com \qquad \text{at} \qquad y=0\com
\eeq
and the matching condition
\beq
\label{eq:match}
u = -\p_y \psi = U\com \qquad \text{for} \qquad y >> \delta\com
\eeq
that is the flow matches the free-stream velocity outside the boundary layer.

We now write the governing equation  \eqref{blasius1} is terms of similarity variables. First we note that
\beq
\p_y \psi = \p_y\left[U \delta(x) f(\eta)\right] = U f'\com
\eeq
\beq
\p^2_{yy} \psi = \frac{U}{\delta}  f''\com\qqand \p^3_{yyy}\psi = \frac{U}{\delta^2}f''' \per
\eeq
where $f' = \dd f / \dd \eta$. Similarly
\beq
\p_x \psi = \p_x\left[U \delta(x) f(\eta)\right] = U \delta' f - U \frac{f'}{\delta}y \delta' = U\delta'\left(f - f'\eta\right)\com
\eeq
where $\delta' = \dd \delta/\dd x$. Hence
\beq
\p^2_{yx} \psi = U\delta' \p_y \left(f' - f'\eta\right) =  -U \frac{\delta'}{\delta}\eta f''
\eeq
Using these expression in \eqref{blasius1} we obtain
\beq
-U^2 \frac{\delta'}{\delta}\eta f f''= \nu \frac{U}{\delta^2}f'''\com
\eeq
which simplifies to
\beq
\label{blasius}
-\frac{U \delta \delta'}{\nu} f f'' = f'''\per
\eeq
As discussed above, for similarity solutions to exit, the must be no dependence on dimensional variables. This is only possible if the coefficient in the equation above is constant.  Hence we obtain the Blasius equation
\beq
\label{blasius2}
 f'''+ C f f'' = 0\com
\eeq
where $C$ is a constant. Both Batchelor and Kundu choose this constant to be $1/2$ for ``latter algebraic convenience'', they claim. It is important to realize, however, that this choise of  $C$ is arbitrary; the constant is undetermined form the given data for this problem.

The boundary conditions in terms of $f$ follow from \eqref{eq:bc}
\beq
\p_y\psi = 0\com y=0 \qquad \Rightarrow \qquad f' = 0\com \eta = 0\com
\eeq
and \eqref{eq:match}
\beq
u \to U\com \frac{y}{\delta}\to \infty \qquad \Rightarrow \qquad f' \to 1\com \eta \to \infty\per
\eeq
A third boundary condition on $f$ comes from no-normal flow at the boundary $v = 0\com y=0$
\beq
\p_x\psi = 0\com y=0 \qquad \Rightarrow \qquad f = 0\com \eta = 0\com
\eeq
since $f'(\eta=0)$.

\end{enumerate}

\end{document}


