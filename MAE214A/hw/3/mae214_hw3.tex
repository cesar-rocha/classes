\documentclass[11pt]{article}

%% WRY has commented out some unused packages %%
%% If needed, activate these by uncommenting
\usepackage{geometry}                % See geometry.pdf to learn the layout options. There are lots.
%\geometry{letterpaper}                   % ... or a4paper or a5paper or ... 
\geometry{a4paper,left=2.5cm,right=2.5cm,top=2.5cm,bottom=2.5cm}
%\geometry{landscape}                % Activate for rotated page geometry
%\usepackage[parfill]{parskip}    % Activate to begin paragraphs with an empty line rather than an indent

%for figures
%\usepackage{graphicx}

\usepackage{color}
\definecolor{mygreen}{RGB}{28,172,0} % color values Red, Green, Blue
\definecolor{mylilas}{RGB}{170,55,241}
%% for graphics this one is also OK:
\usepackage{epsfig}

%% AMS mathsymbols are enabled with
\usepackage{amssymb,amsmath}

%% more options in enumerate
\usepackage{enumerate}
\usepackage{enumitem}

%% insert code
\usepackage{listings}

\usepackage[utf8]{inputenc}

\usepackage{hyperref}


% Default fixed font does not support bold face
\DeclareFixedFont{\ttb}{T1}{txtt}{bx}{n}{12} % for bold
\DeclareFixedFont{\ttm}{T1}{txtt}{m}{n}{12}  % for normal

% Custom colors
\usepackage{color}
\definecolor{deepblue}{rgb}{0,0,0.5}
\definecolor{deepred}{rgb}{0.6,0,0}
\definecolor{deepgreen}{rgb}{0,0.5,0}


% Python style for highlighting
\newcommand\pythonstyle{\lstset{
language=Python,
basicstyle=\ttm,
otherkeywords={self},             % Add keywords here
keywordstyle=\ttb\color{deepblue},
emph={MyClass,__init__},          % Custom highlighting
emphstyle=\ttb\color{deepred},    % Custom highlighting style
stringstyle=\color{deepgreen},
frame=tb,                         % Any extra options here
showstringspaces=false            % 
}}

% Python environment
\lstnewenvironment{python}[1][]
{
\pythonstyle
\lstset{#1}
}
{}

% Python for external files
\newcommand\pythonexternal[2][]{{
\pythonstyle
\lstinputlisting[#1]{#2}}}

% Python for inline
\newcommand\pythoninline[1]{{\pythonstyle\lstinline!#1!}}

%\usepackage{epstopdf}
%\DeclareGraphicsRule{.tif}{png}{.png}{`convert #1 `dirname #1`/`basename #1 .tif`.png}



%% colors
\usepackage{graphicx,xcolor,lipsum}


\usepackage{mathtools}

\usepackage{graphicx}
\newcommand*{\matminus}{%
  \leavevmode
  \hphantom{0}%
  \llap{%
    \settowidth{\dimen0 }{$0$}%
    \resizebox{1.1\dimen0 }{\height}{$-$}%
  }%
}


\title{MAE214, Homework Assignment 3}
\author{Cesar B Rocha}
\date{\today}

\begin{document}

\newcommand{\com}{\, ,}
\newcommand{\per}{\, .}

%% Averages
% Use \bar to over line solo symbols

\newcommand{\av}[1]{\bar{#1}}
\newcommand{\avbg}[1]{\overline{#1}}
\newcommand{\avbgg}[1]{\overline{#1}}

% A nice definition
\newcommand{\defn}{\ensuremath{\stackrel{\mathrm{def}}{=}}}

% equations
\def\beq{\begin{equation}}
\def\eeq{\end{equation}}

% calculus

\newcommand{\p}{\partial}
\newcommand{\ii}{{\rm i}}
\newcommand{\dd}{{\rm d}}
\newcommand{\id}{{\, \rm d}}
\newcommand{\ee}{{\rm e}}
\newcommand{\DD}{{\rm D}}
\newcommand{\wavy}{\text{wavy}}
\newcommand{\qg}{\text{qg}}


\newcommand{\be}{\beta}

\newcommand{\al}{\alpha}
\newcommand{\bx}{\boldsymbol{x}}
\newcommand{\by}{\boldsymbol{y}}
\newcommand{\bu}{\boldsymbol{u}}
\newcommand{\bv}{\boldsymbol{v}}


\newcommand{\half}{\tfrac{1}{2}}
\newcommand{\halfrho}{\tfrac{1}{2}}
\newcommand{\rz}{{}}
\newcommand{\bn}{\boldsymbol{\hat n}}
\newcommand{\br}{\boldsymbol{r}}
\newcommand{\bR}{\boldsymbol{R}}
\newcommand{\bA}{\ensuremath {\boldsymbol {A}}}
\newcommand{\bB}{\ensuremath {\boldsymbol {B}}}
\newcommand{\bU}{\ensuremath {\boldsymbol {U}}}
\newcommand{\bE}{\ensuremath {\boldsymbol {E}}}
\newcommand{\bJ}{\ensuremath {\boldsymbol {J}}}
\newcommand{\bXX}{\ensuremath {\boldsymbol {\mathcal{X}}}}
\newcommand{\bFF}{\ensuremath {\boldsymbol {F}}}
\newcommand{\bF}{\ensuremath {\boldsymbol {F}^{\sharp}}}
\newcommand{\bG}{\ensuremath {\boldsymbol G}}
\newcommand{\bSigma}{\ensuremath {\boldsymbol {\Sigma}}}
\newcommand{\bvarphi}{\ensuremath {\boldsymbol {\varphi}}}
\newcommand{\bxi}{\ensuremath {\boldsymbol {\xi}}}
\newcommand{\avbxi}{\overline{\ensuremath {\boldsymbol {\xi}}}}

% math cal

\newcommand{\J}{\mathcal{J}}
\newcommand{\K}{\mathcal{K}}
\newcommand{\cG}{\mathcal{G}}
\newcommand{\cF}{\mathcal{F}}
\newcommand{\cN}{\mathcal{N}}
\newcommand{\cL}{\mathcal{L}}


% san serif for matrices and differential operators
%\newcommand{\helmn}{\mathsf{H}_n}
\newcommand{\helmm}{\triangle_m}
\newcommand{\helmn}{\triangle_n}
\newcommand{\helms}{\triangle_s}
\newcommand{\helm}{\triangle}
\newcommand{\sA}{\mathsf{A}}
\newcommand{\sB}{\mathsf{B}}
\newcommand{\sG}{\mathsf{G}}
\newcommand{\sI}{\mathsf{I}}
\newcommand{\sJ}{\mathsf{J}}
\newcommand{\sU}{\mathsf{U}}
\newcommand{\sP}{\mathsf{P}}
\newcommand{\sQ}{\mathsf{Q}}
\newcommand{\sR}{\mathsf{R}}
\newcommand{\sL}{\mathsf{L}}
\renewcommand{\sJ}{\mathsf{J}}
\renewcommand{\sI}{\mathsf{I}}
\renewcommand{\L}{\mathsf{L}}
\newcommand{\sM}{\mathsf{M}}
\newcommand{\sT}{\mathsf{T}}
\newcommand{\sGamma}{\mathsf{\Gamma}}
\newcommand{\sOmega}{\mathsf{\Omega}}
\newcommand{\sSigma}{\mathsf{\Omega}}
\newcommand{\sbeta}{\mathsf{\beta}}
\newcommand{\sPi}{\mathsf{\Pi}}
\newcommand{\sC}{\mathsf{C}}
\newcommand{\sQy}{\mathsf{Q}}


% angle brackets

\def\la{\langle}
\def\ra{\rangle}
\def\laa{\left \langle}
\def\raa{\right \rangle}


%grads and div's
\newcommand{\bcdot}{\hspace{-0.1em} \boldsymbol{\cdot} \hspace{-0.12em}}
\newcommand{\bnabla}{\boldsymbol{\nabla}}
\newcommand{\bnablaH}{\bnabla_{\! \mathrm{h}}}
\newcommand{\grad}{\bnabla}
\newcommand{\gradH}{\bnablaH}
\newcommand{\curl}{\bnabla \!\times\!}
\newcommand{\diver}{\bnabla \bcdot }
\newcommand{\cross}{\times}
\newcommand{\lap}{\triangle}


%varthetas and thetas
\newcommand{\vth}{\vartheta}
\newcommand{\psii}{\psi^{\mathrm{i}}}
\newcommand{\thb}{\theta^{\mathrm{-}}}
\newcommand{\vthb}{\vartheta^{\mathrm{-}}}
\newcommand{\vthbhat}{{\hat{\vartheta}}^{\mathrm{-}}}
\newcommand{\vThb}{\varTheta^{\mathrm{-}}}
\newcommand{\psib}{\psi^{\mathrm{-}}}
\newcommand{\tht}{\theta^{\mathrm{+}}}
\newcommand{\vtht}{\vartheta^{\mathrm{+}}}
\newcommand{\vththat}{{\hat{\vartheta}}^{\mathrm{+}}}
\newcommand{\vthtbhat}{{\hat{\vartheta}}^{\pm}}
\newcommand{\vTht}{\varTheta^{\mathrm{+}}}
\newcommand{\vthtb}{\vartheta^{\pm}}
\newcommand{\vThtb}{\varTheta^{\pm}}

% nondimensional numbers
\renewcommand{\Re}{\mathrm{Re}}
\newcommand{\Ro}{\mathrm{Ro}}
\newcommand{\Ri}{\mathrm{Ri}}

%psi's
%Galerking coefficient for psi:
\newcommand{\gpsi}{\breve \psi}
\newcommand{\gphi}{\breve \phi}
\newcommand{\gq}{\breve q}
\newcommand{\gU}{\breve U}
\newcommand{\gQ}{\breve Q}
\newcommand{\gsigma}{\breve \sigma}


\newcommand{\psit}{\psi^{\mathrm{+}}}
\newcommand{\psithat}{{\hat{\psi}}^{\mathrm{+}}}
\newcommand{\psibhat}{{\hat{\psi}}^{\mathrm{-}}}
\newcommand{\psitb}{\psi^{\pm}}
\newcommand{\psitbhat}{{\hat{\psi}}^\pm}
\newcommand{\St}{S^{\mathrm{+}}}
\newcommand{\Sb}{S^{\mathrm{-}}}
\newcommand{\phb}{\phi^{\mathrm{-}}}
\newcommand{\pht}{\phi^{\mathrm{+}}}
\newcommand{\tautb}{\tau^{\pm}}
\newcommand{\sigmatb}{\sigma^{\pm}}


\newcommand{\bur}{\left(\tfrac{f_0}{N}\right)^2}
\newcommand{\ibur}{\left(\tfrac{N}{f_0}\right)^2}
\newcommand{\Nm}{N_{\mathrm{mix}}}
\newcommand{\xim}{\xi_{\mathrm{mix}}}
\newcommand{\hs}{h_*}
\renewcommand{\sp}{\mathsf{p}}
\newcommand{\se}{\mathsf{e}}
\newcommand{\sptb}{\mathsf{p}^\pm}


%nmax is a problem:
%\newcommand{\nmax}{n_{\mathrm{max}}}
\newcommand{\nmax}{\mathrm{N}}

\newcommand{\WKB}{\mathrm{WKB}}
\newcommand{\Lam}{\Lambda}
\newcommand{\tha}{\theta}
\newcommand{\kap}{\kappa}
\newcommand{\bphi}{\boldsymbol{\phi}}
\newcommand{\third}{\tfrac{1}{3}}
\newcommand{\cs}{c^\star}
\newcommand{\nt}{n^{\mathrm{trnc}}}
\newcommand{\sDp}{\mathsf{D}^1_{\nmax}}
\newcommand{\sDpp}{\mathsf{D}^2_{\nmax}}
\newcommand{\sD}{\mathsf{D_2}}
\newcommand{\sK}{\mathsf{K_2}}
\newcommand{\stheta}{\mathsf{\theta}}
\newcommand{\sphi}{\mathsf{\phi}}
\newcommand{\sq}{\mathsf{q}}
\newcommand{\cosech}{\text{csch}\,}
\newcommand{\sinc}{\text{sinc}\,}

%%%%%%%%% %%%%
\newcommand{\zp}{z^+}
\newcommand{\zm}{z^-}
\newcommand{\qA}{q^A_{\nmax}}
\newcommand{\psiB}{\psi^B_{\nmax}}
\newcommand{\phiB}{\phi^B_{\nmax}}
\newcommand{\eye}{\boldsymbol{\hat{i}}}
\newcommand{\jay}{\boldsymbol{\hat{j}}}
\newcommand{\kay}{\boldsymbol{\hat{k}}}
\newcommand{\psiG}{\psi^{\mathrm{G}}}
\newcommand{\qG}{q^{\mathrm{G}}}
\newcommand{\uG}{u^{\mathrm{G}}}
\newcommand{\UG}{U^{\mathrm{G}}}
\newcommand{\UGN}{U^{\mathrm{G}}_{\nmax}}
\newcommand{\QGN}{Q^{\mathrm{G}}_{\nmax}}
\newcommand{\sumoddn}{\sum_{n = 1, n~ \text{odd}}^{\nmax}}

% bretherton 
\newcommand{\qBr}{q_{\mathrm{Br}}}
\newcommand{\psiBr}{\psi_{\mathrm{Br}}}



\maketitle

\section*{Problem 1}

\begin{enumerate}[label=(\alph*)]

    \item We aim to derive the following expression for the turbulent dissipation rate
        \beq
            \label{eq:ep_target}
            \epsilon = 15 \nu \frac{u^2}{\lambda^2}\com
        \eeq
        where u is the root-mean square velocity and $\lambda$ is the Taylor microscale. 

    Following Pope's hint, we start with the fourth-order tensor
    \beq
        \left< \frac{\p u_i}{\p_{x_j}} \frac{\p u_k}{\p_{x_l}} \right>\com
    \eeq
    which is isotropic in homogeneous isotropic turbulence. Hence
    \beq
        \label{eq:most_gen_4ot}
        \left< \frac{\p u_i}{\p_{x_j}} \frac{\p u_k}{\p_{x_l}} \right> = \alpha \delta_{ij}\delta_{kl} 
                        + \beta \delta_{ik}\delta_{jl} + \gamma \delta_{il}\delta_{jk}\com
    \eeq
    where the right-hand side of \eqref{eq:most_gen_4ot} is simply the most general fourth-order
    isotropic tensor, with $\alpha$, $\beta$, and $\gamma$ scalars and $\delta_{pq}$ the Dirac delta.
     Continuity requires $\p u_m / \p x_{m} = 0$. Thus, with $i=j$ and $k=l$ we obtain
    \beq
        \left< \frac{\p u_i}{\p_{x_i}} \frac{\p u_k}{\p_{x_k}} \right> = \alpha \delta_{ii}\delta_{kk} 
                        + \beta \delta_{ik}\delta_{ik} + \gamma \delta_{ik}\delta_{ik} = 0\com
    \eeq
    and therefore
    \beq
        \label{eq:constraint}    
       3\alpha + \beta + \gamma = 0\per
    \eeq
    Note that statistical homogeneity implies
    \beq
        \frac{\p}{\p x_j}\left<u_i \frac{\p u_j}{\p x_l}\right> = 0\per
    \eeq
    Thus
    \beq
        \left< \frac{\p u_i}{\p x_j}\frac{\p u_j}{\p x_l} + u_i
        \frac{\p}{\p x_l}\underbrace{\frac{\p u_j}{\p x_j}}_{=0} \right>  = 0\com
    \eeq
    and therefore
    \beq
       \left< \frac{\p u_i}{\p x_j}\frac{\p u_j}{\p x_l}\right> = 0 \per
    \eeq
    Using \eqref{eq:most_gen_4ot} this implies that
    \beq
        \label{eq:deltas}
        \alpha \delta_{ij}\delta_{jl} + \beta \delta_{ij}\delta_{jl} + \gamma \delta_{il}\delta_{jl} = 0 \per
    \eeq
    The first two terms on the left of \eqref{eq:deltas} are non-zero only if $i = l$, and thus
    \beq
        \label{eq:constraint2}
        \alpha + \beta + 3 \gamma = 0\per
    \eeq
    With the two constraints \eqref{eq:constraint} and \eqref{eq:constraint2} we obtain
    \beq
        \alpha = \gamma = -\frac{\beta}{4}\com
    \eeq
    and therefore
    \beq
    \left< \frac{\p u_i}{\p_{x_j}} \frac{\p u_k}{\p_{x_l}} \right> = \beta \left( \delta_{ik}\delta_{jl} 
                    - \frac{1}{4} \delta_{ij}\delta_{kl} -  \frac{1}{4} \delta_{il}\delta_{jk}\right)\per
    \eeq
    Thus
    \beq
    \left<\left(\frac{\p u_1}{\p x_1}\right)^2\right> = \frac{1}{2} \beta\com
    \eeq
    \beq
    \left<\left(\frac{\p u_1}{\p x_2}\right)^2\right> = \beta = 2\left<\left(\frac{\p u_1}{\p x_1} 
     \right)^2\right> \com
    \eeq
    Thus the turbulent kinetic energy dissipation is
    \beq
        \epsilon \defn \nu \left< \frac{\p u_i}{\p x_j}\frac{\p u_i}{\p x_j} \right>  \per
    \eeq
    From the results above we have that, for homogeneous isotropic turbulence, the contribution from the 
     diagonal term is
     \beq
     \left<\frac{\p u_i}{\p x_j}\frac{\p u_i}{\p x_j}\right> = 3 \left<\left(\frac{\p u_1}{\p x_1}\right)^2\right>
     \com \qquad i = j\per
     \eeq
    Similarly, the contribution from the off-diagonal terms is
    \beq
          \left<\frac{\p u_i}{\p x_j}\frac{\p u_i}{\p x_j}\right> = 6\times 3 \left<\left(\frac{\p u_1}{\p x_1}\right)^2\right>
     \com \qquad i \neq j\per
    \eeq
    Collecting the results above we obtain an expression for the turbulence kinetic energy dissipation
    \beq
        \epsilon = 15 \nu  \left<\left(\frac{\p u_1}{\p x_1}\right)^2\right>
 \per        
    \eeq
    Now  given the longitudinal velocity autocorrelation $f(r,t)$, we have
    \begin{align}
    - u'^2 f''(0,t) = \lim_{r\to 0} \frac{\p^2}{\p r^2 }\frac{<u_1(\vec{x}+\vec{e}_1 r,t)u_1(\vec{x},t)>}{u'^2}
     = - \lim_{r\to0}\left<\left(\frac{\p^2 u_1}{\p x_1^2}\right)_{\vec{x}\vec{e_1}r}u_1(\vec{x},t)\right> \\
     -\left<\left(\frac{\p^2 u_1}{\p x_1^2}\right)u_1\right> = -\left<\frac{\p}{\p x_1}\left(u_1\frac{\p u_1}{\p x_1}\right)-\left(\frac{\p u_1}{\p x_1}\right)^2\right>
      \com
  \end{align}
    where $u'$ is the root-mean square velocity. Using statistical homogeneity we obtain
    \beq
        \label{eq:neat1}
        - u'^2 f''(0,t) = \left<\left(\frac{\p u_1}{\p x_1}\right)^2\right>\per
    \eeq
    The longitudinal Taylor microscale is the scale in which a parabola that osculates $f$ crosses the axis
    \beq
        \label{eq:lamf}
    \lambda_f^2 \defn \frac{1}{ 2 f''(0)}\per
    \eeq
    In complete analogy, the transverse Taylor microscale is defined as
    \beq
        \label{eq:lamg}    
    \lambda_g^2 \defn \frac{1}{ 2 g''(0)}\per
    \eeq
    In isotropic turbulence these are simply related
    \beq
        \label{eq:lamfg}    
    \lambda_g^2 = \frac{\lambda_g^2}{2}\per
    \eeq
    Thus, using \eqref{eq:neat1} through \eqref{eq:lamfg}, the turbulent kinetic energy dissipation is
    \beq
    \label{eq:ep}
    \epsilon = 15 \nu \frac{u'^2}{\lambda_g^2}\com
    \eeq
    which is the advertised answer. 


\end{enumerate}


\section*{Problem 2}

\begin{enumerate}[label=(\alph*)]

        \item Kolmogorov's $k^{-5/3}$ spectral law is a prediction for kinetic energy spectrum in the inertial
             subrange. This is the subrange encompasses scales between the energy containing range and the 
              dissipation range. In the inertial range, the kinetic energy spectrum is simply determined by the
               turbulent dissipation rate (which, in this range, equals the kinetic energy transfer rage). Simple dimension analysis
                gives the $\tfrac{5}{3}$ power dependency on the wavenumber. Now, for the inertial subrange to exist, 
                 there should be a clear separation between energy-containing and dissipation scales, and therefore
                  a sufficiently high Reynolds number is required for that law to be valid.
              \item If the hot wire measure at enough high-frequency, then place than at a distance $\Delta x_1$ downstream, and measure the velocity at these two positions. Calculating statistics for various $\Delta x_1$, one can estimate the longitudinal spatial autocorrelation, and then compute the Taylor microscales ($\lambda$s) and the root-mean square velocity. Using these, one can estimate the turbulent energy dissipation rate $\epsilon$ using energy dissipations using \ref{eq:ep}.
                  
                  A straightforward alternate approach is to measure the flow in a single fixed point, and calculate its temporal autocorrelation. One can then use Taylor's hypothesis to approximate the spatial autocorrelation from the temporal autocorrelation. From the temporal data one can also estimate the root-mean square velocity, and use it together with $\lambda_g$ to obtain an estimate for the turbulent energy dissipations using \ref{eq:ep}. Essentially, this hypothesis assumes that a patch of turbulence is being advected by the energy-containing turbulent flows. Over the time interval $\Delta t$ interval this patch will have traveled a distance of $U \Delta t$. For this hypothesis to be valid, the turbulence field must remain unchanged over $\Delta t$. One can then simply relate the temporal autocorrelation to the spatial autocorrelation.


\end{enumerate}

\section*{Problem 3}


%        \begin{figure}[ht]
%        \begin{center}
%        \includegraphics[width=18pc,angle=0]{pb2_cdf.png}
%        \includegraphics[width=18pc,angle=0]{pb2_pdf.png}
%        \end{center}
%        \caption{Cumulative distribution function (pdf) and probability distribution function (pdf) 
%                for a log-normally-distributed random variable $y$ with $<y>=1$, and different variances.}
%        \label{fig:pb2}
%        \end{figure}

\end{document}


